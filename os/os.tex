\documentclass[a4paper]{article}
% Русский язык
\usepackage[utf8]{inputenc}
\usepackage[russian]{babel}
% Математические символы и формулы
\usepackage{mathtools}
% Разбиение на разные файлы
\usepackage{subfiles}
% Отступы с краев(margin)
\usepackage[margin=1in]{geometry}
% Форматирование заголовков
\usepackage{titlesec}
% Красная строка
\usepackage{indentfirst}

\begin{document}
\section{Лекция 1. Введение}

\subsubsection{Режимы работы процессора}
\begin{enumerate}
\item режим ядра - защита на аппаратном уровне от любых попыток внесения изменений со стороны пользователя
\item режим пользователя - большой объем, сложная структура и длительные сроки испльзования
\end{enumerate}

\subsubsection{Функции ОС}
\begin{enumerate}
\item Предоставление прикладным программам понятного абстрактного набора ресурсов взамен неупорядоченного набора аппаратного обеспечения
\item Управление этими ресурсами
\end{enumerate}

Операционная система - комплекс системных управляющих и обрабатывающих программ предназначенных для предоставления интерфейса между аппаратурой кмопьютера и пользователем с его задачами, а также для управления эффективным расходованием ресурсов вычислительной системы и организации надежных вычислений.

\section{Лекция 2}

\subsubsection{Первое поколение - электронные лампы 1945-1955}
\begin{enumerate}
\item Джон Атанасов и Клиффорд Берри
\item Конрад Цузе - Z3
\item Алан Тьюринг и Colossus 1944-1946 гг.
\item Говард Айкен и Mark-I
\item Уильям Моучли и Джон Преспер Эккерт ENIAC
\end{enumerate}

\subsubsection{Второе поколение - транзисторы 1955-1965}
IBM 1401,
IBM 7094.
Ранняя система пакетной обработки

Типичные ОС
\begin{enumerate}
\item Fortran Monitor System
\item IBSYS - ОС компьютера IBM 7094
\end{enumerate}

\subsubsection{третье поколение 1965-1980 интегральные схемы и многозадачность}
IBM System/360

\begin{enumerate}
\item OS/360
\item PCP - однозадачная ОС 1966 г. Макс память 128 кб RAM
\item MFT - Мультипрограмирование с фиксированным числом задач
\item MULTIX - мультиплексная информационная вычислительная служба. Разделение времени, разделение памяти
\item Кен томпсон и Денис Ритчи - UNIX
\end{enumerate}

\subsubsection{Четвертое поколение 1980-Наши Дни}
\begin{enumerate}
\item Гэри Килдэлл и CP/M
\item Билл Гейтс MS-DOS, Windows
\end{enumerate}

\subsubsection{Пятое поколение. Мобильные ОС.}
Palm OS
Symbian
............

\subsection{Советские компьютеры}
MISS, Демос

\section{Лекция 3}
Операционные системы мейнфреймов
\subsubsection{Виды обслуживания}
Пакетная Обработка
Обработка транзакций
Работа в режими разделения времени

Пример OS/390


\subsubsection{Серверные ОС}
\begin{enumerate}
\item Solaris
\item FreeBSD
\item Linux
\item Windows Server 201x
\end{enumerate}

\subsubsection{Многопроцессорные ОС}
\subsubsection{ОС для встраиваемых систем}
\subsubsection{ОС реального времени(Мягкого/Жесткого). eCos}


\section{Лекция 4. Основные понятия и определения}
Прерывания - механизм позволяющий координировать параллельное функционирование отдельных устройств вычислительной системы и реагировать на особые состояния возникающие при работе процессора. 
Обработка прерываний
    установление факта прерывания и идентификация прерывания
    и идентификация прерывания
    запоминание состояния прерванного процесса вычислений
    сохранниния информации о прерванной программе
    выполнение программы обработчика прерывания
    восстановление прерывания
    возврат на прерванную программу

\subsection{Процесс}
Процесс - программа во время ее выполнения. 
    Ресурсы связанные с этой программой. Регистры счетчик комманд и указатель стека.
    список открытых файлов
    необработанные предупреждения
    список связанных процессов
    служебная информация

Дочерний процесс - процесс порожденный другим процессом
Межпроцессное взаимодейстивие(IPC)

UID - User Identifier
GID - Group Identifier

\subsection{Адресное пространство}
Список адресов ячеек памяти от нуля до некоторого максимума, откуда процесс может считывать и записывать данные 
Содержит выполняемую программу, данные программы и ее стек.

\subsection{Файл}
Именованый набор данных организованных в виде совокупности записей одинаковой структуры.


\subsection{Ввод/Вывод данных}
У каждой ОС своя подсистема ввода-вывода
Для некоторых устройств используются драйверы

\subsection{Безопасность}
Обеспечение доступа к файлам только пользователям имеющим на это право. Защита системы от нежелательных вторжений

\section{Архитектура ОС}

\subsection{Состав ОС}

\begin{enumerate}
\item Исполняемые и объектные модули стандартных для данной ОС форматов.
\item Различные библиотеки.
\item Модули исходного текста программ
\item Модули специального формата. (Драйверы, загрузчик)
\end{enumerate}

Ядро - основные функции ОС.
Модули, выполняющие вспомогательные функции

\subsubsection{Состав ядра ОС}
\begin{enumerate}
\item Управление процессами
\item Управление памятью
\item Упавление устройствами IO
\item Решение внутрисистемных задач организации вычислительного процесса
\end{enumerate}
\subsubsection{Нечеткость границы между ОС и приложениями}

\subsubsection{Вспомогательные модули}
\begin{enumerate}
\item Утилиты 
\item Системные обрабатывающие программы
\item Программы представления пользователю доп.услуг
\item Библиотеки процедур различного назначения
\end{enumerate}
\subsubsection{Режимы работы процессора}
    Пользовательский, Привилегированный, Виртуализация....

\subsection{Многослойность структуры ОС}
Утилиты системные обрабатывающие программы -> Ядро -> Аппаратура.

Аппаратура

Средства аппаратной поддержки ОС
    средства поддержки привилегированного режима. 
    cистема прерываний
    cистема переключения контекста процесса
    средства защиты областей памяти
    системный таймер
    средства трансляции адресов
Машинно зависимые компоненты
Машинно зависимые модули
Базовые механизмы ядра
    Диспетчеризация прерываний
    Перемещение страниц из памяти на диск и обратно
Менеджеры ресурсов
    Менеджер процессов
    Менеджер памяти
    Менеджер файловой системы
    Менеджер ввода-вывода
Интерфейс системных вызовов
    Непосредственно взаимодействует с приложениями. Образует прикладной програмный интерфейс API.

Архитектуры ОС
Микроядро -> Медленное
Монолитное Ядро
Гибридное ядро(Современные ОС)
Виртуальные машины
Экзоядро
Клиент-Серверная модель

Основные виды системных вызовов
Управление процессами
Управление файлами
Управление устройствами
Сопровождающая информационная
Коммуникация

os.ispu.ru

\section{Диспетчеризация процессов}
Диспетчеризация процессора - это распределение его времени между процессами в системе.
Цель диспетчеризации - максимальная загрузка процессора. 
Исполнение процесса можно рассматривать как цикл CPU/IO
Поскольку процессор постоянно переключается между процессами, поэтому нельзя заранее просчитать скорость исполнения процесса.
Диспетчер процессора - компонент операционной системы предоставляющий процессор тому процессу, который был выбран планировщиком.
Латентность диспетчера - скрытая активность диспетчера, время требуемое диспетчером для остановки одного процесса и стартовать другой. dispatch latency.


Критерии диспетчеризации.
Использование процессора - поддержание процессора в режиме занятости максимальное время. CPU utilization
Пропускная способоность системы. Число процессов завершающее свое выполнение за единицу времени. throughput
Время обработки процесса, время необходимое для исполнения какого-либо процесса. turnaround time
Время ожидания. waiting time. Время которое процесс ждет в очереди процессов готовых к выполнению.
Время ответа. Response time. Время требуемое от первого запроса до первого ответа.

First Come First Served
Shortest Job First 
    Без прерывания процессов
    С прерывание процессов
Диспетчеризация по приоритетам
    С каждым из процессов ассоциируется число, которое определяет его приоритет. 
    Процессор выдается процесссу с наивысшим приоритетом.
    Возникающие проблемы
        эффект голодания
        учет возраста процесса
    
Round Robin

Многоуровневые очереди 
    Процессы реального времени, требуют такого планирования, чтобы гарантировать окончание процесса за конкретное время или к конкретному моменту времени. 
    Интерактивные Процессы, время не больше допустимой реакции на запрос пользователя.
    Пакетные Процессы, время существования практически не ограничивается
    \hline
    Системные Процессы
    Пользовательские Процессы

Основная очередь RR
Фоновая очередь

Виды диспетчеризации между очередями
- с фиксированным приоритетом
- выделение отрезка времени
Какждая очередь получает выделенный отрезок времени ЦП, который может разделять между процессами.


Многоуровневые аналитические очереди
    Классы выполнения
    


\end{document}