\documentclass[a4paper]{article}
% Русский язык
\usepackage[utf8]{inputenc}
\usepackage[russian]{babel}
% Математические символы и формулы
\usepackage{mathtools}
% Разбиение на разные файлы
\usepackage{subfiles}
% Отступы с краев(margin)
\usepackage[margin=1in]{geometry}
% Форматирование заголовков
\usepackage{titlesec}
% Красная строка
\usepackage{indentfirst}

\begin{document}
\section{Лекция 1. Введение}

\subsubsection{Режимы работы процессора}
\begin{enumerate}
\item режим ядра - защита на аппаратном уровне от любых попыток внесения изменений со стороны пользователя
\item режим пользователя - большой объем, сложная структура и длительные сроки испльзования
\end{enumerate}

\subsubsection{Функции ОС}
\begin{enumerate}
\item Предоставление прикладным программам понятного абстрактного набора ресурсов взамен неупорядоченного набора аппаратного обеспечения
\item Управление этими ресурсами
\end{enumerate}

Операционная система - комплекс системных управляющих и обрабатывающих программ предназначенных для предоставления интерфейса между аппаратурой кмопьютера и пользователем с его задачами, а также для управления эффективным расходованием ресурсов вычислительной системы и организации надежных вычислений.

\section{Лекция 2}

\subsubsection{Первое поколение - электронные лампы 1945-1955}
\begin{enumerate}
\item Джон Атанасов и Клиффорд Берри
\item Конрад Цузе - Z3
\item Алан Тьюринг и Colossus 1944-1946 гг.
\item Говард Айкен и Mark-I
\item Уильям Моучли и Джон Преспер Эккерт ENIAC
\end{enumerate}

\subsubsection{Второе поколение - транзисторы 1955-1965}
IBM 1401,
IBM 7094.
Ранняя система пакетной обработки

Типичные ОС
\begin{enumerate}
\item Fortran Monitor System
\item IBSYS - ОС компьютера IBM 7094
\end{enumerate}

\subsubsection{третье поколение 1965-1980 интегральные схемы и многозадачность}
IBM System/360

\begin{enumerate}
\item OS/360
\item PCP - однозадачная ОС 1966 г. Макс память 128 кб RAM
\item MFT - Мультипрограмирование с фиксированным числом задач
\item MULTIX - мультиплексная информационная вычислительная служба. Разделение времени, разделение памяти
\item Кен томпсон и Денис Ритчи - UNIX
\end{enumerate}

\subsubsection{Четвертое поколение 1980-Наши Дни}
\begin{enumerate}
\item Гэри Килдэлл и CP/M
\item Билл Гейтс MS-DOS, Windows
\end{enumerate}

\subsubsection{Пятое поколение. Мобильные ОС.}
Palm OS
Symbian
............

\subsection{Советские компьютеры}
MISS, Демос

\section{Лекция 3}
Операционные системы мейнфреймов
\subsubsection{Виды обслуживания}
Пакетная Обработка
Обработка транзакций
Работа в режими разделения времени

Пример OS/390

Серверные ОС
Solaris
FreeBSD
Linux
Windows Server 201x

Многопроцессорные ОС
ОС для встраиваемых систем
Операционные системы реального времени (Мягкого/Жесткого) eCos
\end{document}