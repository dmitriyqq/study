\documentclass[a4paper]{article}
\usepackage[utf8]{inputenc}
\usepackage[russian]{babel}
\usepackage[margin=1in]{geometry}
\usepackage{float}
\usepackage{graphicx}
\usepackage{setspace}
\usepackage{indentfirst}
\title{Лабораторная работа}
\author{Шулайкин Д.А.}
\begin{document}
\onehalfspacing
\thispagestyle{empty}

\begin{center}
Министерство образования и науки Российской Федерации
\vspace{10pt}

Федеральное государственное бюджетное образовательное учереждение высшего образования "Ивановский государственный энергетический университет имени В.И. Ленина"
\vspace{40pt}

Кафедра ПОКС
\vspace{40pt}

\textbf{Отчет по лабораторной работе №1}
Дисциплина: ``БАЗЫ ДАННЫХ''
Тема: ``КОНЦЕПТУАЛЬНАЯ МОДЕЛЬ БАЗЫ ДАННЫХ''
\end{center}

\vspace{330pt}

\begin{flushright}
\textbf{Выполнили:} \\
ст. гр. 2-42В \\
Шулайкин Д.А. \\

\textbf{Проверил:} \\
Гурфова О.М.
\end{flushright}

\vspace{40pt}

\begin{center}
	Иваново 2019
\end{center}

\pagebreak

\section{Задание}
Аквапарк предлагает своим клиентам отдых и развлечения в трех зонах: аквапарк, сауны, кафетерий. Зоны различаются стоимостью одной минуты пребывания в них(для кафетерия - 0 руб.) и стоимостью дневного абонемента.

Клиенты расплачиваются посредством магнитных браслетов. Каждый браслет хранит количество минут, проведенных клиентом в каждой зоне, сумму, потраченную в кафетерии и величину залога, внесенного при входе. На выходе все траты пересчитываются и посетителю возвращают разницу или берут с него доплату.

\section{Цель работы}
Построить концептуальную модель.

\section{Ход работы}

%create table Customers
% (
% 	id bigint identity
% 		constraint Customers_pk
% 			primary key nonclustered,
% 	balance bigint not null
% )
%

\section{Вывод}
В ходе работы была построена концептуальная модель.

\end{document}