\documentclass[a4paper]{article}
% Русский язык
\usepackage[utf8]{inputenc}
\usepackage[russian]{babel}
% Математические символы и формулы
\usepackage{mathtools}
% Разбиение на разные файлы
\usepackage{subfiles}
% Отступы с краев(margin)
\usepackage[margin=1in]{geometry}
% Форматирование заголовков
\usepackage{titlesec}
% Красная строка
\usepackage{indentfirst}

\begin{document}
\section{Лекция 1. Введение}

\subsubsection{Режимы работы процессора}
\begin{enumerate}
\item режим ядра - защита на аппаратном уровне от любых попыток внесения изменений со стороны пользователя
\item режим пользователя - большой объем, сложная структура и длительные сроки испльзования
\end{enumerate}

\subsubsection{Функции ОС}
\begin{enumerate}
\item Предоставление прикладным программам понятного абстрактного набора ресурсов взамен неупорядоченного набора аппаратного обеспечения
\item Управление этими ресурсами
\end{enumerate}

Операционная система - комплекс системных управляющих и обрабатывающих программ предназначенных для предоставления интерфейса между аппаратурой кмопьютера и пользователем с его задачами, а также для управления эффективным расходованием ресурсов вычислительной системы и организации надежных вычислений.

\section{Лекция 2}

\subsubsection{Первое поколение - электронные лампы 1945-1955}
\begin{enumerate}
\item Джон Атанасов и Клиффорд Берри
\item Конрад Цузе - Z3
\item Алан Тьюринг и Colossus 1944-1946 гг.
\item Говард Айкен и Mark-I
\item Уильям Моучли и Джон Преспер Эккерт ENIAC
\end{enumerate}

\subsubsection{Второе поколение - транзисторы 1955-1965}
IBM 1401,
IBM 7094.
Ранняя система пакетной обработки

Типичные ОС
\begin{enumerate}
\item Fortran Monitor System
\item IBSYS - ОС компьютера IBM 7094
\end{enumerate}

\subsubsection{третье поколение 1965-1980 интегральные схемы и многозадачность}
IBM System/360

\begin{enumerate}
\item OS/360
\item PCP - однозадачная ОС 1966 г. Макс память 128 кб RAM
\item MFT - Мультипрограмирование с фиксированным числом задач
\item MULTIX - мультиплексная информационная вычислительная служба. Разделение времени, разделение памяти
\item Кен томпсон и Денис Ритчи - UNIX
\end{enumerate}

\subsubsection{Четвертое поколение 1980-Наши Дни}
\begin{enumerate}
\item Гэри Килдэлл и CP/M
\item Билл Гейтс MS-DOS, Windows
\end{enumerate}

\subsubsection{Пятое поколение. Мобильные ОС.}
Palm OS
Symbian
............

\subsection{Советские компьютеры}
MISS, Демос

\section{Лекция 3}
Операционные системы мейнфреймов
\subsubsection{Виды обслуживания}
Пакетная Обработка
Обработка транзакций
Работа в режими разделения времени

Пример OS/390


\subsubsection{Серверные ОС}
\begin{enumerate}
\item Solaris
\item FreeBSD
\item Linux
\item Windows Server 201x
\end{enumerate}

\subsubsection{Многопроцессорные ОС}
\subsubsection{ОС для встраиваемых систем}
\subsubsection{ОС реального времени(Мягкого/Жесткого). eCos}


\section{Лекция 4. Основные понятия и определения}
Прерывания - механизм позволяющий координировать параллельное функционирование отдельных устройств вычислительной системы и реагировать на особые состояния возникающие при работе процессора. 
Обработка прерываний
\begin{enumerate}
    \item установление факта прерывания и идентификация прерывания и идентификация прерывания
    \item запоминание состояния прерванного процесса вычислений
    \item сохранниния информации о прерванной программе
    \item выполнение программы обработчика прерывания
    \item восстановление прерывания
    \item возврат на прерванную программу
\end{enumerate}

\subsection{Процесс}
Процесс - программа во время ее выполнения. 
Процесс состоит из: 
\begin{enumerate}
\item Ресурсы связанные с этой программой. 
\item Регистры счетчик комманд и указатель стека.
\item Список открытых файлов
\item Необработанные предупреждения
\item Список связанных процессов
\item Служебная информация
\end{enumerate}

Дочерний процесс - процесс порожденный другим процессом
Межпроцессное взаимодейстивие(IPC)

UID - User Identifier
GID - Group Identifier

\subsection{Адресное пространство}
Список адресов ячеек памяти от нуля до некоторого максимума, откуда процесс может считывать и записывать данные 
Содержит выполняемую программу, данные программы и ее стек.

\subsection{Файл}
Именованый набор данных организованных в виде совокупности записей одинаковой структуры.

\subsection{Ввод/Вывод данных}
У каждой ОС своя подсистема ввода-вывода
Для некоторых устройств используются драйверы

\subsection{Безопасность}
Обеспечение доступа к файлам только пользователям имеющим на это право. Защита системы от нежелательных вторжений

\section{Архитектура ОС}

\subsection{Состав ОС}

\begin{enumerate}
\item Исполняемые и объектные модули стандартных для данной ОС форматов.
\item Различные библиотеки.
\item Модули исходного текста программ
\item Модули специального формата. (Драйверы, загрузчик)
\end{enumerate}

Ядро - основные функции ОС.
Модули, выполняющие вспомогательные функции

\subsubsection{Состав ядра ОС}
\begin{enumerate}
\item Управление процессами
\item Управление памятью
\item Упавление устройствами IO
\item Решение внутрисистемных задач организации вычислительного процесса
\end{enumerate}
\subsubsection{Нечеткость границы между ОС и приложениями}

\subsubsection{Вспомогательные модули}
\begin{enumerate}
\item Утилиты 
\item Системные обрабатывающие программы
\item Программы представления пользователю доп.услуг
\item Библиотеки процедур различного назначения
\end{enumerate}
\subsubsection{Режимы работы процессора}
    Пользовательский, Привилегированный, Виртуализация....

\subsection{Многослойность структуры ОС}
Утилиты системные обрабатывающие программы -> Ядро -> Аппаратура.

\subsubsection{Аппаратура}
Состоит из
\begin{enumerate}
\item Средства аппаратной поддержки ОС:
    \begin{enumerate}
        \item средства поддержки привилегированного режима. 
        \item cистема прерываний
        \item cистема переключения контекста процесса
        \item средства защиты областей памяти
        \item системный таймер
        \item средства трансляции адресов
    \end{enumerate}
\item Машинно зависимые компоненты
\item Машинно зависимые модули
\item Базовые механизмы ядра
    
\item Диспетчеризация прерываний
\item Перемещение страниц из памяти на диск и обратно
\item Менеджеры ресурсов
\begin{enumerate}
    \item Менеджер процессов
    \item Менеджер памяти
    \item Менеджер файловой системы
    \item Менеджер ввода-вывода
\end{enumerate}
\item Интерфейс системных вызовов - непосредственно взаимодействует с приложениями. Образует прикладной програмный интерфейс API.
\end{enumerate}

!TODO .... Другие слои

\subsection{Архитектуры ОС}
\subsubsection{Микроядро -> Медленное}
\subsubsection{Монолитное Ядро}
\subsubsection{Гибридное ядро(Современные ОС)}
\subsubsection{Виртуальные машины}
\subsubsection{Экзоядро}
\subsubsection{Клиент-Серверная модель}

\subsection{Основные виды системных вызовов}
\begin{enumerate}
    \item Управление процессами
    \item Управление файлами
    \item Управление устройствами
    \item Сопровождающая информационная
    \item Коммуникация
\end{enumerate}

% Материалы os.ispu.ru

\section{Диспетчеризация процессов}
\subsection{Диспетчеризация процессора}
\textbf{Диспетчеризация процессора} - это распределение его времени между процессами в системе.
Цель диспетчеризации - максимальная загрузка процессора. 
Исполнение процесса можно рассматривать как цикл CPU/IO
Поскольку процессор постоянно переключается между процессами, поэтому нельзя заранее просчитать скорость исполнения процесса.
Диспетчер процессора - компонент операционной системы предоставляющий процессор тому процессу, который был выбран планировщиком.
\textbf{Латентность диспетчера} - скрытая активность диспетчера, время требуемое диспетчером для остановки одного процесса и стартовать другой. (dispatch latency)


\textbf{Критерии диспетчеризации}
\begin{enumerate}
    \item Использование процессора - поддержание процессора в режиме занятости максимальное время. CPU utilization
    \item Пропускная способоность системы. Число процессов завершающее свое выполнение за единицу времени. throughput
    \item Время обработки процесса, время необходимое для исполнения какого-либо процесса. turnaround time
    \item Время ожидания. waiting time. Время которое процесс ждет в очереди процессов готовых к выполнению.
    \item Время ответа. Response time. Время требуемое от первого запроса до первого ответа.
\end{enumerate}

\subsection{Алгоримты shcheduler'a}
\subsubsection{First Come First Served}
\subsubsection{Shortest Job First}
    Существует без прерывания процессов, с прерывание процессов
\subsubsection{Диспетчеризация по приоритетам}
    С каждым из процессов ассоциируется число, которое определяет его приоритет. 
    Процессор выдается процесссу с наивысшим приоритетом.
    Возникающие проблемы: эффект голодания, возможное решение - учет возраста процесса.
\subsubsection{Round Robin}

\subsubsection{Многоуровневые очереди}
Процессы делятся на виды:
\begin{enumerate}
    Процессы реального времени, требуют такого планирования, чтобы гарантировать окончание процесса за конкретное время или к конкретному моменту времени. 
    Интерактивные Процессы, время не больше допустимой реакции на запрос пользователя.
    Пакетные Процессы, время существования практически не ограничивается    
\end{enumerate}

\begin{enumerate}
    Системные Процессы
    Пользовательские Процессы
\end{enumerate}

Основная очередь RR, Фоновая очередь
Виды диспетчеризации между очередями
- с фиксированным приоритетом
- выделение отрезка времени
Какждая очередь получает выделенный отрезок времени ЦП, который может разделять между процессами.
Многоуровневые аналитические очереди
Классы выполнения
TODO fix this i do not remember
    
\section{Синхронизации процессов}
Атомарная операция это такая операция, которая должна быть выполнена полностью без каких либо прерываний.

\textbf{Race condition} - конкуренция за общие данные. Ситуация в которой взаимодействубщие процессы могут одновременно обращаться к общим данным.

\subsection{Проблема синхронизации процессов решается различными вариантами.}

Синхронизация по критическим сессиям
Критическая сессия это фрагмент кода в каждом процессе в котором происходит обращение к общим данным.

Для решения проблемы по критическим сессии необходимо три условия:

\subsubsection{Взаимное исключение}
Если процесс исполняет свою критическую сессию, то никакой другой процесс не должен исполнять свою.

\subsubsection{Прогресс}
Если в данный момент нет процессов, исполняющих критическую секцию, но есть несколько процессов, желающих начать исполнение критической сессии, то выбор системой процесса, которому будет разрешен запуск критической сессии не может продолжаться бесконечно.

\subsubsection{Ограниченное ожидание}
В системе должно существовать ограничение на число раз, которое процессам разрешено входить в свои критические секции, начиная от момента, когда некоторые процессы сделали запрос о входе в некоторую критическую  секцию и до момента, когда этот запрос удовлетворен.

\subsection{Алгоритмы решения проблемы критической секции}

\subsubsection{Алгоритм булочной. Алгоритм Л. Лэмпорта}
sleep wait, wakeup signal

\subsection{Классические задачи}
\begin{enumerate}
    \item Ограниченный буфер
    \item Читатели-Писатели. В каждый момент времени, может работать несколько читателей. Но не более одного писателя.
    \item Обедающие философы
\end{enumerate}


\subsection{Мьютексы}
\textbf{Мьютекс} - это совместно используемая переменная которая может находится в одном из двух состояний

mutex_lock
mutex_unlock


\end{document}