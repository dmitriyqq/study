\documentclass[a4paper]{article}
\usepackage[utf8]{inputenc}
\usepackage[russian]{babel}
\usepackage[margin=1in]{geometry}
\usepackage{float}
\usepackage{graphicx}
\usepackage{setspace}
\usepackage{indentfirst}
\usepackage{subfiles}
\usepackage{textcomp}
\usepackage{xcolor,listings}

\lstset{upquote=true}

\title{Лабораторная работа}
\author{Шулайкин Д.А.}
\begin{document}
\onehalfspacing
\thispagestyle{empty}

\begin{center}
Министерство образования и науки Российской Федерации
\vspace{10pt}

Федеральное государственное бюджетное образовательное учереждение высшего образования "Ивановский государственный энергетический университет имени В.И. Ленина"
\vspace{40pt}

Кафедра ПОКС
\vspace{40pt}

\textbf{Отчет по курсовой работе}
Дисциплина: ``БАЗЫ ДАННЫХ''
Тема: ``БАЗЫ ДАННЫХ''
\end{center}

\vspace{330pt}

\begin{flushright}
\textbf{Выполнили:} \\
ст. гр. 2-42В \\
Шулайкин Д.А. \\

\textbf{Проверил:} \\
Гурфова О.М.
\end{flushright}

\vspace{40pt}

\begin{center}
	Иваново 2019
\end{center}

\pagebreak

\section{Задание}
Аквапарк предлагает своим клиентам отдых и развлечения в трех зонах: аквапарк, сауны, кафетерий.
Зоны различаются стоимостью одной минуты пребывания в них(для кафетерия - 0 руб.) и стоимостью дневного абонемента.
Клиенты расплачиваются посредством магнитных браслетов. 
Каждый браслет хранит количество минут, проведенных клиентом в каждой зоне, сумму, 
потраченную в кафетерии и величину залога, внесенного при входе.
На выходе все траты пересчитываются и посетителю возвращают разницу или берут с него доплату.

\section{Создание базы данных}

\subsection{Описание сущностей}

\subsection{Концептуальная модель}

\subsection{Логическая модель}

\subsection{Физическая модель}

\subsection{DDL скрипты}

\section{Данные}

\section{Запросы}

% Задание:
% 7-10 запросов:
% - объединение таблиц
% -с условием WHERE
% -с группировкой
% -с условием HAVING
% -с агрегатными функциями
% -подзапросы : с EXISTS и объединение UNION
% -оператор выбора CASE

\subsection{Запросить все счета}

\subsection{Запрос с объединением таблиц}

\subsection{Запрос с условием}

\subsection{Запрос с группировкой}

\subsection{Запрос с агрегатными функциями}

\subsection{Запрос с подзапросом}

\subsection{Запрос с оператором выбора}

\section{View}

\section{Триггеры}

\section{Вывод}

\end{document}