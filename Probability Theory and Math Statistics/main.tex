\documentclass[a4paper]{article}
% Русский язык
\usepackage[utf8]{inputenc}
\usepackage[russian]{babel}
% Математические символы и формулы
\usepackage{mathtools}
\usepackage{float}
% Разбиение на разные файлы
\usepackage{subfiles}
% Отступы с краев(margin)
\usepackage[margin=1in]{geometry}
% Форматирование заголовков
\usepackage{titlesec}
\usepackage{indentfirst} % Красная строка
% Схемки
% \usepackage{circuitikz}
% Картинки
\usepackage{graphicx}
\usepackage{breqn}

\begin{document}
\section{Теория вероятностей}
Теория вероятностей - это математическая наука изучающая закономерности случайных явлений.

Математическая статистика - раздел математики, изучающий 
методы сбора, систематизации и обработки результатов наблюдений с целью выявления закономерности.

\section{Основы комбинаторики}

Факториал $n!$- произведение первых n сомножителей.

$$ n! = 1*2*3*...*n $$

Основное свойство факториала

$$ n! = (n-1)! * n $$

Размещение
Размещение из $n$ элементов называется такое соединение по k элементов, которые отличаются друг от друга Кирилл блять кто так пишет.

$$ A_n^k = \frac{n!}{(n-k)!} $$

Перестановки

Перестановками из $n$ элементов называется их соединение, отличающееся от других только порядком входящих в них элементов.
Число перестановок из $n$ элементов: $P_n = n!$

Если среди $n$ элементов (a, b, c) имеются одинаковые (a - $\alpha$ раз, b - $\beta$ раз, c - $\gamma$ раз...) то количество перестановок:

$$ P_n = \frac{n!}{\alpha!\beta!\gamma!}$$

Сочетанием из $n$ элементов по $K$ называются их соединения отличающиеся друг от друга только самими элементами.

$$ C_n^k = \frac{n!}{k!*(n-k)!} $$

Основное свойство:

$$ C_n^k = C_n^{n-k}$$

Основной закон комбинаторики

Пусть нужно провести $K$ действий, причем первое действие можно провести $n_1$ способами, второе - $n_2$ способами, ... k-ое - $n_k$ способами. Тогда все действия можно провести 
$ n_1 = n_2 * ... * n_k$ способами.

\section{Основные понятия теории вероятностей}

Комбинаторное событие можно разделить на 3 вида
\begin{enumerate}
\item достоверное
\item невозможное
\item случайное
\end{enumerate}

Событие называется \textbf{достоверным}, если оно обязательно произойдет при выполнении данного ряда условий.

\textbf{невозможное} если оно заведомо не произойдет при выполнении данного ряда условий.

\textbf{испытание} - осуществление ряда условий. 

\textbf{несовместные события} - если появление одного события исключает появление других в одном и том же испытании. 

\textbf{единственно-возможное событие}

\textbf{равновозможные события}

\textbf{Элементарный исход} - каждый из возможных результатов испытания.

\textbf{Полной группой} - называется совокупность ед. возможных событий испытания. \textbf{Противоположными} называются два единственно возможных события образующих единую группу. 

Сумму нескольких событий называют событие, которое состоит в появлении хотя бы одного из этих событий.

Вероятность события $A$ называют отношение числа благоприятных исходов к общему числу всех исходов испытания, при условии что все исходы равновозможны. 

$$ P(A) = \frac{m}{n} $$

\textit{Свойства}

\begin{enumerate}
    \item вероятность события - это величина от $1.0$ до $0.0$.
    \item вероятность невозможного события равна $0.0$
    \item вероятность достоверного события равна $1.0$
    \item вероятность суммы несовмесных событий равна сумме вероятностей этих событий. 
\end{enumerate}

Задачи на классическое определение вероятности описываются общей схемой:

\emph{Имеется совокупность из $n_1$ элементов перовго вида, $n_2$ второго вида. Какова вероятность того, что при выборе совокупности из $k$ элементов. Она состоит из $k_1$ элементов первого вида и $k_2$ элементов второго вида?}
 
$$ k = k_1 + k_2, k_1 \leq n_1, k_2 \leq n_2 $$

$$ p(A) = \frac{C_{n_1}^{k_1}*C_{n_2}^{k_2}}{C_{n_1n_2}^{k_1+k_2}}$$

....

\section{Формула полной вероятности}
Вероятность события $A$, которое может наступить при условии появления одного из несовмесных событий $B_1, B_2, ..., B_n$, образующих полную группу и называемых гипотезами,
равна сумме произведений вероятности каждого из этих событий на соответствующее условие вероятности события $A$. Т.е

$$ p(A) = p(B_1)p(A \setminus B_1) + p(B_1)p(A \setminus B_1) + ... + p(B_n)p(A \setminus B_n)$$

... 

\section{Формула Баеса}
Пусть событие $A$ может наступить при условии появления одного из несовмесных событий $B_1, B_2, ..., B_n$ образующих полную группу, тогда условная вероятность любого события $B_i, i=\vec{1,n}$,
при условии что событие $A$ уже произошло вычисляется по формуле Баеса:

$$ p(B_i \setminus A) = \frac{p(B_i)p(A \setminus B_i)}
                    {p(B_1)p(A \setminus B_1) + p(B_2)p(A \setminus B_2) + ... +  p(B_n)p(A \setminus B_n)}$$


\section{Схема Бернулли}

Испытания называются независимыми относительно события $A$, если при нескольких испытаниях вероятность события $A$ независит от исходов других испытаний.

Испытания проводятся по схеме Бернулли, если для них выполняются следующие условия: 

\begin{enumerate}
    \item испытания являются независимыми
    \item количество испытаний известно заранее
    \item результате испытания может произойти только два исхода: успех или неуспех. 
    \item вероятность успеха в каждом испытании одна и та же
\end{enumerate}

Формула Бернулли - вероятность того, что при n испытаниях, успех осуществляется ровно $k$ раз, и следовательно неуспех ровно $n - k$ раз, вычисляется по формуле:

$$ p_n(k) = C_n^k*p^k*q^{n-k}$$

Где $p$ - вероятность успепха, $q = 1 - p$

\section{Локальная и интегральная теорема Лапласа}

TODO!!!

\section{Формула Пуассона}

TODO!!!


МАТЕМАТИЧЕСКАЯ статистика

\section{Случайная величина}

TODO!!!

\section{Дискретная случайная величина ДСВ}
Предположим что производится некоторое испытание, ДСВ называется переменная величина $X$, которая в результате проведения данного испытания может принять одно из нескольких значений $x_1, x_2, ... x_n$, причем заранее неизвестно какое именно.

\textbf{Законом распределения} дискретной случайной величины $X$ называется таблица в верхней строке которой перечислены порядки возрастания все возможные значения, которые может принять $X$, а в нижней соответствующие им вероятности.

\begin{table}[ht]
\centering
\begin{tabular}{|c|c|c|c|c|}
\hline
$X$ & $x_1$ & $x_2$ & $...$ & $x_n$ \\
\hline
$P$ & $p_1$ & $p_2$ & $...$ & $p_n$ \\
\hline
\end{tabular}
\end{table}

Очевидно что для любого закона распределения сумма вероятностей должна быть равна единице.

\begin{table}[ht]
\centering
\begin{tabular}{|c|c|c|c|c|}
\hline
$X$ & $-1$ & $0$ & $2$ & $3$ \\
\hline
$P$ & $0.2$ & $?$ & $0.3$ & $0.1$ \\
\hline
\end{tabular}
\end{table}

p = 0.4
Важнейшими числовыми характеристиками случайной величины являются математическое ожидание,
дисперсия и среднее квадратическое отклонение $\sigma(x)$

Математическое ожидание - среднее значение
$$ M(X) = \Sigma_{n=1}^n x_i*p_i$$
$$ M(X^2) = \Sigma_{n=1}^n x_i^2*p_i$$
Дисперсия 
$$ D(X) = M(x - M(x))^2 = \Sigma^n_{n=1}(x_i - m)^2*p_i$$
$$ D(X) = M(x^2) - M^2(x) $$

$$ \sigma(x) = \sqrt{D(x)} $$

\subsection{Пример}


\begin{table}[ht]
\centering
\begin{tabular}{|c|c|c|c|}
\hline
$y$ & $-1$ & $0$ & $3$ \\
\hline
$P$ & $0.1$ & $0.2$ & $? = 0.7$ \\
\hline
\end{tabular}
\end{table}

$$ M(y) = (-1) * 0.1 + 0 * 0.2 + 3 * 0.7 = -0.1 + 2.1 = 2$$
$$ D(y) = M(y^2) - M^2(y) = -2^2$$
$$ M(y^2) = \Sigma^3_{i=1} = (-1)^2 * 0.1 + 0^2*0.0 + 3^2*0.7 = 6.4$$
$$ \sigma(y) = \sqrt{6.4 - 4} = 2.4$$

\subsection{Свойства математического ожидания}

Математическое ожидание от константы равно константе.

$ M(c) = c, c = const $

Постоянный множитель можно выностить за знак математического ожидания. 

$ M(cX) = c * M(X)$

Математическое ожидание от алгебраической суммы конечного числа независимых случайных величин
равно алгебраической сумме математических ожиданий этих случайных величин.

$$ M(X + Y) = M(X) + M(Y) + 7$$
        
$$ M(XY) = M(X)*M(Y)$$
    
$$Z = 3x + 2y, M(X) = 5, M(Y) = 2, M(Z) = ?$$

$$ M(Z) = M(3X) + M(2Y) + M(7)= 3M(X) + 2M(Y) + M(7) = 3*5 + 2*2 + 7 = 26$$

\subsection{Свойства дисперсии}

\begin{enumerate}
    \item дисперсия от числа равна нулю $ D(c) = 0, c = const$
    \item постоянный множитель выностится за знак дисперсии во второй степени $D(cX) = c^2 * D(X)$
    \item дисперсия от конечного числа независимых случайных величин равна сумме дисперсий
    $D(X + Y) = D(X) + D(Y)$
\end{enumerate}

$$ D(X) = 5$$
$$ D(Y) = 2$$
$$ D(Z) = D(3x) + D(2y) + D(7) = 9D(x) + 4*D(y) + D(7) + 0 = 45 + 8$$ 

Предположим что для двух случайных величин известны законы их распределения

\begin{table}[H]
\centering
\begin{tabular}{|c|c|c|c|c|}
\hline
$x$ & $x_1$ & $x_2$ & $...$ & $x_n$ \\
\hline
$p$ & $p_1$ & $p_2$ & $...$ & $p_n$ \\
\hline
\end{tabular}
\end{table}

\begin{table}[H]
\centering
\begin{tabular}{|c|c|c|c|c|}
\hline
$y$ & $y_1$ & $y_2$ & $...$ & $y_m$ \\
\hline
$p$ & $q_1$ & $q_2$ & $...$ & $q_m$ \\
\hline
\end{tabular}
\end{table}

Найти распределение для $Z$

\begin{table}[H]
\centering
\begin{tabular}{|c|c|c|c|c|}
\hline
$z$ & $x_1 + y_1$ & $x_2 + y_2$ & $...$ & $x_n + y_m$ \\
\hline
$p$ & $p_1 * q_1$ & $p_2 * q_2$ & $...$ & $p_n * q_m$ \\
\hline
\end{tabular}
\end{table}
    

Если в верхней строке числа совпадают, то эти столбцы таблицы нужно объединить сложив соответсвующие вероятности.

Задан закон о распределении двух независимых случайных величин

\begin{table}[H]
\centering
\begin{tabular}{|c|c|c|c|}
\hline
$x$ & $0$ & $1$ & $2$ \\
\hline
$P$ & $0.2$ & $0.5$ & $? = 0.3$ \\
\hline
\end{tabular}
\end{table}
    
\begin{table}[H]
\centering
\begin{tabular}{|c|c|c|c|}
\hline
$y$ & $1$ & $2$ & $3$ \\
\hline
$P$ & $0.3$ & $0.6$ & $ 0.1 $ \\
\hline
\end{tabular}
\end{table}

Найти закон распределения случайной величины $z = x + y$
Найти математическое ожидание, дисперсию и среднее квадратическое отклонение двумя способами:
используя определение и свойства мат ожидания и дисперсии.

Составляем закон распределения для $z$

\begin{table}[H]
\centering
\begin{tabular}{|c|c|c|c|c|c|c|c|c|c|}
\hline
$z$ & $0 + 1$ & $0 + 2$ & $0 + 3$ & $1 + 1$ & $1 + 2$ & $1 + 3$ & $2 + 1$ & $2 + 2$ & $2 + 3$ \\
\hline
$P$ & $0.2 * 0.3$ & $0.2 * 0.6$ & $0.2 * 0.1$ & $0.5 * 0.3$ & $0.5 * 0.6$ & $0.5 * 0.1$ & $0.3 * 0.3$ & $0.3 * 0.6$ & $0.3 * 0.1$ \\
\hline
\end{tabular}
\end{table}

\begin{table}[H]
\centering
\begin{tabular}{|c|c|c|c|c|c|c|c|c|c|}
\hline
$z$ & $1$ & $2$ & $3$ & $2$ & $3$ & $4$ & $3$ & $4$ & $5$ \\
\hline
$P$ & $0.06$ & $0.12$ & $0.02$ & $0.15$ & $0.3$ & $0.05$ & $0.09$ & $0.18$ & $0.03$ \\
\hline
\end{tabular}
\end{table}

\begin{table}[H]
\centering
\begin{tabular}{|c|c|c|c|c|c|}
\hline
$z$ & $1$ & $2$ & $3$ & $4$ & $5$ \\
\hline
$P$ & $0.06$ & $0.27$ & $0.41$ & $0.23$ & $0.03$\\
\hline
\end{tabular}
\end{table}

$$ M(z) = \Sigma_{n=1}^5 z_i*p_i = 0.06 * 1 + 0.27 * 2 + 0.41 * 3 + 0.23 * 4 + 0.03 * 5 = 2.9$$
$$ M(z) = M(x) + M(y) = 1.1 + 1.8 = 2.9 $$
$$ M(x) = \Sigma_{n=1}^3 = 0.2 * 0 + 0.5 * 1 + 2 * 0.3 = 1.1 $$
$$ M(x^2) = \Sigma_{n=1}^3 = 0.2 * 0^2 + 0.5 * 1^2 + 2^2 * 0.3 = 1.7 $$ %%%%%%%%%% REDO THIS
$$ M(y) = \Sigma_{n=1}^3 = 1 * 0.3 + 2 * 0.6 + 3 * 0.1 = 1.8 $$
$$ M(y^2) = \Sigma_{n=1}^3 = 1^2 * 0.3 + 2^2 * 0.6 + 3 ^ 2 * 0.1 = 3.6 $$

$$ D(z) = M(z^2) - M^2(z) =  9.26 - 2.9 = 6.35 $$
$$ M(z^2) = 0.06 * 1^2 + 0.27 * 2^2 + 0.41 * 3^2 + 0.23 * 4^2 + 0.03 * 5^2 = 9.26 $$
$$ D(z) = D(x) + D(y) = 2.4 $$
$$ D(x) = M(x^2) - M^2(x) = 1.7 - 1.1 = 0.6$$
$$ D(y) = M(y^2) - M^2(y) = 3.6- 1.8 = 1.8$$


\section{Задание}
Заданы законы распределения двух независимых случайных величин

\subsection{Распределение x}
\begin{table}[H]
\centering
\begin{tabular}{|c|c|c|c|}
\hline
$x$ & $0$ & $1$ & $2$\\
\hline
$P$ & $0.4$ & $0.2$ & $0.4$ \\
\hline
\end{tabular}
\end{table}

\subsection{Распределение y}
\begin{table}[H]
\centering
\begin{tabular}{|c|c|c|}
\hline
$y$ & $1$ & $2$\\
\hline
$P$ & $0.3$ & $0.7$ \\
\hline
\end{tabular}
\end{table}
  
Требуется составить закон распределения дискретной случайной величины $Z = 2X + 3Y$ и найти математическое ожидание, дисперсию и среднее квадратическое отклонение 
двумя способами по определению и используя свойства мат ожидания и дисперсии     

$$ 2*0 + 3*1 = 3, p = 0.12$$
$$ 2*0 + 3*2 = 6, p = 0.28$$
$$ 2*1 + 3*1 = 5, p = 0.06$$
$$ 2*1 + 3*2 = 8, p = 0.14$$
$$ 2*2 + 3*1 = 7, p = 0.12$$
$$ 2*2 + 3*2 = 10, p = 0.28$$

\subsection{Распределение Z}
\begin{table}[H]
\centering
\begin{tabular}{|c|c|c|c|c|c|c|}
\hline
$z$ & $3$ & $5$ & $6$  & $7$  & $8$  & $10$ \\
\hline
$P$ & $0.12$ & $0.06$ & $0.28$ & $0.12$ & $0.14$ & $0.28$\\
\hline
\end{tabular}
\end{table}

\subsection{По определению}
$$M(Z) = 3 * 0.12 + 5 * 0.06 + 6 * 0.28 + 7 * 0.12 + 8 * 0.14 + 10 * 0.28 = 7.1 $$
$$M(Z^2) =  3*3*0.12 + 5*5*0.06 + 6*6*0.28 + 7*7*0.12 + 8*8*0.14 + 10*10*0.28 = 55.5 $$
$$D(Z) = 55.5 - 7.1 * 7.1 = 5.09 $$ 

\subsection{Используя свойства}
$$ M(X) = 0 * 0.4 + 1 * 0.2 + 2 * 0.4 = 1 $$
$$ M(Y) = 1 * 0.3 + 2 * 0.7 = 1.7 $$
$$ M(Z) = M(2X) + M(3Y) = 2 * 1 + 3 * 1.7 = 7.1 $$

$$ M(X^2) = 0*0*0.4 + 1*1*0.2 + 2*2*0.4 = 1.8 $$  
$$ M(Y^2) = 1*1 * 0.3 + 2*2*0.7 = 3.1 $$
$$ D(X) = 1.8 - 1*1 = 0.8 $$
$$ D(Y) = 3.1 - 1.7*1.7 = 0.21 $$
$$ D(Z) = 2*2*0.8 + 3*3*0.21 = 5.09 $$


\section{Непрерывные случайные величины}

$X$ - произвольная случайная величина, функцией распределения этой величины называется функция, которая обозначается $F(x) = P(X < x)$ - это вероятность того, что случайная величина.
Случайная величина называется непрерывной, если ее функция распределения $F$ является непрерывной и кусочно-дифференцируемой.

$f(x) = F'(X)$ Называется плотностью распределения случайной величины $X$

Перечислим основные свойства функции распределения и плотности распределения $F(x)$ и $f(x)$
\begin{enumerate}
    \item $ F(x) = P(X < x) $
    \item $F(x)$ - непрерывная и монотонно неубывающая функция
    \item $ 0 <= F(X) <= 1 $
    \item $ \lim_{x\rightarrow-\infty}F(X) = 0$
    \item $ \lim_{x\rightarrow+\infty}F(X) = 1$
\end{enumerate}

\begin{enumerate}
    \item $f(x) >= 0$ 
    \item $F(x) = \int_{-\infty}^{+\infty}f(x)dx $
    \item $\int_{-\infty}^{+\infty}f(x)dx = 1$
    \item $P(a < x < b) = F(b) - F(a) $
    или $\int_a^b f(x)dx$
\end{enumerate}

$$ M(x) = \int_{-\infty}^{+\infty}xf(x)dx $$

$f(x) = A *x$; $x$ принадлежит $(-1, 2)$
$A$ - ? 
$M(x)$ - ?

Решение 

$$ \int_{-\infty}^{+\infty} f(x) dx = 1 $$
$$ \int_{-\infty}^{+\infty} A x dx = 1 $$

$$ \frac{A(x)^2}{2}|_{-1}{2} = 1$$
$$ \frac{A}{2}*(4 - 1) = 1 $$
$$ A = \frac{2}{3}$$

$$ M(X) = \int_{-\infty}^{+\infty} x f(x)dx = \int_{-\infty}^{+\infty} (2-1)\frac{x^2}{3}dx = 
\frac{2}{3}*\frac{x^3}{3}*|_{-1}^{2} =\frac{2}{9} * (8 + 1) = 2 $$

$$ D(X) = M (X - M(X))^2 = \int_{-\infty}^{+\infty} (x - m)^2*f(x)dx $$
Но 
$ D(X) = M (X^2) - M^2(x)$, где
$$ M(X) = \int_{-\infty}^{+\infty} xf(x)dx $$
$$ M(X^2) = \int_{-\infty}^{+\infty} x^2 f(x) dx$$

$ f(x) = Ae^(-x)$, $x$ принадлежит $(0, +\infty)$
$$ A - ?  $$
$$ D(x) - ? $$

Решение
$$ \int_{0}^{+\infty} f(x) dx = 1 => \int_{0}^{+\infty} Ae^xdx = A(-e^(-x)) |_{0}^{+\infty} = A(-0, + 1), A = 1 $$

\begin{dmath}
    M(x) = \int_{0}^{+\infty}xf(x)dx = \int_{0}^{+\infty}xe^(-x)dx = \int_{0}^{+\infty}xde^(-x) = -(xe^(-x) |_{0}^{+\infty} -\int_{0}^{+\infty}e^(-x)dx) = -e^{-x}|_{0}^{+\infty} = -0 + 1 = 1
\end{dmath}

\begin{dmath}
    M(x^2) = \int_0^{+\infty}x^2f(x)dx = \int_0^{+\infty}x^2e^{-x}dx=
    -\int_0^{+\infty}x^2de^{-x} =-(x^2*e^{-x}dx^2)=\int_0^{+\infty}2xe^{-x}dx=-2\int_0^{+\infty}xde^{-x}=
    -2(xe^(-x) |_0^{+\infty} - \int_0^{+\infty}e^xdx)
    = 2\int_0^{+\infty}e^xdx = -2e^(-x) |_0^{+\infty} = -2*(0-1) = 2
\end{dmath}

$$ D(X) = M(X^2) - M^2(X) = 2 - 1^2 = 1 $$

$$\sigma(X) = \sqrt{D(x)}$$

Заданая функция распределения $F(X)$ непрерывной случайной велчины X

% рисунок
\begin{enumerate}
    \item Определить недостающие параметры этого распределения
    \item найти плотность распределения f(x) и схематично построить ее график
    \item найти мат ожидание дисперсию, среднее квадратическое отклонение
\end{enumerate}

$$ k(4 + 3x - x^2) = 0. x = -1, 4, A(-1, 0), B(1, 1) $$

$$ k(4+3*1-1^2) = 1 => k = \frac{1}{6} $$

$$F(x) = \begin{cases}
    0, x < -1 \\
    1/6(4 + 3x - x^2), -1 <= x <= 1 \\
    1, x > 1 \\
\end{cases}$$

$$f(x) = \begin{cases}
    0, x < 1 \\
    1/6 (3 - 2x), -1 <= x <= 1 \\
    0, x > 1 \\
\end{cases}$$

\begin{dmath}
    M(x) = \int_{-\infty}{+\infty} x*f(x) dx = \int_{-\infty}^{-1} x 0 dx +
    \int_{-1}^{+1}\frac{1}{6}(3-2x)dx+\int_{1}^{+\infty}x^2*0*dx = \int_{-1}^{1} x * 1 / 6 (3-2x)dx = -\frac{2}{9}
\end{dmath}

$$ D(X) = M (X^2) - M^2(x)  = 1/3 - 4/81 = 23 / 81 $$

\begin{dmath}
     M(X^2) = \int_{-\infty}{+\infty}x^2f(x)dx = \int_{-\infty}^{-1} x^2 0 dx + \int_{-1}^{+1} x^2 * 1/6 (3-2x)dx + \int_{1}^{+\infty}x^2 0*dx = 1/3
\end{dmath}

$$ \sigma = \sqrt{D(x)} = \sqrt{\frac{23}{81}} = \frac{\sqrt{23}}{81} $$ 

Задана плотность распределения f(x) непрерывной случайной величины X. 

Определить недостающие параметры этого распределения
Найти функция распределения $F(x)$
и схематично построить ее график.
Найти только мат. ожидание.
Найти вероятность попадания $X$ в интервал от -1 до 1

В вершине
$ k(4x - x^2) = 0 $
$ x_B = 2 $
$$ \int_{-\infty}^{+\infty} f(x)dx  = 1 $$

\begin{dmath}
\int_0^2 k(4x - x^2) dx = k (\frac{4x^2}{2} |_0^2 - x^3/ 3 |_0^2 ) = k (8 - \frac{8}{3}) = \frac{3}{16}
\end{dmath}

f(x) = 
a) $0, x < 0 $
b) $\frac{3}{16}(4x - x^2), 0 <= x <= 2$
d) $1, x > 2$

$ F(x) = \int_{-\infty}{\infty} f(x)dx = \int{-\infty}{0}0dx + \int3/16(4x - x^2)dx = 3/16 (4*x^2/2 - x^3/3) $

a) $0, x < 0$
b) $3/8 * x^2 - x^3/16, 0 <= x <= 2$
c) $1, x > 2$

\begin{dmath}
 M(x) = \int_{-\infty}^{+\infty} f(x) dx = \int _{-\infty}^{0}x*0dx + \int_0^2 x * \frac{3}{8} (4x - x^2)dx + \int_{2}^{+\infty}x*0*dx = \int_0^2 x * \frac{3}{8} (4x - x^2)dx = \frac{5}{4} 
\end{dmath}

$$ P(-1 < x < 1) = F(1) - F(-1) = \frac{3}{8} - \frac{1}{16} - 0 = \frac{5}{16} $$

\section{Основные определения и понятия математической статистики}
В математической статистике исследуется способы получения выводов на основе имперических данных.
Рассмотрим некоторую случайную величну $X$ дискретную или непрерывную, точное распределения екоторой нам не известно

Предположим, что было произведено $n$ испытаний, в результате которых случайная величина $X$ приняла следующие значения.

$$X: {x_1, x_2, ..., x_n}$$ 

Совокупность полученных чисел называется выборкой объема $n$ из случайной величины $X$

Графическим способом представление выборки является империческая функция распределения: $F*(x) = \frac{n(x)}{n}$
где $n$ - объем выборки, а $n(x)$ - число элементов выборки, которые не превосходят $x$.

Функция $F*(x)$  является ступенчатой монотонно неубывающей функцией, ступеньки которой расположены в точках $x_i$,
а высота ступеньки равна $\frac{k_i}{n}$ где $k_i$ - число элементов выборки, равных $x_i$.

Если случайная величина $X$ непрерывна, а объем выборки $n$ достаточно большой, то империческая функция распределения близка к $F*(x) \rightarrow F(x)$, точный вид которой нам не известен.

Если $X$ - дискретная случайная величина, причем спектр ее возможных значений сравнительно небольшой, то выборку можно записать в виде статистического ряда.

\begin{table}[H]
\centering
\begin{tabular}{|c|c|c|c|c|c|}
\hline
$x_i$ & $x_1$ & $x_2$ & $x_3$ & $...$ & $x_k$ \\
\hline
$n_i$ & $n_1$ & $n_2$ & $n_3$ & $...$ & $n_k$ \\
\hline
\end{tabular}
\end{table}

$x_i < x_2 < ... < x_n$
$n_i$ - соответсвующая частота, то есть число элементов данной выборки, равных $x_i$.
$$ n = n_1 + n_2 + n_3 + ... + n_k $$
    
В данном случае графическим представлением выбокри является полигон частот, то есть ломанная линия вершины которой имеют координаты $(x_i, n_i)$

Выборка представлена в виде статистического ряда. 

\begin{table}[H]
\centering
\begin{tabular}{|c|c|c|c|c|c|c|c|c|}
\hline
$x_i$ & $18.5$ & $19.5$ & $20.5$  & $21.5$ & $22.5$ & $23.5$ & $24.5$ & $25.5$ \\
\hline
$n_i$ & $3$ & $11$ & $24$ & $28$ & $19$ & $5$ & $3$ & $4$\\
\hline
\end{tabular}
\end{table}

Построить полигон частот.
    
Если $X$ - непрерывная случайная величина или дискретная, но с широким спектром возможных значений, то более удобной является запись этой выборки в виде интревального вариационного ряда, для этого весь диапазон возможных значений $X$ разбивается на несколько интревалов одинаковой длинны $h$

Для каждого такого интервала $(a_i-1, a_i)$
обозначим через $n_i$ число элементов данной выборки принадлежащих этому интревалу, в результате получим таблицу, которая называется интревальным вариационным рядом.


\begin{table}[H]
    \centering
    \begin{tabular}{|c|c|c|c|}
\hline
$(a_0, a_1)$ & $(a_1, a_2)$ & $...$ & $(a_{n-1}, a_{n})$\\
\hline
$n_1$ & $n_2$ & $...$ & $n_k$ \\
\hline
\end{tabular}
\end{table}

$$n = n_1 + n_2 + ... + n_k$$
$$a_i = a_{i-1} + h$$

Графическим представлением выборки, которая задана в виде интервального вариационного ряда, является гистограмма, представляющая собой объединение прямоугольников, у каждого из которых основанием служит интревал $(a_i-1, a_i)$
а высота это соответсвующая частота $n_i$
    
При большом объеме выборки и удачном выборе шага $h$. Гистограмма будет близка к теоретической плотности распределения $f(x)$
Пример второй:
Выборка предствалена в виде интервального вариационного ряда.
\begin{table}[H]
    \centering
    \begin{tabular}{|c|c|c|c|c|c|c|c|c|}
        \hline
        $x_i$ & $(18, 19)$ & $(19, 20)$ & $(20, 21)$ & $(21, 22)$ & $(22, 23)$ & $(23, 24)$ & $(24, 25)$ & $(25, 26)$\\
        \hline
        $n_i$ & $3$ & $11$ & $24$ & $28$ & $19$ & $5$ & $3$ & $4$\\
        \hline
    \end{tabular}
\end{table}

\section{Статистические оценки}

Рассмотрим выборку из случайной величины $X$
Пусть $\theta$ некоторый параметр распределения случайной величины x. Либо математическое ожидание $a$, либо диспесия $\sigma^2$. Так как точное распределение случайной величины нам неизвестно, то и точное значение этого параметра нам нельзя. Однако по данной выборке, можно оценить параметр $\theta$ приближенно. Полученные значения $\theta(x_1, x_2, ..., x_n)$
Называется точечной оценкой параметра. Если оцениваемым параметром является математическое ожидание $a$, то в качестве оценки берут среднее арифметическое всех элементов выборки 
$ \vec{n} \frac{1}{n}(x1 + x2 + ... + x_n) = \frac{1}{n}\Sigma x_i$

Это выражение называется выборочной средней. Можно показать что наилучшей оценкой дисперсии будет выражение: 
$ S^2 = \frac{1}{n-1}((x_1-\vec{x})^2+ (x_1-\vec{x})^2 + ... + (x_n-\vec{x})^2)$
Называется исправленная выборочная диспресия. 

Фиксируем некоторое число гамма, близкое к единице
$\gamma \rightarrow 0$
Которое называется доверительной вероятностью. 
Интревал $(\theta_1, \theta_2)$ границы которого являются функциями от элемнтов $y$ - называется доверительным интревалом, для оценки неизвестного параметра $\theta$, соответсвующим доверительной вероятности гамма. Если с вероятностью гамма истинное значение параметра $\theta$ принадлежит этому интервалу. 

Если выборка взята из нормально распределенной случайной величины, то доверительный интервал для оценки математического ожидания имеет вид $\vec{x}-t_{\gamma}\frac{s}{\sqrt{n}}<a<vec{x}+t_{\gamma}\frac{s}{\sqrt{n}}$  

$n$ - объем выборки.
$\vec{n}$ - выборочное среднее. 
$S = \sqrt{S^2}$
$t = t_{\gamma} = t(\gamma, n)$

Рассмотрим задачу когда выборка представлена в виде интервального ряда
\begin{table}[H]
    \centering
    \begin{tabular}{|c|c|c|c|c|c|c|}
        \hline
        $x_i$ & $(a_0, a_1)$ & $(a_1, a_2)$ & $(a_2, a_3)$ & $(a_3, a_4)$ & $(a_4, a_5)$ & $(a_5, a_6)$\\
        \hline
        $n_i$ & $n_1$ & $n_2$ & $n_3$ & $n_4$ & $n_5$ & $n_6$ \\
        \hline
    \end{tabular}
\end{table}

$$ n = n_1 + n_2 + n_3 + n_4 + n_5 + n_6 $$
$$ h = a_1 - a_{i-1}$$

Схема решения

\begin{enumerate}
    \item Найти середины интревалов $x_1, x_2, x_3, x_4, x_5, x_6, x_i = (a_{i+1} - a_i) / 2$
    \item Через c мы полагаем середину третьего интревала, и перейти к новым переменным $z_i = \frac{x_i - c}{h}$
    \item Записать данный вариационный ряд, перейдя к новым переменным
    \item Найти выборочную дисперсию и выборочное среднее по формулам
    \item $\vec{x} = \vec{z} * h + c, S^2 = n/(n-1) S_z^2*h^2$
    \item По таблицам распределения стьюдента найти значение $t_{\gamma} = t(\gamma, n)$
    \item $(\vec{x} - t_{\gamma} \frac{x}{\sqrt{n}}, \vec{x} - t_{\gamma} \frac{x}{\sqrt{n}})$
\end{enumerate}

\begin{table}[H]
    \centering
    \begin{tabular}{|c|c|c|c|c|c|c|}
        \hline
        $z_i$ & $-2$ & $-1$ & $0$ & $1$ & $2$ & $3$\\
        \hline
        $n_i$ & $n_1$ & $n_2$ & $n_3$ & $n_4$ & $n_5$ & $n_6$ \\
        \hline
    \end{tabular}
\end{table}

$$ z_6 = \frac{a_6 + a_5}{2} = \frac{a_6 - a_5}{2} - \frac{a_3 - a_2}{2} = \frac{6}{2} = 3 $$

$$ \vec{z} = \frac{1}{n}\Sigma_{i=1}^nz_i*n_i = \frac{1}{n}(-2 * n_1 + (-1)*n_2 + 0 * n_3 + 1 * n_4  + 2 * n_5  + 3 * n_6) $$
$$ \vec{z}^2 = \frac{1}{n}\Sigma_{i=1}^nz_i^2*n_i = \frac{1}{n}(-2^2 * n_1 + -1^2*n_2 + 0^2 * n_3 + 1^2 * n_4  + 2^2 * n_5  + 3^2 * n_6)$$
$$ S_z^2 = \vec{Z^2} - (\vec{S})^2$$

Выборочная совокупность представоена в виде интервального вариационного ряда.

\begin{table}[H]
    \centering
    \begin{tabular}{|c|c|c|c|c|c|c|}
        \hline
        $(9, 11)$ & $(11, 13)$ & $(13, 15)$ & $(15, 17)$ & $(17, 19)$ & $(19, 20)$ \\
        \hline
        $6$ & $8$ & $14$ & $10$ & $8$ & $4$\\
        \hline
    \end{tabular}
\end{table}

Предполагая нормальное распределения найти доверительный интервал для оценки математического ожидания, доверительную вероятность принять $\gamma = 0.95$

$$ n = 6 + 8 + 15 + 10 + 8 + 4 = 50$$
$$ h = 11-9 = 2, c = 14$$

\begin{table}[H]
    \centering
    \begin{tabular}{|c|c|c|c|c|c|}
        \hline
        $(a_i, a_i)$   & $n_i$ & $x_i$ & $z_i$ & $z_in$ & $z_{i^2}n$ \\
        \hline
        $(a_9, a_11)$  & $6$ & $10$ & $-2$ & $-12$ & $24$ \\
        $(a_11, a_13)$ & $8$ & $12$ & $-1$ & $-8$ & $8$ \\
        $(a_13, a_15)$ & $14$ & $14$ & $0$ & $0$ & $0$ \\
        $(a_15, a_17)$ & $10$ & $16$ & $1$ & $10$ & $10$ \\
        $(a_17, a_19)$ & $8$ & $18$ & $2$ & $16$ & $32$ \\
        $(a_19, a_20)$ & $4$ & $20$ & $3$ & $12$ & $36$ \\
        $\Sigma$ & $ 0 $ & $0$ & $0$ & $18$ & $110$ \\
        \hline
    \end{tabular}
\end{table}

$$ \vec{z} = \frac{1}{50} * 18 = 0.36$$
$$ \vec{z^2} = \frac{1}{50} * 110 = 2,2$$
$$ S_{z}^2 = 2.2 - (0.36)^2 = 2,0704 $$
$$ S_{z} = \sqrt{S_{z}^2} = \sqrt{2,0704} $$
$$ \vec{x} = 14.72$$

$$ S^2 = \frac{50}{50 - 1}*2,0704$$
$$ S = \sqrt{\frac{50}{50 - 1}*2,0704} = 2.907$$
$$ t_{\gamma} = t(0,95, 50) = 2,0009$$
$$ t_{\gamma} \frac{S}{\sqrt{n}} = 2,0009 * \frac{2.907}{\sqrt{50}} = 0.8226$$
$$ S_z = \sqrt{S_z^2} = 1.4389 $$
$$ \vec{x} - t_{\gamma}\frac{S}{\sqrt{n}}, \vec{x} + t_{\gamma}\frac{S}{\sqrt{n}}$$
$$ (14.72 - 0.8226, 14 + 0.8226)$$
$$ (13.8976, 15.5426)$$

\section{Статистическая проверка гипотез}

Предположим что в результате n испытаний получена выборка $x_1, x_2, ... x_n$ из случайной величины X. Точное распределение этой случайной величины нам не известно, однако мы можем сделать некоторые предположения о виде этого распределения или значении его параметров. 
\begin{enumerate}
    \item Случайная величина x имеет нормальное распределение.
    \item Математическое ожидание $a$ равно заданному числу $a_0$.
    \item Дисперсия $\sigma^2$ меньше заданного числа $\sigma_0^2$
\end{enumerate}

Такие предположения принято называть гипотезами. Задача статистической проверки гипотезы состоит в следующем.
Задаются некоторым малым числом $\alpha$, которое называется уровнем значимости. Тогда по заданной выборке, пользуясь некоторым критерием требуется ответить на вопрос согласуется ли эта выборка с гипотезой $H_0$

При этом $\alpha$ это фактическая вероятность принятия ошибочного решения.


Говорят что дискретная случайная величина $X$ распределена по закону Пуассона, если это распределение имеет вид

e^{-\lambda}
x: 0 1 2 ... k
p: e^{-\lambda} 
! TODO
k * \lambda k! e^{-lambda}

\lambda - это число большее нуля называется параметром распределения Пуассона, причем математиеское ожидание совпадает с \lambda;

Предположим что имеется выборка из случайной величины x, которая представления в виде статистического ряда.

\begin{table}[H]
    \centering
    \begin{tabular}{|c|c|c|c|c|c|}
        \hline
        $x$ & $0$ & $1$ & $2$ & $...$ & $k$ \\
        \hline
        $n$ & $n_0$ & $n_1$ & $n_2$ & $...$ & $n_k$
        \hline
    \end{tabular}
\end{table}

Для проверки гипотезы что данная случайная величина имеет распределение Пуассона, как и для проверки других аналогичных гипотез применяется метод, который носит название ``Критерий согласия'' $$\xi^2$$ Пирсона
$$ k = 5 $$

Задать уровень значимости альфа либо \alpha = 0.01, либо \alpha = 0.05 
Сформулировать две гипотезы

    Случайная величина X из которой взята выборка имеет распределение Пуассона(основная гипотеза)
    Противоположная(конкурирующая гипотеза)

По данному статистическому ряду находим выборочное среднее.

$\vec{x} = 1 / n \Sigma i=0 k x_i n_i = $ 

В качестве \lambda полагаем \vec{x}

p_i = k * \lambda k! e^{-lambda}
p_1 = k * \lambda k! e^{-lambda}
p_2 = \frac{\lambda}{2} * p_1
p_3 = \frac{\lambda}{3} * p_1
p_3 = \frac{\lambda}{4} * p_1
p_3 = \frac{\lambda}{5} * p_1

По найденным вероятностям p_i рассчитать теоретические частоты распределения Пуассона по формулам.
n_i' = n * p_i
\delta n = n_i - p_i для всех i.

Найдем велечину \delta n_i^2
Найдем велечину \delta n_i^2 / n_i'


\begin{table}[H]
    \centering
    \begin{tabular}{|c|c|c|c|c|c|}
        \hline
        $x$   & $n_i$ & $p_i$ & $n_i'$ & $\delta n_i$ & $\delta n_i^2$ & $\delta n_i^2 / n_i' $ \\
        \hline
        $\Sigma$ & $ 0 $ & $0$ & $0$ & $18$ & $110$ & $\Sigma!!!$\\
        \hline
    \end{tabular}
\end{table}

Сложив числа последнего столбца вычисляем значение \xi^2 наблюдаемое 
По числу степеней свободы m = S - 2 = 4 (S - количество строк) и заданному уровню значимости \alpha по таблицам находим число \xi^2_{\alpha, m}
Сравниваем \xi^2_наблюдаемое \xi^2_{\alpha, m} необходимо сделать один из двух выводов
Если \xi^2 наблюдаемой меньше \xi^2_{\alpha, m}, то это означает что на уровне значимости \alpha данная выборка согласуется с гипотезой о распределении случайной величины по закону Пуассона. Если же больше то данная выборка противоречит гипотезе по распределению пуассона

\end{document}
