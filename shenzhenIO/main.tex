\documentclass[a4paper]{article}
% Русский язык
\usepackage[utf8]{inputenc}
\usepackage[russian]{babel}
% Математические символы и формулы
\usepackage{mathtools}
% Разбиение на разные файлы
\usepackage{subfiles}
% Отступы с краев(margin)
\usepackage[margin=1in]{geometry}
% Форматирование заголовков
\usepackage{titlesec}
\usepackage{indentfirst} % Красная строка
        
\begin{document}    
\section{Лекция 2}
\subsection{Способы записи функций алгебры логики}
\subsubsection{Таблица истиности}
\subsubsection{Описание ФАЛ в виде алгебраического выражения.}
По таблице истиности можно составить алгебраическое(булево) выражение. При этом запись алгебраического выражения осуществляется с использованием дизъюнктивной нормальной формы(ДНФ)
или конъюнктивной нормальной формы(КНФ)
Для представления логической функции $$F$$ в виде ДНФ необходимо составить сумму(дизъюнкцию)
произведений(конъюнкций) значений логической функции $$F_i$$ и минтернов $$m_i$$ т.е.
$$ F = \sigma_{i = 1}^n F_i * m_i $$. $$F$$ - число строк в таблице истиности
Минтерм это логическое произведение всех переменных, причем переменные равные нулю записываются с инверсией.
% \begin{tabular}[c|c|c|c]
    $$x_3$$ $$x_2$$ $$x_1$$ $$F$$
    0 0 0 0
    0 0 1 0
    0 1 0 1
    0 1 1 0
    1 0 0 1
    1 0 1 1
    1 1 0 1
    1 1 1 1
% \end{tabular}

$$ m_1 = !x_3!x_2!x_1 $$ 
$$ m_2 = !x_3!x_2x_1  $$
$$ m_3 = !x_3x_2!x_1  $$
$$ m_4 = !x_3x_2x_1  $$
$$ m_5 = x_3!x_2!x_1  $$
$$ m_6 = x_3!x_2x_1  $$
$$ m_7 = x_3x_2!x_1  $$
$$ m_8 = x_3x_2x_1  $$
$$ F = m_1*0 + m_2*0 + m_3*1 + m_4*0 + m_5*1 + m_6*1 + m_7*1 + m_8*1$$
$$ F = ... $$


Для записи функции в дизъюнктивной нормальной форме можно использовать следующее правило
следует записать столько дизъюнктивных членов, представляющих собой произведение всех переменных, сколько раз функция принимает значение единицы. Причем переменные равные нулю записываются с инверсией.


Для представления логической функции $$F$$ в виде конъюктивной нормальной формы необходимо составить произведение или конъюнкцию сумм или дизъюнкций и макстермов $$k_i$$. Причем число произведений равно числу строк в таблице истиности.
КНР


$$ F = mult(1, n)(F_I + k_i) $$


Макстерм - логическая сумма всех переменных, причем переменные равные единице записываются с инверсией. 


$$ k_1 = x_3 + x_2 + x_1 $$
$$ k_2 = x_3 + x_2 + !x_1 $$
$$ k_3 = x_3 + !x_2 + x_1 $$
$$ k_4 = x_3 + !x_2 + !x_1 $$
$$ k_5 = !x_3 + x_2 + x_1 $$
$$ k_6 = !x_3 + x_2 + !x_1 $$
$$ k_7 = !x_3 + !x_2 + x_1 $$
$$ k_8 = !x_3 + !x_2 + !x_1 $$


$$ F = (k_1 + 0) * (k_2 + 0) * (k_3 + 1) * (k_4 + 0) * (k_5 + 1) * (k_6 + 1) * (k_7 + 1) * (k_8 + 1)$$


Для записи функции в КНФ использую следующее правило. Следует записать столько конъюнктивных членов
представляющих собой дизъюнкции всех переменных, сколько раз функция принимает значение нуля. Причем переменные равные единице записываются с инверсией.


Должна быть реализована схема с тремя входами, которая на выходе выдает единицу тогда, когда четное число входных сигналов равно единице.
Или все сигналы равны нулю.


$$ x_2 x_1 x_0 F $$
0 0 0 1 
0 0 1 0
0 1 0 0
0 1 1 1
1 0 0 0
1 0 1 1
1 1 0 1
1 1 1 0 


ДНФ
0 3 5 6
$$ F = (!x2!x1!x0) + (!x2!x1!x0) + (x2!x1x0) + (x2x1!x0)$$

$$ x_2 $$
$$ x_1 $$
$$ x_0 $$

% Схема 1 


КНФ
1 2 4 8
$$ F = (x2+x1+!x0)*(x2+!x1+x0)*(!x2+x1+x0)*(!x2+!x1+!x0)$$
\end{document}

% Схема 2

\subsubsection{Описание ФАЛ в виде последовательности десятичных чисел}
Иногда для сокращения записи ФАЛ представляют в виде последовательно записаных десятичных эквивалентов двоичных кодов соответствующих конституант единице и нуля.
ДНФ
$$ F(x_2, x_1, x_0) = \sigma(0, 3, 5, 6) $$
КНФ
$$ F(x_2, x_1, x_0) = mult(1, 2, 4, 7) $$

\section{Комбинационные устройства}
Комбинационные устройства не имеют внутренней памяти.
Уровни их выходных сигналов всегда однозначно определяются текущими уровнями входных сигналов
и никак не связаны с их предыдущими значениями.
Любое изменение входных сигналов обязательно изменяет состояние выходных.
Все комбинационные микросхемы построены из набора простейших логических элементов
\section{Мультиплексор}