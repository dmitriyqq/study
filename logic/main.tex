\section{Функции алгебры логики}
С точки зрения алгебры высказываний логические операции полностью характеризыются таблицами истинности,
при этом можно забыть о том, что мы рассматриваем какие-то операции на высказываниях и иметь дело с самими таблицами истинности.
Таким образом, мы приходим к понятию функций алгебры логики.
Однако не следует забывать о логических связках так как они проясняют целый ряд соотношений в алгебре логики.
В частности в обозначениях некоторых функций алгебры логики совпадают с обозначениями логических связок.
Будем употреблять вместо ``истины'' и ``ложь'' - ``единицу'' и ``ноль'' соотвественно.

Определение:
Функцией алгебры логики и функцией буля $f(x_1, x_2, ..., x_n)$ называется функция принимающая значения $1$ и $0$ и аргументы которой принимают значения 1 или 0. Такая функция имеет тип:

${0, 1}^n -> {0, 1}$

Ясно что тождественно истинные представляют собой постоянные функции, а две равносильные формулы выражаю одну и ту же функцию. Будем говорить что функция f(x_1, ...,  x_i-1, 0, x_i+1, ..., x_n) + f(x_1, ..., x_i-1, 1, x_i+1, ..., x_n)Существенно зависит от переменной x_i. переменные от которых функция существенно зависит называют существенными переменными. Остальные фиктивынми.

Будем отождествялть функции, из которых добавлением фиктивных переменных можно получить одну и ту же функцию.
Фуникция может быть задана таблицей своих значений. Другой способ записи логической функции это формула. Таким образом, любую логическую функцию, записанную таблицей можно представить в виде формулы и наоборот. Пусть f, g функции алгебры логики и x_1, x_2, ..., x_n - совокупность аргументов, входящих по краней мере в одну из этих функций. Добавляя необходимые фиктивные аргументы можно считать что x_1, x_2, ..., x_n входят в каждую из функций f, g. Будем говорить что функции f, g равны и писать f = g, если при всех значениях x_1, x_2, ...., x_n значения f и g совпадают.

Введем специальные обозначения для основных функций алгебры логики

x принадлежит {1, 2}

0(x) = 0
1(x) = 1
i(x) = x
Г x = 0, x = 1
Г x = 1, x = 0

x & y = 1, x = y = 1
x & y = 0, или x, или y = 0

x v y = 0, x = y = 0
x v y = 1, если x = 1  или y = 1

x + y = 0, x == y
x + y = 1, x != y

x->y = !xvy
x <-> y = !(x+y)
A ~ B <=> f(A) = f(B)

Каждой формуле А алгебры высказываний можно сопоставить функцию f(A) алгебры логики, так что A эквивалентно B тогда и только тогда когда f(A) = f(B)

В самом деле, пусть x_1, x_2, ..., x_n - все высказывательные переменные, которые используются в записи формул A и B.

f(X_i) = x_i, если P и Q - какие либо формулы то полагаем,
f(!P) = !f(P)
f(P&Q) = f(P)f(Q)
f(PvQ) = f(P)vf(Q)
f(И) = 1, f(Л) - 0
f(x->y) = f(x) -> f(y)

Можно проверить, что построенные таким образом функции алгебры логики, f(A) и f(B) по эквивалентным формулам A и B равны.

A = !(X_1 & X_2)
B = ! (X_1) & ! (x_2)

f(A) =
f(B) =


Докажем что f(A) == f(B), для это составим таблицы значений этих функций используя определение основных функций алгебры логики, через которые выражены f(A) и f(B)


x_1 | x_2 | f(A) | f(B)


Мы видим что таблицы функций f(A) и f(B) совпадают, то есть f(A) == f(B).

Выясним каково число функций n переменных.
Очевидно каждую функцию алгебры логики можно задать с помощью таблиц истинности, которая будет содержать 2^n строк, следовательно каждая функция n переменных принимает 2^n значений состоящих из нулей и единиц, таким образом, функция полностью определяетс янабором значений из нулей и единиц длинны 2^n. Общее число наборов состоящих из нулей и единиц длинны 2^n равно 2^2^n значит число различных функций от n переменных равно 2^2^n в частности различных функций одной переменной.

Выпишем все фунции алгебры логики одной переменной.

из этой таблицы следует что две функции одной переменной будут постоянными (f1, f4).

x f1(x) f2(x) f3(x) f4(x)
1 1 1 0 0
0 1 0 1 0

x y f_1 f_2 f_3 f_4 f_5 f_6 f_7 f_8 f_9 f_10 f_11 f_12 f_13 f_14 f_15 f_16
1 1 1 1 1 1 1 1 1 1 0 0 0 0 0 0 0 0
1 0 1 1 1 1 0 0 0 0 1 1 1 1 0 0 0 0
0 1 1 1 0 0 1 1 0 0 1 1 0 0 1 1 0 0
0 0 1 0 1 0 1 0 1 0 1 0 1 0 1 0 1 0

f_1 = 1
f_16 = 0

\section{Представление произвольной функции алгебры логики в виде формулы алгебры логики}
Пусть f(x1, x_2, ..., x_n) - произвольная функция алгебры логики от n переменных. Рассмотрим формулу:

$$ (F(1, 1, ..., 1)&x_1&x_2&...&x_n)V $$
$$ (F(1, 1, ..., 0)&x_1&x_2&...&!x_n)V $$
$$ ................................... $$
$$ (F(0, 0, ..., 0)&!x_1&!x_2&...&!x_n)V $$

Которая составлена следующим образом - каждое слагаемое этой логической суммы пердстваляет конъюнкцию, в которой первый член язвляется значением функции $F$, при некоторых определенных значениях переменных x_1, x_2, ..., x_n. Остальные же члены конъюнкции представляют собой переменные и их отрицания. При этом под знаком отрицания находятся те и только те и только те переменные, которые в первом члене конъюнкции имеют значение ноль. Вместе с тем формула 1 содержит в виде логических слагаемых всевозможные конъюнкции указанного вида, ясно что формула 1 полностью определяет функцию $F(x_1, x_2, ..., x_n)$. Иначе говоря, значение функции $F$ и формулы 1 совпадают на всех наборах значений переменных x1, x_2, ..., x_n.
Например, если x_1 == 0, x_2, ..., x_n = 1, то функция F принимает значение F(0, 1, ..., 1) при этом логическое слагаемое входящее в формулу принимает такое же значение F, все остальные логические слагаемые формулы 1 имеют значение 0.

Действительно в них знаки отрицания над переменными распределяются иначе чем в рассмотренном слагаемом, но тогда при замене переменных теми же значениями в конъюнкцию войде символ 0, без знака отрицания и символ 1 под знаком отрицания. В таком случае один из членов конъюнкции имеет значение ноль. А потому вся конъюнкция имеет значение 0.

В связи с этим на основании равносильности x v 0 = x значением функции 1 является F(0, 1, ..., 1)
Ясно что вид формулы 1 может быть значительно упрощен, если в ней отбросить те логические слагаемые, в которой первый член конъюнкции имеет значение 0. И следовательно вся конъюнкция имеет значение 0. Если же в логическом слагаемом первый член конъюнкции имеет значение единица, то пользуясь равносильностью 1 & x = x, этот член конъюнкции можно не выписывать.
Таким образом получается формула один, которая сожержит только элементарные переменные высказывания и обладает следующими свойствами:

Каждое логическое слагаемое формулы содержит все переменные, входящие в функцию F(x_1, x_2, ..., x_n)
Все логические слагаемые различны.
Ни одно логическое слагаемое формулы не содержит одновременно переменную и ее отрицание
Ни одно логическое слагаемое формулы не содержит одну и ту же переменную дважды

Перечисленные свойства будем называть свойствами совершенства. Из приведенныз рассуждений видно, что каждой нетождественно ложной функции соотвествует единственная формула указанного вида.

Если функция F(x_1, x_2, ..., x_n) задана таблицей истинности, то соответсвующая ей формула может быть получена проще.
Для каждого набора значений переменных, на котором функция $F$ принимает значение единицу, запишем конъюнкцию элементарных переменных высказываения. Возьмем значение x_n, если значение x_n на указанном наборе есть единица и !x_n, если !x_n = 1
Дизъюнкция записанных конъюнкиций и будет формулой.

x_1 x_2 x_3 F(x_1, x_2, x_3)
1 1 1 0
1 1 0 1 x_1 & x_2 & !x_3 V
1 0 1 1 x_1 & !x_2 & x_3 V
1 0 0 0
0 1 1 0
0 1 0 1 !x_1 & x_2 & !x_3 V
0 0 1 0
0 0 0 1 !x_1 & !x_2 & !x_3 V

Искомая формула обладает свойствами совершенства
Основная операция, которую можно производить над функциями алгебры логики, называется суперпозицией.

Интуитивный смысл этого понятия состоит в том что аргументы функции посдставляются в другие функции и эта процедура может повторятся. Можно сказать что формула это суперпозиция основных функций !x, x|y, x&y, x->y, x~y;

Введенное ранее понятие эквивалентности формул очевидно согласуется с определением равенства функций.

Основные эквивалентности формул перечисленные в пункте 5 можно записать как равенство функций алгебры логики.

!! x = x
xy = yx
x v y = y v x
x(yz) = (xy)z
xv(yvz) = xy

Используя алгебраическую терминологию можно сказать что равенство 2 и 3, 4 и 5 выражают коммутативность и ассоциативность конъюнкции и дизъюнкции. Равенства 6 и 7 выражают дистрибутивность конъюнкции относительно дизъюнкции и соотвественно дистрибутивность дизъюнкции относительно конъюнкции. Эти свойства позволяют преобразовывать выражения по правилам умножения многочлена на многочлен учитывая 7.

Пункт 8. нормальные формы.
Ранее был изложен способ выяснения является ли данная функция ТИ формулой, ТЛ формулой или является выполнимой с помощью таблиц истинности.

Если переменных много то этот процесс долгий и скучный.
Существует другой способ решения этого вопроса. Основанный на приведении к так называемой ``нормальной'' форме.
Пусть задана система высказывательных элементов. X_1, X_2, ..., X_N

Элементарной дизъюнкцией, конъюнкцией системы X_1, X_2, ..., X_N называется дизънкция некоторых переменных этой системы или их отрицаний.

Элементарная дизъюнкиция называется правильной, если в нее каждая переменная входит не более одного раза, включая и ее вхождение под знаком отрицания. Если в ЭД(ЭК) входит каждая высказывательная переменная системы, с отрицанием или без него.
При том только один раз, то она называется полной элементарной дизъюнкцией.

x_1 v x_2 v !x_3
F(x_1, x_2, x_3)

% МЕТОДИЧКА 2590

x_1 & x_2 & x_3,
x_1 & x_2 & !x_3,
x_1 & !x_2 & x_3,
x_1 & !x_2 & !x_3,
!x_1 & x_2 & x_3,
!x_1 & x_2 & !x_3,
!x_1 & !x_2 & x_3,
!x_1 & !x_2 & !x_3,

Если в полных элементарных конъюнкциях заминить знаки на дизънкцию, то получим все полные элементарные дизънкции высказывательных переменных x_1, x_2, x_3.

Ясно что для n - высказывательных перменных существует 2^n неэквивалентных полных элементарных конъюнкций.
И такое же число неэквивалентных элементарных дизъюнкций.

Рассмотрим какую-нибудь полную элементарную дизънкцию $\sigma$ она может принять значение ноль только для одного набора значений высказывательных переменных.

x_1, x_2, ..., x_n. А именно, когда каждая x_i входящее в $\sigma$ без отрицания имеет значение 0, а с отрицанием значение 1.

Систему значений высказывательных переменных, для которых данная полная элементарная дизъюнкция принимает значение 0 назовем нулем полной элементарной дизъюнкции. Двойственным образом, полная элементарная конъюнкция может принимать значение 1 только для одного набора высказывательных переменных x_1, x_2, ..., x_n. Когда любое x_i входящее в $\sigma$ без отрицания имеет значение 1, а с отрицанием значение 0. Такую систему значений высказывательных перменных назовем единицей полной элементарной конъюнкции.

Единицами полной элементарной конъюнкции 1, 1, 1 ...............

Теорема 1.

Чтобы элементарная дизъюнкция была тождественно истинной, необходимо и достаточно чтобы в ней содержалось вместе с некоторой высказывательной переменной x_i и ее отрицание !x_i.

Теорема 2.

Чтобы элементарная конъюнкция была тождественно ложной необходимо и достаточно чтобы в ней содержалась хотя бы одна пара множителей из которых один является отрицанием другого.


Формула A называется КНФ системы высказывательных переменных x_1, x_2, ..., x_n, если A  является конъюнкцией элементарных дизъюнкций высказывательных переменных этой системы.

Формула A называется ДНФ системы высказывательных переменных x_1, x_2, ..., x_n, если A  является дизъюнкцией элементарных конъюнкиций высказывательных переменных этой системы.

Как уже говорилось ранее, все логическии операции можно свести к трем дизъюнкцию, конъюнкцию и отрицание.
Предположим, что формулы для которых мы будем определеять нормальную форму содержат только эти операции.
Знак отрицания можно предполагать отнесенным только к элементарным высказываниям.

Как мы видели выше, с формулой составленных из переменных и их отрицаний при помощи операции конъюнкции и дизъюнкции можно производить такие же преобразования, как и с алгебраическими выражениями. Можно следовательно раскрыть все скобки и представить всякую такую формулу в виде сумме элементарных произведений. Таким образом доказано, что для любой формулы существует ДНФ.

X&(!Y & Z)&(UvV) = X & (Yv!Z) & (UvV) = (XYvX!Z)(UvV)  = XYUv XYV v X!ZU v X!ZV

Для каждой формулы существует эквивалентная ей конъюнктивно нормальная форма.

X(X->Y) = X(!XvY) = X!X v XY = XY

XvY <-> XY

Заметим что для каждой формулы A существует не одна КНФ и ДНФ.
Произведя разными способами дистрибутивные операции, можно придти к различным нормальным формам.

Пункт 9.
Совершенная ДНФ. Совершенная КНФ.

Среди КНФ и ДНФ выделяется класс формул однозначно определяемых формулой A.

Формула А наызвается совершенной КНФ от высказывательных переменных x_1, x_2, ..., x_n
если она является конъюнкцией различных полных элементарных дизъюнкций от высказывательных перменных этой системы.
При этом порядок членов в элементарных дизъюнкциях не учитывается.

Например формула (x_1 | x_2 | !x_3) & (!x_1 | x_2 | x_3) & (x_1 | x_2 | x_3)
является СКНФ x_1, x_2, ..., x_n
так как КНФ является ТИ формулой тогла и только тогда, когда каждая входящая в нее элементарная дизъюнкция является ТИ формулой.
А последняя является ТИ формулой тогда и только тогда, когда она вместе с некоторой высказывательной переменной x_i содержит ее отрицание !x_i, то СКНФ не может быть ТИ формулой.

Определение СДНФ двойственно СКНФ. Формула Б называется СДНФ от высказывательных переменных x_1, x_2, ..., x_n
если она является дизъюнкцией различных полных элементарных конъюнкций от высказывательных перменных этой системы.
Другими словами Б - СДНФ, если Б равняется A*, для некоторой СКНФ формулы A.

Напомним что СДНФ A(x_1, x_2, ..., x_n) обладает следующими свойствами, в ней нет одинаковых слагаемых, ни одно слагаемое не содержит двух одинаковых множителей. Никакое слагаемое не содержит переменную вместе с ее отрицанием. В каждом слагаемом содержится в качестве множителя либо x_i, либо !x_i.

Условия 1-4 являеются необходимыми и достаточными для того, чтобы ДНФ была СДНФ.

Вместе с тем эти условия дают возможность высказать правила позволяющие выводить или приводить любую не ТЛ формулу к ДНФ.
Опишем эти правила.

Пусть дана произвольная формула A(x_1, x_2, ..., x_n) приведем к ДНФ форме B, затем если какое-нибудь слагаемое формулы B не содержит x_i, то заменим его на B&(x_i v !x_i) = B&x_i v B&!x_i.

Таким образом, мы можем изменить нашу ДНФ так, чтобы условие 4 было выполнено. Если в полученном выражении получаться одинаковые слагаемые, то удалив все, кроме одного из них, мы получим опять равносильное выражение. Если после этого в некотором слагаемом окажется по нескольку одинаковых множителей. То лишние множители можно удалить. Наконец, можно удалить те слагаемые которые содержат переменную вместе с ее отрицанием, так как слагаемые представляют собой тождественно ложное выражение. После этого получим СДНФ.

Заметим что нам нет необходимости знать заранее является ли формула ТЛ формулой или нет. Проделывая указанные операции мы это выясним после того, как удалим все слагаемые содержащие переменную вместе с ее отрицанием. Если формула А - ТЛ формула, то все слагаемые будут удалены и мы не получим СДНФ.

Можно доказать что каждая нетождественно истинная формула имеет единственную с точностью до порядка расположения множителей и слагаемых СКНФ. Правило приведения произвольной формулы к СКНФ аналогичны тем, которые мы описывали для нахождения СДНФ и выражаются в двойственных терминах.

$$A = X v Y (X v !Y) = X v XY v Y!Y = X = X & (Y|!Y) = XY | X!Y$$

Приведем формулу А к СКНФ, используя понятие двойственности.

$$A* = X&(Y(Xv!Y)) = X&YvX&X&!Y = X&Y v X&!Y$$

$$B* = X&Y v X&!Y$$

$$B = (XvY) & (Xv!Y)$$

Привести к СКНФ
$$(XvY)&(!Yv!Z)v(Xv!Y)&(YvZ)$$

Опишем Алгоритм получения для формулы A, заданной таблицей истинности эквивалентные ей СКНФ. Возьмем в таблице нули формулы А. По нулям построим полные элементарные дизъюнкции, которые на этих наборах принимают значение ноль, беря x_i, если x_i равняется нулю, если x_i равно единице и образуем их конъюнкцию, таким образом мы получим СКНФ. Эквивалентную формуле А.

x_1 x_2 x_3 A B
1 1 1 0 0
1 1 0 1 1
1 0 1 0 1
1 0 0 1 1
0 1 1 0 0
0 1 0 0 1
0 0 1 1 1
0 0 0 1 0

СКНФ
$$A_1 = (!x_1 v !x_2 v !x_3 ) & (!x_1 v x_2 v !x_3) & (x_1 v !x_2 & !x_3) & (x_1 & !x_2 & x_3)$$
СДНФ
$$A_2 = x_1 & x_2 & !x_3 & x_1 & !x_2 & !x_3 v !x_1 & !x_2 & x_3 v !x_1 & !x_2 & !x_3$$

Совершенно нормальные формы позволяют дать критерий равносильности двух произвольных формул А и Б. В самом деле каковы бы ни были формулы А и Б можно предполагать что они содержат одни и те же переменные. Иначе можно добавить
$B & (x_iv!x_i)$,
$B v (x_i&!x_i)$.

Таким образом, любые две формулы можно заменить равносильными им формулами содержащими одинаковые переменные. После этого эти формулы надо привести к СДНФ или СКНФ. Если А и Б равносильные формулы, то в силу единственности совершенно нормальных форм как ДНФ так и КНФ этих формул должны полностью совпадать. Таким образом сравнение совершенно нормальных форм формул А и Б решает вопрос об их равносильности. В заключении этого раздела отметим некоторые соотношения равносильности полезные для упрощения формул.

$$XvXY = X$$
$$X(XvY) = X$$
$$Xv!XY = XvY$$
$$!XvXY = !XvY$$
$$X(!XvY) = XYU$$
$$!X(XvY) = !XY$$

Эти соотношения можно сформулировать в виде правил следующим образом, если слагаемое некоторой суммы входит множителем в другое слагаемое, то второе слагаемое можно из суммы удалить.

Если множитель некоторого произведения входит слагаемым в другой множитель, то второй множитель можно удалить.

В каждом слагаемом можно удалить множитель, который равносилен отрицанию другого слагаемого.

В каждом множителе можно удалить слагаемое, которое равносильно отрицанию другого слагаемого.

1) $A = X v XY v YZ v !XX = X v !XZ v YZ = XvZvYZ = X v Z$
2) $B = (X v Y) (!X!YvZ)v!Zv(!XvY)(UvV)$

\subsection{Некоторые приложения теории алгебры логики.}

Релейно-контактные схемы.
Среди технических схем автоматизации значительное место занимают устройства релейно контактного действия.
Они широко используются в технике автоматического управления, в электронно-вычислительной технике и т.д.
Эти устройтсва их в общем случае называют переключательными схемами содержат сотни реле, полупроводников и электромагнитных элементов. Описание и контруирование таких схем
указал на возможность применения аппарата алгебры логики при исследовании релейно контактных схем. Использование алгебры логики в конструировании РКС оказалось возможным в связи с тем что каждой схеме можно поставить в соотвествие некоторую формулу алгебры логики и каждая формула алгебры логики реализуется с помощью некоторой схемы.

Это обстоятельство позволяет выявить возможности заданной схемы изучая соответсвующую формулу а упрощение схемы свести к упращению формулы. С другой стороны, до построения схемы можно описать с помощью формулы те функции, которые схема должна выполнять.

Рассмотрим как устанавливается связь между формулами алгебры логики и релейно-контактными схемами

Под переключательной схемой понимается некоторое изображение некоторого устройства состоящего из следующих элементов:

1) Переключателей, которыми могут быть механически действующие устройства(переключатели, переключающие ключи, кнопочные устройства и так далее)
2) Соединяющих их проводников
3) Входов в систему и выходов из нее(клем на которые подается электрическое напряжение)
4) Они называются полюсами схемы
5) Сопротивление, конденсаторы и так далее на схемах не изображаются. Переключательной схемой принимается в расчет только два состояния каждого переключателя, которые называются ``замкнутым'' и ``разомкнутым''.

Рассмотрим простейшую схему содержащую один переключатель P и один вход A и один выход B.

Переключателю P поставим в соответствие высказывание p:
переключатель P замкнут. 

Если p истинно, то схема проводит ток. Если p ложно, то переключатель разомкнут и схема тока не проводит.
Если принять во внимание не смысл высказывания а только его значение, то можно считать что любому высказыванию может быть поставлено в соотвествие переключательная схема 1.

Конъюнкции двух высказываний p и q будет представлена двух полюсной схемой с последовательным соединением двух переключателей

Эта схема пропускает ток тогда и только тогда когда истинно p и q одновременно

Дизъюнкция изобразиться схемой: рис 3

Если высказывание !p есть отричацние высказывания p, то ТИ формула изобразиться схемой

Нарисовать схему для
XY v !X!Y

Дана схема, для нее формулу алгебры логики и упростить

XYZ v !XYZ v X!YZ v XY!Z 

= yz(xv!x) v xz(!yvy) v xy(!zvz)
= yz v xz v xy = z(yvx) v xy

Из примера 2 следует что для некоторых РКС путем равносильных преобразований формулы алгебры логики можно получить РКС содержащую меньшее число переключателей, проблема решения этой задачи носит название проблемы минимизации.

Приведем проблемы минимизации.
Приведем пример построения РКС по заданным условиям с оценкой числа контактов.
Построить контактную схему для оценки результатов некоторого спортивного соревнования тремя судьями при следующих условиях
Судья засчитывающий результат нажимает имеющуюся в его распоряжении кнопку.
В случае если кнопки нажали не менее двух судей должна загореться лампочка, положительное решение судей принято простым большинством голосов. Ясно что работа нужной ркс описывается функцией буля трех переменных F(x, y, z), где переменное высказывание x, y, z означают - X - судья X голосовал за, .......


Таблица истинности функции F(x, y, z) имеет вид

X Y Z F
0 0 0 0
0 0 1 0
0 1 0 0
0 1 1 1
1 0 0 0
1 0 1 1
1 1 0 1
1 1 1 1

xyz v !xyz v x!yz v xy!z

Пункт второй
Решение логических задач методами алгебры логики.
Суть применения методов алгебры логики к решению логических задач состоит в том что имея конкретные условия логической задачистараются записать их в виде формулы алгебры логики. Дальнейшим путем равносильных преобразований упрощают полученную формулу. Простейший вид формулы как правило приводит к ответу на все вопросы.

Покажем на ряде конкретных примеров как использовать возможности алгебры логики для решения элементарных логических задач.

Пример четвертый
Пытаясь вспомнить победителей прошлогоднего турнира, пять бывших зрителей заявили:
Антон был вторым а Борис был пятым
Виктор был вторым а Денис третьим
Григорий был первым а Борис третьим
Антон был третьим а Евгений шестым
Виктор был третьим а Евгений Четвертым.

Впоследствие выяснилось что каждый зритель ошибся в одном из своих высказываний.
Каково было истинное распределение мест в турнире.
Решение.

Будем обозначать высказывания x_y, где X - первая буква имени участника, y - место которое он занял в турнире.
Так как в паре высказываний каждого зрителя одно истинно, а другое ложно, то будут истинны дизъюнкции этих высказываний.

A2 v Б5 = 1
В2 v Д3 = 1
Г1 v Б3 = 1
А3 v Е6 = 1
В3 v E4 = 1

(A2 v Б5) & (В2 v Д3) & (Г1 v Б3) & (А3 v E6) & (В3 v E4) = 1

Пример 5
Жили четыре мальчика. Альберт, Карл, Дидрих и Фридрих.
Фамилии друзей те же, что и имена только так что ни у кого из них имя и фамилия не были одинаковы.
Кроме того,
Фамилия Дидриха не была Альберт
Требуется определить фамилию каждого из мальчиков если известно, что имя мальчика, у котого фамилия фридрих, есть фамилия того мальчика, имя которого фамилия Карла.
Поставим в соотвествие каждому мальчику символ X с индексом Y, где X - имя, а Y - фамилия мальчика.
Тогда по условию ложны высказывания: Aa, Кк Фф, Дд, Фд

Есть мальчик такой, что истинна конъюнкция

Пример 6
По подозрению в совершении преступления задержали:
Брауна, Джона и Смита. Один из них был уважаемым в городе стариком.
Другой был малоизвестным чиновником, третий известным мошенником.
В процессе следствия старик говорил правду, мошенник лгал, а третий задержанный в одном случае говорил правду, а в другом ложь. Вот что они утверждали:

Б: Я совершил это, Джон не виноват
Д: Браун не виноват, преступление совершил Смит
С: Я не виноват, Виноват Браун.

Кто из них виноват, если преступник 1.
Решение

Обозначим буквами Б Д С, тогда утверждение высказанные задержанными модно записать в виде конъюнкции:
Б Д С | Б & !D | B & !C | C & !B | L
0 0 0 0 0 0 0
0 0 1 0 1 0 1
0 1 0 1 0 0 1
0 1 1 1 1 0 1
1 0 0 0 0 1 1
1 0 1 0 0 0 0
1 1 0 0 0 1 1
1 1 1 0 0 0 0

Формула L истинна в 5 из 8 занумерованных случая, формулу 4 сразу исключаем так как в ней истинны сразу две конъюнкции, а это противоречит условию задачи.

В случаях 2, 3, 5 оказываются истинны по два высказывания B и Д, Б и C, Д и С, что противоречит условию задачи.
Следовательно справедлив случай 7, то есть преступник Смит, он известный мошенник.

Отсюда ясно, что Джон уважаемый в городе старик, а Браун чиновник

Пункт 11 логическое следствие
Задача логики - дать принципы рассуждения наша цель получить критерий для решения механическим путем. Вопроса о том, можно ли некоторую цепь рассуждений основываясь на ее форме считать логичной, цель рассуждений представляет собой просто конечную последовательность высказываний приводимых в обоснование утверждения, что последнее высказывание в этой последовательности(заключение) может быть выведено из некоторых начальных высказываний. В обыденных условиях посылки вывода считаются истинными(на основании имещегося опыта, эксперимента или убеждения)

Если признать посылки вывода истинными, а принципы использованные в цепи рассуждений основанных на этих посылках правильными
то мы вынужденны рассматривать полученное заключение как истинное. В математической теории дело обстоит иначе. Там мы интересуемся исключительно заключениями(так называемыми теоремами теории), которые можно вывести из принятой начальной системы высказываний(так называемых аксиом теории) В соотвествии с правилами установленными в какой-либо логической системе. В частности в самой теории понятие истинности не играет никакой роли.

Вклад исчисления высказывания в теорию вывода заключается в следующем: оно дает критерий вместе с практическими формами его применения для решения того когда к заключительному предложению рассуждения следует приписать истинное значение 1, если каждой посылке приписывается значение 1. Этот критерий имеет форму определения.

Определение
Высказывание B есть логическое следствие A_1, А_2, ..., A_n |= B Если высказывание B истинно всякий раз, когда истинны высказывания A_1, A_2, ..., A_n.

Будем писать |=A Для обозначения того что формула A общезначима или тавтология.


Теорема 1
A |= B, тогда и только тогда когда формула B общезначима и тавтология

A |= B => |=A->B

По таблице для A->B имеем
A->B = 0, когда A = 1, B = 0.
На основании принятого допущения, такая комбинация истинностных значений не встречается. Следовательно A->B равно 1 тавтологий.

Пусть А->B общезначима, рассмотрим такое распределения истинностных значений, принимаемых простыми компонентами, что А получает значение 1.
Поскольку A->B равно 1, то из таблицы истинности для импликации следует, что B получает значение 1, то есть A вытекает(|=) B,что и требовалось доказать.


Теорема 2

Пусть A_1, A_2, ..., A_n |= B, тогда следующее утверждения равносильны:

1) A_1, A_2, ..., A_n |= B
2) A_1 & A_2 & ..., & A_n |= B
3) A_1 & A_2 & ..., & A_n -> B

Пусть B - логическое следствие формул A_1, A_2, ..., A_n.

Пусть A_1 & A_2 & ... & A_n -> B = 1, тогда A_1, A_2, ..., A_n = 1

Из 2 следует 3, смотри доказательство теоремы 1. Пусть A_1 & A_2 & ... & A_n -> B = 0 
Такое невозможно так как B - логическое следствие A_1 & A_2 & ... & A_n 

Из 3 следует 1.
Нас интересует A_1 & A_2 & ... & A_n = 1, то есть по определению логического следствия A_1, A_2, ..., A_n следует B.

