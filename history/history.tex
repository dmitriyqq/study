\documentclass[a4paper]{article}
% Русский язык
\usepackage[utf8]{inputenc}
\usepackage[russian]{babel}
% Математические символы и формулы
\usepackage{mathtools}
% Разбиение на разные файлы
\usepackage{subfiles}
% Отступы с краев(margin)
\usepackage[margin=1in]{geometry}
% Форматирование заголовков
\usepackage{titlesec}
\usepackage{indentfirst} % Красная строка
\usepackage{tabulary}
\begin{document}

\section{Введение}

1 семинар - 1 тк (Лекции)
Д/з 1!, 3 вопросы (Проанализировать
политику первых князей по признакам гоударства)
\subsection{Литература}

\begin{enumerate}
\item Методичка 2208
"Отечественная история", Иваново 2014
\item Бумажные из библиотеки
    \begin{enumerate}
    \item Орлов и Георгиев "История России для тех. вузов"
    (Плюсы: вся фактура в одном месте, Минусы: Марксистский формационный подход)
    \item Личман "История России для тех. вузов"
    (Плюсы: есть концепции, Минусы: слабая фактура)
    \item "История России с др. времен до 19 в.", Кафедральный учебник
    (Говно, но нужно брать)
    \item "История России 1917 - 1945", Кафедральный учебник
    (Более приличная часть)
    \end{enumerate}
\item Найти(совет) "История России в схемах и таблцах"
\item Электронные учебники
    \begin{enumerate}
    \item "История России с др. времен до 19 в.", эл. ресурс
    (Посмотреть структуру)
    \item "История России 1917 - 1945", эл. ресурс
    \end{enumerate}
\end{enumerate}

\subsection{1 ТК}

Автор(Персона)
Названиние Концепии
Суть Концепции(Основной вывод по проблеме)
Базовые термины
Даты (к концепциям, принципиально важные)

\section{Основы методологии (как проводить историческое исследование)
исторической науки}

\subsection{Методика(этапы) исторического исследования}

\begin{enumerate}
    \item Анализ исторический источников
    \item Изучение научных парадигм, концепций, разных точек зрения на эту проблему
    \item Анализ первого и второго, если надо ищите дополнительные исторические факты
    \item Формулируете собственную аргументированную точку зрения
\end{enumerate}

\subsection{Исторические источники}
Исторический источник(созданный в ту эпоху, которую вы изучаете) - это любой памятник прошлого, который содержит исторический факт или информацию об историческом событии.
Группы исторический источников

\begin{enumerate}
    \item материальные - археологические раскопки
    \item письменные - повесть временных лет
    \item изобразительные
    \item лингвистические
    \item кино, фото, фоно документы (XIX-XX в.)
\end{enumerate}

\subsection{Научные парадигмы}
Парадигма - концептуальная схема, модель, изучение исторических источников и трактовки исторических событий.
Выделяют пять базовых научных парадигм

\subsubsection{План}
\begin{enumerate}
    \item Авторы
    \item базовые факторы исторического развития
    \item термины
    \item модель исторического развития
    \item историки исслеователи, представители этой парадигмы
    \item +
    \item -
\end{enumerate}

\subsubsection{Религиозная(идеалистическая)}
\begin{enumerate}
    \item Фома Аквинский V в.н.э.
    провиденциализм - все события происходят по божеьей воле.
    \item линейная и конечная
    \item
    \begin{enumerate}
        \item В.Н. Татищев - 18 в. (Основатель российской исторической науки),
        \item Н.М. Карамзин
        \item С.Н.Соловьев
        \item В.О. Ключевский
    \end{enumerate}
    \item Так как выразителем божественной воли является правитель, то историческое исследование будет выглядеть как история правления отдельных князей, царей и императоров.
    \item - Недоказуема, так как основная на вере. Субъективна и однобока, так как изучает только один фактор - политику правителя
    \item + Богатейший фактический материал по политической истории
\end{enumerate}

\subsubsection{Формационная}
\begin{enumerate}
    \item Альт. названия:
    ОЭФ - общественно-экономическая формация, \\
    Марксистская, \\
    Материалистическая, \\
    "Советская", \\
    \item
        \begin{enumerate}
        \item Фридрих Энгельс 40е годы 19 в.
        \item К. Маркс
        \end{enumerate}
    \item Социально-экномические социальные факторы,
    \item Классы, классовая борьба(очень сильное сужение предмета исследования) \\
        ОЭФ - ступень этап в развитии общества, который отличается от других этапов своим базисом(экономическим строем), который определяет особенности надстройки
    \item 5 базовых экономических формаций. Схема Ступеньки:
    \begin{enumerate}
        \item Первобытная
        \item Рабовладельческая
        \item Феодальная. Базовый фактор: земля. Ручной труд. Натуральное хозяйство. Базовые классы: феодалы, крепостные крестьяне. \\
        Феодализм в России 1649 г. Начинается с соборного уложения Алексея Михайловича.
        \item Паровой двигатель => Капитализм. Преход от одной ОЭФ к другой осуществляется в результате социальной революции(изменение базовых классов общества). \\ Изменение в средствах производства, новые технологии. Факторы производства: труд и капитал, в результате этого происходит переход к машинному индустриальному производству. Классы: буржуазия, пролетариата.
    \end{enumerate}


    Комментарии к схеме:
    По Марксу преход к комунизму возможен при двух условиях. \\
    Два пути перехода: Кризис капитализма, или абсолютное большинство класса пролетариат.((Социализм)). Комунизм. Отсутсвие классов, всеобщее равенство. Формирование общественной собственности на средства производства. Нет небходимости в существовании государства. Базовый принцип: от каждого по возможности каждому по потребности.
    \begin{enumerate}
    \item Если капитализм себя изживет, прийдет к кризису, появятся новые технологии
    \item Если пролетариат составит 70 процентов населения страны, тогда он имеет право на революцию и установление диктатуры пролетариата.
    \end{enumerate}

    В. И. Ленин дополнил это учение учением о социализме, как осбой переходной стадии от капитализма к комунизму, цель которой увеличение численности пролетариата, завершение индустриализации. Т.к. в октябре 1917 года пролетариата в России 9\%, и следовательно объективных предпосылок для социалистической революции в россии не было.

    Основатели советской исторической науки
    Б.А. Рыбаков \\
    "Экономика определяет все" К. Маркс \\
    В СССР формационная концепция стала доминирующей и единственно возможной и на этой модели основывалась вся советская историческая наука
    \item
    Идельно подходит для анализа стран Западной Европы
    Попыталась создать универсальную модель, которая анализировала бы закономерности исторического процесса
    Продуктивно изучать общество на основе экономики
    \item
    Не подходит для стран востока и России, следовательно не является универсальной.
    Практически не изучали влияние других факторов на исторический процесс
    Т.к. эта модель стала доминирующей в СССР, произошли фальсификации, искажения исторического процесса, так как история России в эту модель не вписывается.
\end{enumerate}
\subsubsection{Этногенетическая}
\begin{enumerate}
\item Л. Н. Гумилев. 1930е годы. Существуюет с 90х годов
\item Динамика развития этноса, этногенез. Базовые термины:
    Этнос - это большая группа людей, которая отличается от других групп, четко обозначеной и осознаваемой целью своего существования. \\
    Пассионарность - это особый вид интеллектуальной энергии. Способность отдельных личностей формулировать цели развития этноса и организовывать большие массы людей для достижения этих целей.
\item Схема:
    \begin{enumerate}
        \item Появление идеи 
        \item Пассионарный толчок
        \item Формулирование идеи 
        \item Появление этноса 
        \item Развитие этноса 
        \item !TODO FIX THIS
    \end{enumerate}
    Субпассионарии 
\end{enumerate}

Дома к семинару сформулировать плюсы и минусы данной концепции.

\subsubsection{Цивилизационная}
\begin{enumerate}
    \item О. Шпенглер, А. Тойнби, Н. Я. Данилевский
    \item социо-культурные ценности идеи, идеологии культуры
    \item Нет общего единого универсального определения понятия цивилизации. Есть только 3 критреия, которые всем понятны
    Высокий уровень развития общества
    Каждая имеет свой уникальный путь исторического развития
    Одна цивилизация принципиально отличается от другой базовой системой идей ценностей.
    \item Кружки с цивиоизациями греч вост европейская
    \item Вообще не изучаеют общество находящее на ранних стадиях развития
    Так как нету единого определения цивилизации, следовательно нет общих критериев для их сравнения. Следовательно модель вообще не изучает закономерности исторического процесса, следовательно не может являться универсальной.
    Изучать отдельно историю какой-то страны невозможно потому что н а нее постоянно влияют какие-то бесконечные внешние факторы.
    \item Сформулировать свои размышления
\end{enumerate}

\subsubsection{Новая социальная история. История повседневности. Французская школа анналов}
\begin{enumerate}
    \item М. Блок и Л. Февр
    \item быт, нравы, духовная атмосфера эпохи
    \item Историю нельзя загонять в заданные схемы развития. -
    \item Изучая изменение в повседневной жизни. Анализируют влияние на нее различных факторов социально-экономических и духовных. Следовательно это попытка провести комплексное исследование исторического процесса.
%комплекный подход к изучению истории. Влияние человека на исторический процесс. Совокупность факторов.
    \item Очень большой объем материала, информации. Нету четкой методики, часто не доводятся до конца.
\end{enumerate}

\subsection{Третий шаг исторического исследования}
Сравнение источников и выводов научных школ и парадигм на основе принципа историзма. Метод анализа причинно следственных связей, который анализирует количественные и качественные изменения в каждом этапе и дает оценку причин и последствий этих изменений.

\section{План подготовки к семинару}
\begin{enumerate}
\item Политическая раздробленость на Руси.
\item Причины раздробленности (Логика от концепций)
    Социально-экономические Причины.
    Духовные Причины.
    Политические причины внутренние/внешние.
\item Сущность раздробленности(Определение, хранические рамки, сравнение с аналогичным процессом в других странах)
\item Последствия раздробленности
\item Точки зрения на раздробленность. +-. Собственное мнение.
\end{enumerate}

\section{Тема 2. Проблема происхождения государства у восточных славян}
Это одна из самых горячо обсуждаемых проблем российской истории, так как отсутсвуют прямые исторические источники по этому периоду.
Единственный источник это повесть временных лет, это источник XII века. А повествует о событиях IX века. И в этом источнике приводится легенда о призвании варягов.
Славяне по причине междоусобиц в 862 году пригласили варягов - "Русь"
Рюрик начал княжить в Новгороде. Трувор в Изборске. Синеус на Белоозере.
А Оскольд и Дир пошли на греки и начали княжить в Киеве

Все ученые которые обсуждают эту проблему обсуждают следующие вопросы:
\begin{enumerate}
    \item Было ли призвание варягов и кто они такие?
    \item Как трактовать термин "Русь"?
    \item Кто, когда и почему образовал государство у Восточных славян?
\end{enumerate}

\subsection{Норманская концепция}
Авторы этой концепции: Байер, Миллер, Шлецер. Немецкие ученые 18 века
\begin{enumerate}
    \item Да, естественно было и ПВЛ об этом однозначно свидетельствует
    Варяги это нормано-скандинавы, которые проживали по северному побережью Балтийского моря
    \item Русь это самоназвание того племени варягов, которое пришло.
    \item Государство образовали 862 г. приглашенные норманы во главе с Рюриком, т.к. у них уже на тот момент уже существовало государство и они принели этот опыт с собой чтобы прекратить междоусобицы.
\end{enumerate}
Выводы этой концепции сразу приобрели политический характер и стали использоваться врагами россии для обоснования неполноценности славянской нации(сами даже не смогли организовать государсво).

\subsection{Антинорманская концепция}
Дореволюционная Норманская концепция. М. В. Ломоносов.
Реакция на норманскую концепцию. 
\begin{enumerate}
    \item + Призвание варягов было. Варяги происходит от славянского Вар. что означает море. Этим термином обозначали поморских славян. Которые проживали по южному побережью Балтийского моря. И которых поморские славяне, которых восточные славяне пригласили для оказания военной помощи. 
    \item Термин Русь - видимо название племени поморских славян. Нет объяснения. 
    \item Государство образовал в 882 году князь Олег. Так как во время похода из новгорода в Киев, он завоевал, объединил все славянские племена проживавщие по днепру и сформировал общую территорию и перенес столицу в Киев.
\end{enumerate}
Ломоносов упустил в своей концепции очень много проблем и в 30х годах XX в. немецкие историки используют эти пробелы и будут доказывать несостоятельность антинорманской концепции.
Проблемы этой концепции:
\begin{enumerate}
    \item Против кого пригласили оказывать военную помощь в Новгород, если главный враг (Хазарский каганат) находится на юге?
    \item Если варяги это поморские славяне, то почему данные археологических раскопок курганов первых киевских князей показывают доминирование норманской, скандинавской культуры?
    \item А кто такие Оскольд и Дир. Какова их роль в образовании государства?
\end{enumerate} 

\subsection{Советская антинорманская концепция. Б.А. Рыбаков.}
\begin{enumerate}
    \item Призвание варягов было. Варяги это нормано-скандинавы. О чем свидетельствуют ПВЛ и археологические источники. Норманов пригласило племя полян, которые жили в районе Киева, для оказания военной помощи против Хазарского каганата. Но увидев междуусобицы варяги этим воспользовались и завоевали незащищенные города по побережью Балтийского моря и начали в них княжить. А часть дружины отправили в Киев против Хазар.
    \item Термин Русь происходит от названия притока днепра реки Рось.
    По которой проживали поляне и варяги примут на себя это название и под ним потом войдут во все русские источники.
    \item Легенда о призвании Варягов вообще не имеет никакого отношения к процессу образования государства. Так как ни одна личность не может влиять на исторический процесс, если для этого не сложились необходимые исторические предпосылки. К моменту призвания варягов у славян уже были все необходимые предпосылки для образования государства.
\end{enumerate}

\subsubsection{Комплексная концепция}
Исаев. Юрганов. Кацва. 
\begin{enumerate}
    \item однозначно нормано-скандинавы. И ПВЛ и арх. источники и лингвистический анализ. Они проводят лингвистический анализ имен первых киевских князей и приходя к следующим выводам что все имена имеют нормано скандинавское происхождение. Они делают следующие выводы:
    Они признают историчность Рюрика, что он существовал, и считают что главный его вклад в основании первой княжеской династии на Руси и все последующие князья будут Рюриковичи. Они считают что Синеус и Трувор - вымышленные персонажи. неправильный перевод ``меч и дружина''.
    \item
    Термин Русь происходит от скандинавского ``ротс''(Гребцы) и этим термином славяне всегда называли представителей норманских племен.
    \item Вывод, к 862 году у славян уже существовало большинство признаков государства. А завершат процесс образования государства первые киевские князья. И окончательно оно сформируются только к концу правления святослава.
\end{enumerate}

\subsection{Признаки государсва}
\begin{enumerate}
    \item + Общность территории
    \item + Публичная власть: Аппарат управления, Аппарат принуждения.
    \item + Налоги
    \item +-? Формально определенная система права.
    \item -+ Суверенитет. Внутренний/Внешний.
\end{enumerate}

К началу XIII века на территории по Днепру уже существовало три крупных конфедеративных союза восточных славян. 
\begin{enumerate}
    \item Славия со столицей в Новгороде. 
    \item Иордания со столицей в Смоленске. 
    \item Куявия со столицей в Киеве
\end{enumerate}

\subsubsection{Территория}
Восточные славяне объединялись так как через их территорию проходил торговый путь из варяг в греки. Единственный в то время путь, который связывал Западную Европу и Азию. И Славяне объединялись для защиты этого торгового пути от нападения чтобы получать таможенные пошлины.

\subsubsection{Аппарат управления}
На рубеже VII-VIII веков у Славян происходит распад родовой общины. И формируется общественный строй который именуется ``Военная демократия'', это система управления когда верховная власть принадлежит народному собранию вече. В работе которого могут участвовать только мужчины, которые защищают племя. В промежутках между созывами речи власть переходит к совету старейшин.

\subsubsection{Аппарат принуждения}
Князь, НО Он не управляет, это только военный руководитель и плюс по делегированию вече он начинает выполнять судебные функции.

Налоги собирались с середины VIII века в форме полюдья. Его главная особенность в том, что размер не был четко фиксирован, четко определен и не князь решал сколько взять, а община сколько дать.

Первым из дошедших до нас русским законом является русская правда принятая при Ярославе Мудром. (только в 30х годах XI в.) Но в первых международных договорах Руси и Византии 911 и 941 гг. есть ссылки и приводятся статьи из закона Росикова. Видимо в XI в. существовал свод законов, который у нас не сохранился.

\subsubsection{Суверинитет}
Внутренний суверенитет - это способность государственной власти принимать решения и проводить самостоятельную политику. Не существовал.
Внещний суверенитет - признание другими государствами в качестве самостоятельного независимого партнера.
Существовал так как племена объединялись для защиты своей территории и торгового пути. Существовала какая-то ситсема торговых отношений с другими странами.

Дома сформулировать собственную позицую с 2 аргументамя по этой проблеме.
Древнерусское государство. Проблема социально экономического развития киевской руси.
Кивеская Русь это древнее государство, которе существовало с XI в. до 30х годов XII в. И которое стремилось к проведению единой, централизованной политики. Самая Дискусионная проблема - сложился ли в Киевской Руси феодализм. А если нет, то какой тогда существовал социально-экономический строй.
Феодализм - это экономический строй(ОЭФ), при которой владелец земли феодал эксплуатирует экономически зависимых крепостных крестьян.
Признаки: 
\begin{enumerate}
    \item Наличие класса владельцев земли феодалов
    \item Наличие класса крестьян, которые арендуют часть земли у феодала и за пользование этой землей платят феодальную ренту(барщина и оброк).
    \item Наличие системы крепостного права. Это законодательное прикрепление крестьян к земле. По которому им запрещается переход от одного феодала к другому.
\end{enumerate}

\section{Тема 2. Сущности раздробленности}
Дискуссии о сущности раздробленности:
Политическая или феодальная.


\subsection{Тема номер 2. Вопрос 1. Сущность раздробленности}
Ищи три модели политического развития русских княжеств.
В этот период система управления во всех княжествах представляла собой взаимодействие трех элементов: князь, вече и боярская дума. Которые боролись за власть друг с другом и в зависимости от доминирования того или иного элемента сформировались три модели политического развития.

\subsubsection{Владимиро-Суздальское княжество}
Формируются предпосылки для перехода к самодержавной монархии. Главной в этой системе становится князь, а два других ему подчинялись.

Найти как формировалась сильная княжеская власть. В рамках трех правлений.
\begin{enumerate}
\item Ю. Долгоруков.
\item А. Боголюбского.
\item В. Большое гнездо
\end{enumerate}

\subsubsection{Новгородское княжество}

\subsubsection{Галицко-Волынское княжество.}
Олигархическая ограниченая монархия. 

Вопрос на ПК1: Причины сущности последствия раздробленности на Руси.

\subsection{Тема 2. Вопрос номер 2}
Русь между Ордой и католической Европой.

Главным негативным последствием раздробленности стало ослабление обороноспособности русских земель.
В 30-40х годах 12в. русь одновременно подвергнется нападению с двух сторон.

C запада шла волна католической экспансия. Крестовый поход объявленный римским папой. С целью обращения русских земель в католическую веру. Две волны нашествия были остановлены Александром Невским. Который в 1240 одержал победу в Невской битве. в 1242 году в ледовом побоище. НО!!! Экспансия на этом не остановилась и в 1246 году Папа Римский объявил о начале нового крестового похода на Русские земли. Параллельно с этим русские земли подверглись нападению монголо-татар во главе с Батыем.

Монголо-татары разгромили сожгли ограбили русские земли, но первоначально не обложили их данью а пошли дальше в Западную Европу. Русские княжества после нашествия оказались перед проблемой выбора. Было понятно что одновременно с двумя грозами русские князья не справятся и они должны выбрать кого-то в качестве союзника.

\subsubsection{Две позицие русских князей}


\textbf{Первая позиция}
\begin{enumerate}
    \item Даниил Галицкий(Галицко-Волынский князь, Киевский регент)
    \item Андрей Юрьевич.
\end{enumerate}
Они выступили за союз с католической европой и следовательно за принятие католического христианства, чтобы образовать обще-европейский союз против монголо-татар. В 1253 году Даниил Галицкий отказался принять ханского наместника и подписывать договор. По этой причине состоялась еще одна волна нашествия. Котораятполучила название Невьюрова рать.
В результате второго нашесвия в Киевском и Галицко волынском княжестве было установлено прямое правление монголо-татар.


\textbf{Вторая позиция}
Александр Невский. Великий Владимирский князь. Он считал что необходимо заключать союз с монголо татарами, так как еще одной волны нашествия русские земли не переживут. И считал монголо татар меньшим злом по сравнению с западной европой. Так как они веротерпимые. И союз позволит сохранить основы русской православной культуры и потом на этой основе начать восстановление государсва. 

Он сам совершает две поездки 1253, 1262 к монголо-татар. В знак добрых намерений отдал в протекторат Новгород.
В 1253 году подписал с ними договор, по которому согласился платить ордынский выход - дань 10\% со всех княжеств. Рекрутов 1 от каждого из 10 дворов. Плюсом к этому он согласился принимать ханский ярлык и тем самым по документам он признал что русские земли попадают в состояние вассальной зависимости от золотой орды. Следствием этого выбора Становится РАСКОЛ русских земель.

\textbf{Южные княжества}
\begin{enumerate}
    \item Киевское
    \item Галицко-Волынское
    \item Турово-Пинское(Украина и Белорусия)
\end{enumerate}

Отказались подписывать договор с монголо татарами и чтобы им противостоять вошли добровльно в состав великого княжества Литовского.
В 14 в. это княжество подпишет договор с польшей и объедениться в единое государство Речь Посполитая.

Северо-Восточные земли примут вассальную зависимость от Золотой Орды сформируется так называемое иго.

\subsection{Дискуссия о монголо татарском иге}
Все ученые обсуждают два вопроса:
\begin{enumerate}
    \item Было ли иго
    \item Какое влияние оно оказало на русские земли
\end{enumerate}

Есть три точки зрения

\subsubsection{XIX В. Н.М. Карамзин}
\begin{enumerate}
    \item +
    Он доказывал что иго было, и предложил классическое определение этого понятия.
    Иго, с его точки зрения это система экономической(платили дань) и политической(получали ярлык на княжение) и принимали ханских наместников(баскаков).
    \item Последствия ига оценивал положительно так как считал что необходимость его свержения способствовала объединению князей. И более быстрому раннему началу процесса централизации, для которой никаких других предпосылок свержения ига не было.
\end{enumerate}

\subsubsection{Н. Гумилев}
\begin{enumerate}
    \item Ига, в том значении, которое дает Карамзин на Руси не было. Т.к. термин иго в значении подчинение и зависимость стал использоваться в русском языке только с Петра I. До этого на старославянском. Термин иго переводился как союз или взаимодействие. Следовательно был союз добровльно заключенный Александром Невским для противодействия католической европе.
    С этой точки зрения дань это плата за союзную помощь. Рекруты это армия которая охраняла российские границы. Вторая цель этого союза сохранение православия и пока монголо татары были веротерпимыми союз сохранялся. Однако в середине 14 в. государственной религией становится ислам, который должны принять все улусы. Попыткой противостоять принятию ислама стала Куликовская битва 1380 года. И так как русские в ней победили. Им разрешили сохранить православие. Союз сохранился еще на 100 лет.
    \item Оценивал положительно, так как православие сохранили и на этой основе создали централизованное российское государство. А те земли, что вошли в состав княжества Литовского, в итоге подверглись катализации.
\end{enumerate}

\subsubsection{Ключевский, Рыбаков} 
\begin{enumerate}
    \item Монголо татарское иго было. Та же позиция, что выдвинул Карамзин.
    \item Отрицательное. У монголо татар русские князья унаследовали восточно-деспотичный характер самодержавной власти и т.д. Смотри учебник.
\end{enumerate}

\section{Тема 3}
\subsection{Этапы централизации русских земель}
\subsubsection{1301 - 1389 Борьба между Москвой и Тверью за лидерство в объелинении русских земель.}
\textbf{Базовые даты.}
1327 год - Иван Калита, подавив востание в Твери, получил ярлык на великое княжение для московского княжества и право на пожизненый сбор дани со всех русских земель. Часть денег он оставлял себе, на эти деньги он покупал соседние земли.
Территория москвы увеличилась в 27 раз.

1380 год - победа в Куликовской битвы. Главный результат - 1389 год завещание Дмитрия Донского, по которому он оставляет московскую княжество как отчину, не спрашивая ярлыка в орде, и орда этот факт признает.

\subsubsection{1389-1462 Период феодальных войн между Московскими князьями. }
Среди князей было две позиции.
Одна позиция - сторонники централизации, Василий II Темный - сын Д. Донского.
Вторая позиция - сторонники раздробленности,
Юрий Звенигородский, брат Д.Донского.
Этот период завершился приходом к власти Василия II Темного.

\subsubsection{ 1462-1533 период завершения объединения русских земель вокруг москвы и начала формирования централизованного аппарата управления.}
1584 год - завершение формирования централизованного аппарата управления, завершение процесса централизации при Иване Грозном.


\section{Россия в XVII веке}

\subsection{Смутное время в России. Дискуссии историков}
Главная дискуссия по поводу причин, сущности. Последствия, этапы

Две точки зрения

\subsubsection{Советская: Смирнов, Федоров и Буганов}
Смута - это высшая форма борьбы угнетаемых крепостных крестьян против эксплуататорских классов.
Причины:
Экономические: усиление крепостного права, как следствие обострение социальных противоречий и как следствие начало народных бунтов и крестьянских восстаний.
Высшая точка или пик смутного времени: "крестьянская война" под руководством Болотникова. 1606-1607 годов. Максимальное вовлечение крестьян.
\subsubsection{Современная: Кобрин и Скрынников}
Смута - первая гражданская война так как в событиях этого периода участвовали все социальные слои выступая друг против друга. 1598 - 1613
Причины:
    \textbf{Системный кризис}. Кризис который одновременно охватил все сферы общественной жизни. Он стал следствием поражения России в Ливонской войне(1558-1582) и разорением южных регионов+Москвы. 1571-1572 Девлет-Герей. Следствием всех этих событий стало сокращение на 70\% посевных площадей в северо-западных и южных регионах страны. Следствием этого становится массовое бегство крестьян на восток, голод 1601-1603.

    \textbf{Социальный кризис}. Все слои общества были недовольны своим положением.
            \begin{enumerate}
                \item Следствием массового бегства крестьян на восток стало принятие двух указов 1581г указ ``о заповедных летах'', по которому временно запрещали крестьянские переходы в Юрьев день. Фактически этот закон означал введение крепостного права. Указ 1597 года ``об урочных летах'', срок от 5 до 10 лет поимки беглых крестьян. Следствием принятия этих двух указов становится массовое недовольствие всех крестьян. 

                \item Бояре в период опричнины потеряли значительную часть своих привилегий и по военной реформе Иван Грозный обязательно заставил их нести военную службу за землю. Следовательно они хотят вернуть свой привиллегированный статус в обществе.

                \item Дворяне, с одной стороны выдвинулись на высшие государственные посты, но с другой стороны их статус никак не был закреплен. Они требуют уравнивание поместья с вотчиной.

                \item Городское население. В ходе реформы 1549 года Иван Грозный ввел в России земские соборы - выборный законо совещательный орган центральной власти, в работе которого участвовали представители всех сословий, кроме крепостных крестьян. Городское население впревые получило возможность влиять на политику государства. Однако в период опричнины созыв земских соборов и выборный принцип формирования органов местного самоуправления был отменен. Следовательно, городское население выступает за возращение себе все этих прав и полномочий

                \item Церковь. В рамках стоглавого собора 1551 г Иван Грозный запретил рост церковного землевладения.
            \end{enumerate}

        3) Политический кризис. 
            \begin{enumerate}
                \item Внутри политический был связан с династическим кризисом власти. Федор Иванович (1584-1598) - умолишенный, регентом был Борис Годунов. Умер не оставив наследников. Земский собор 1598 года выбрал царем Бориса Годунова. Однако население его в качестве законного правителя не признало.
                \item Внешнеполитический кризис следствием неудачной политики Ивана Грозного стало то что три страны Польша, Швеция и Крымское ханство имели территориальные претензии к России и они попробуют решить эти претензии в период смуты. 
            \end{enumerate}

\subsection{Основные этапы смуты}
\subsubsection{Династический кризис. 1598-1606 гг.}
Население в этот период было готово признать на российском престоле любых самозванцев, но не представителей законной власти.
Закончился тем, что в 1605 году Москву захватил Лжедмитрий I. (1605-1606 гг.)
Попытался провести реформы направленные на модернизацию России по европейскому образцу, настроил против себя все общество. 
\subsubsection{Социальный кризис. 1606-1610 гг.}
Так как в этот период в события смуты оказались одновременно вовлечены все социальные слои и все регионы страны. События разворачиваются по 4 параллельным процессом.
1606-1610. Новым русским царем бы избран Василий Шуйский. Во время выборов он подписал закон с боярской думой. С подписанием этого соглашения в России сложилась ограниченная олигархическая монархия.
Второй параллельный процесс - "Крестьянская война" - Крестьянские выступления по руководством Болотникова и предпринял поход на Москву и был разгромлен под Коломной.
Третий процесс - появление Лжедмитрия II. 1610 г. дошел до г. Тушино но напасть на Москву так и не решился. 
Четвертый процесс - начало открытой польско-шведской интервенции. 1609-1610 гг. 
Свержение Василия Шуйского
\subsubsection{Национальный кризис. 1610-1613 гг.}
В этот период была реальная угроза потери суверенитета и прекращение существования России.

Власть переходит к боярской думе. Семибоярщина. Они подписывают соглашение с Польшой, приглашают на престол польского царевича Владислава.
Параллельно с этим идет шведская интервениция. Они захватили Новгород, все северо-западные земли и к 1612 г. дошла до смоленска.
Попытка создания первого народного ополчения под руководством Прокопий Липунов потерпело поражение.
1612г было создано второе народное ополчение под руководством Минина и Пожарского, которые четвертого ноября 1612 года выйграли битву под москвой и остановили польскую интервенцию.

\subsection{Последствия смуты}

\subsubsection{Внутриполитические}

В 1613 году земский собор выбирает на царствование нового царя - Михаила Романова. Начинается сращивание государственной и церковной власти. Усиливаются самодержавно-деспотические тенденции. При Алексее Михайловиче(1645-1676) в России начинат формироваться предпосылки абсолютной монархии. 

\subsubsection{Внешнеполитические}
 Россия потеряла огромные территории все Северо-Западные земли до Смоленска отошли к польше и все северные земли и выход к балтийскому морю отошли к швеции.

 \subsubsection{Экономические}
 Окончательное закрепощение крестьян и в 1649 г. принимается соборное уложение, которое окончательно отменяет Юрьев день и вводит бессрочный срок поимки беглых крестьян.

 \subsubsection{Социальные последствия}
 Возвышение позиций дворянства. Будут приниматься указы которые будут уравнивать статус поместья и вотчины. К концу XVII поместье=вотчина. 
 Закрепощение крестьян. 

 \subsubsection{Духовные}
 Церковь в ходе всех этих процессов окончательно подчиняется государству. Фактически происходит сращивание государственных аппаратов.
 Происходит падение авторитета царской и любой власти вообще. Остаток XVII в называют "Бунташный век".

... Проебал начало надо спиздить у кого нибудь

\section{Тема ....}
Главное следствие - проблема выбора каким путем дальнейшего развития идти, так как в России только окончательно оформляется феодализм и крепостное право, а в западной европе параллельно с этим идут процессы развития капитализма и начало первых буржуазных революций.
Следовательно в россии оформляется система крепостного права

С одной стороны оформился феодализм Под влиянием новых заподно европейских технических открытий и западно европейских буржуазных революций.

Капитализм - это экономический строй при котором владельцы средств производства(фабрики заводы и оборудование) буржуазия эксплуатирует лично свободных наемных рабочих, которые живут за счет продажи своего труда

\textbf{Признаки капитализма}
\begin{enumerate}
\item Товарное хозяйство
\item Формирование рыночной экономики
\item Наличие капитала, который позволяет развивать переходить к фабрично машинному производству.
\item Свободные наемные руки
\item Конкуренция товаропроизводителей, которая способствует развитию предпринимательства а сдругой стороны стимулирует развитие рыночной экономики.
\end{enumerate}

\subsection{Особенности социально экономического строя в XVII веке}
\begin{center}
    \begin{tabulary}{0.7\textwidth}{|c|c|c|}
        \hline
        Новые капиталистические явления & Феодальные черты \\ \hline
        Сельское хозяйство & Хозяйственная специализация районов (Например поволжье специалищируется на хлебе, а центральные регионы на животноводстве) & Оформляется крепостное право, в следствие чего осуществляется переход к экстенсивному способу ведения хозяйства. Феодалы начинают получать основной доход только за счет феодальной ренты. Путь усиления эксплуатации.
        Самая главная проблема - в соборном уложении не был ограничен размер феодальной ренты. Следствием роста барщины становится череда экономических кризисов. \\
        Промышленность & Начинается промышленная специализация районов(Тула оружие). Появляются первые мануфактуры - первый тип промышленного предприятия с разделением труда, с примитивными первыми механизмами и которая производит продукцию для продажи на рынке. Считается \textbf{первым типом} экономического предприятия. & Доминирует ремесленное производство - это тип предприятия без разделения труда, без механизмов, которое производит продукцию на заказ а не для продажи на рынке. Не развивается товарное хозяйство. Но в России мануфактуры не стали капиталистическими предприятиями так как: Все мануфактуры были государственными или работали на госзаказ. Не способствовали развитию конкуренции и торговли. Базировались на феодальном способе производства. Нет свободных наемных рук. Бесплатный труд крепостных крестьян \\ \hline
        Торговля  & Начинает формироваться всероссийский рынок, уникальная экономическая система, в рамках которой два раза в год, через систему всероссийских ярмарок осуществляется обмен товарами произведенной в ходе специализации районов & Доминирует натуральное хозяйство, товарное хозяйство и торговля не развивается \\\hline
        Внешняя торговля & Россия перешла к политике протекционизма & В ходе смуты россия потеряла выходы ко всем морям, следовательно всю внешнюю торговлю, контролировали иностранные купцы 1653. Россия доходов не получала.\\     
        \hline
    \end{tabulary}
\end{center}

Таким образом проявления капитализма были крайне незначительными, доминировал феодализм. Противоречия между феодальным и капитализмом станет базовым противоречием Российского общества вплоть до октября 1917 года.
    
\begin{center}
        \begin{tabulary}{0.7\textwidth}{|c|c|}
        \hline
        Либеральные & Консервативные \\
        \hline
        Те правители которые способствуют развитию капитализма  & Те кто закреплять или усиливать феодализм и ликвидировать элементы капиталистического строя.
        \hline
        \end{tabulary}
\end{center}
Проанализировать политики Екатерины II, Павла I, Александра I, Николай I по плану.

\begin{enumerate}
    \item Политика в крестьянском вопросе. Общее направление экономической политики
    \item Политика в дворянском вопросе. Дворянские права и привилегии.
    \item Глобальные изменения. Проекты глобальных изменений в системе гос управления
    \item Гиперуспехи или гипернеудачи во внешней политике.
    \item Диаметрально противоположные точки зрения по оценке этих правления с аргументами
    \item Сформулировать собственное мнение с четырмя аргументами.
\end{enumerate}

\section{Проблемы социально экономического и политического развития россии в XVIII-XIX веках}

\subsection{Оценка правления Петра I(1682/1696-1725)}

Сложилась в XIX в. в рамках полемики между западниками и славянофилами. Соотвественно получается что западнки оценивали правление Перта I исключительно положительно. Их главный вывод - реформы Петра I способствовали модернизации России по западному образцу.

Славянофилы оценвиали правление исключительно отрицательно и приводили три основных аргумента:
\begin{enumerate}
    \item Внедрил западно-европейскую культуру. Что стало началом кризиса традиционной русской православной культуры.
    \item Ликвидировав земские соборы разорвал традиционную связь царя и народа.
    \item Создав бюрократический аппарат управления, разрушил традиционно обищинную систему управления.
\end{enumerate}

\subsubsection{Советская точка зрения}
В. Сталин.
Краткий курс истории 1936 года.
\begin{enumerate}
    \item ``Петр первый великий реформатор''
    \item ``Прорубил окно в европу''
\end{enumerate}
Минус - насильственные методы проведения реформ

\subsubsection{Современная точка зрения}
Нет однозначной оценки петра первого. Принято выделять плюсы и минусы в его правлении.
Провел или нет модернизацию россии по западному образцу???

\begin{center}
\begin{tabulary}{0.7\textwidth}{|c|c|c|c|}
\hline
- & Западная Европа & + & - \\
Экономика & Начало промышленного переворота, переход от мануфактурного к фабрично-заводскому производству, следовательно капитализм & Для обеспечения армии и флота Петр I стимулировал развитие российских мануфактур и тем самым он заложил основу для развития легкой текстильной и тяжелой металлургичесокй промышленности & Большинство мануфактур государственные и работают на гос заказ, развиваются на гос. средства и следовательно частный капитал не развивается. Для работы на мануфактурах Петр I вводит две категории крестьян. 1703 г. Приписные крестьяне - три дня в неделю бесплатной работы на мануфактуре. Низкая производительность труда, низкое качество работы. Рабочий класс не формируется. в 1721 году - Посессионные крестьяне - собственность мануфактур - три дня работил в счет уплаты гос. налога. три дня в счет мануфактры и только один день на себя. Не отменил крепостное право, а наоборот усилил процесс закрепощения крестьян. \\
Социальная сфера & Идет политика ликвидации сословий и принятия законов которые способствовали широкой социальной мобильности населения & +сословного строя, сословной политики. Петр I мечтал создать государство всеобщего блага, в котором все сословия служат государству и все платят какой-то вид налога. 1714 год. Указ о единонаследии, по которому поместье уравнивается с вотчиной. Может передаваться по наследств только старшему сыну, а все остальные дворянские дети должны нести пожизненную службу государству. 1703 г. Ликвидировался институт холопства. Все крестьяне были либо государственными - Лично свободные крестьяне, которые платят только гос налог, либо крепостными & Только только окончательно оформляется сословный строй и четко определяются права и обязанности каждого сословия. Оформление сословного строя. \\
Политическая сфера & a) Период начала буржуазных революций => ограничение самодержавия(Парламентская монархия в Англии) или переход к буржуазным демократическим республикам(Нидерланды). б) Уравнивание всех перед законом & а) Создал эффективную систему центрального и местного самоуправления, которая с незначительными изменениями до 1905 г. б) Ввел в стране систему контрольно-надзорных и репрессивных органов. Фискальная служба и генеральная прокуратура. & а) В россии сложился военно-полицейский тип абсолютной монархии, которая характеризовалась установить абсолютный контроль над всеми сферами жизни общества и государства. Многочисленные законы детально регламентировали жизнь граждан б) \\
Духовная сфера & отделение государсва от церкви, свобода совести, свобода вероисповедания. Как следствие переход к светской культуре. Формирование научно рационалистической картины мира, развитие образования, 
науки и техники. Доминирующими становятся идеи антропоцентризма & способствуют проникновению в России научно технических открытий, вводит европейскую систему образования для дворянства, способствует распространению Европейской бытовой культуры среди дворян, в результате всего это начинается преодоление религиозно-патриархальных устоев общества, закладываются основы светской культуры & 1) ----
2) 3) Только для дворянтсва. В результате этих реформ произошел разрыв между дворянской культурой и культурой всего остального общества, что заложило основу для культурного кризиса. 4) Все культурные преобразования носили внешний характер и не затронули базовых ценностей российской культуры. Идея антропоцентризма вообще не получила никакого распространения.
\hline
\end{tabulary}
\end{center}

Реформы Петра I. не способствовали модернизации России по западному образцу, так как меняла только отдельные элементы системы не меняя феодальную структуру общества в целом, наоборот усиливали феодализм.

\section{Эпоха дворцовых переворотов}
Период в истории России 1725-1762 г. Который характеризовался борьбой дворянских группировок за власть.

Группы
    "Новые дворяне" - дворяне которые выделились при Петре I, благодаря табелю о рангах. Не знатные, не потомственные хотели сохранить свое положение и выступали за продолжение Петровских реформ. 
    Меньшиковы, Бестужие, Лопухины, Орловы, Толстые

    "Старые дворяне" - потомственное родовитое дворянство, имеющие земли. Которое в период правления Петра I утратило свои позиции. Было отстранено от управления государством. Выступало за отмену табеля о рангах. Ликвидацию нового дворянства, свертование петровских реформ, возврат собственных привиллегированный
    Голицины, Долгорукие, Минихи, Остерманы

Схема дворцовых переворотов.
Формальным поводом к череде дворцовых переворотов был указ о престолонаследии 1724 года, что царь сам может выбирать любого наследника. 

Первый дворцовый переворот.
Совершили новые дворяне во главе с Меньшиковым, посадив на престол его жену Екатерину I 1725-1727. При ней был Верховный тайный совет, во главе с меньшиковым, который стал органом реальной власти.
Екатерина I продолжала политику Петра I. Умирая она оставила престол внуку Петра I - Петру II 1727-1730. 
Второй дворцовый переворот произойдет через полгода вошествия на престол.

Совершат его князья Голицины и Долгорукие, они отправят меньшикова в сибирь, сами возглавят Верховный тайный совет и начнут политику по свертыванию Петровских реформ. Столица перенесена обратно в Москву. Прекратится строительство и финансирование флота. Будет ликвидирован сенат.
После внезапной смерти Петра II Князья Долгорукие приглашают племянницу Петра I - Анну Иоановну, они предложили ей подписание кондиций, это соглашение по которому вся реальная власть передавалась Верховному тайному совету, а она оставалась номинальной правительницей.

Третий дворцовый переворот.
Произошел в день коронации. Совершили его новые дворяне.
Татищевы и Черкаские, Прокопович, Толстые. Анна Ионовна, несмотря на фаворита Бирона, тем не менее проводила политику на продолжение петровских преобразований. Вернула столицу обратно в петербург. Восстановила сенат. Ликвидировала верховный тайный совет.
В два раза увеличила расходы на содержание флота и армии. Передала власть сыну своей племянницы - Ивану VI(2 месяца, 1740-1741 гг.). Формально регентом при нем была Анна Леопольдовна. Реально провителем стал Бирон. 

Елизавета. Новые дворяне (Братья Орловы). Продолжает политику Петра I по всем направлениям. При ней новые дворяне получают самое большое число привиллегий. Оставила престол племяннику Петру III Федоровичу. (1761-1762). Настроил против себя все российское общество тем что отдал Пруссии все территории, которые были получены в ходе 7-летней войны.

Независимо от того какая дворянская группировка приходила к власти, каждая из них получала какие-либо особые права, привилегии. В результате к концу эпохи дворцовых переворотов дворянство трансформируется из служилого в привиллегированное сословие и окончательно их привиллигерованный статус закрепит жалованная грамота дворянству, которая будет дарована им Екатериной II в 1785 году.

\end{document}