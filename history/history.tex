\documentclass[a4paper]{article}
% Русский язык
\usepackage[utf8]{inputenc}
\usepackage[russian]{babel}
% Математические символы и формулы
\usepackage{mathtools}
% Разбиение на разные файлы
\usepackage{subfiles}
% Отступы с краев(margin)
\usepackage[margin=1in]{geometry}
% Форматирование заголовков
\usepackage{titlesec}
\usepackage{indentfirst} % Красная строка
    
\begin{document}

\section{Введение}

1 семинар - 1 тк (Лекции)
Д/з 1!, 3 вопросы (Проанализировать
политику первых князей по признакам гоударства)
\subsection{Литература}

\begin{enumerate}
\item Методичка 2208
"Отечественная история", Иваново 2014
\item Бумажные из библиотеки
    \begin{enumerate}
    \item Орлов и Георгиев "История России для тех. вузов"
    (Плюсы: вся фактура в одном месте, Минусы: Марксистский формационный подход)
    \item Личман "История России для тех. вузов"
    (Плюсы: есть концепции, Минусы: слабая фактура)
    \item "История России с др. времен до 19 в.", Кафедральный учебник
    (Говно, но нужно брать)
    \item "История России 1917 - 1945", Кафедральный учебник
    (Более приличная часть)
    \end{enumerate}
\item Найти(совет) "История России в схемах и таблцах"
\item Электронные учебники
    \begin{enumerate}
    \item "История России с др. времен до 19 в.", эл. ресурс
    (Посмотреть структуру)
    \item "История России 1917 - 1945", эл. ресурс
    \end{enumerate}
\end{enumerate}

\subsection{1 ТК}

Автор(Персона)
Названиние Концепии
Суть Концепции(Основной вывод по проблеме)
Базовые термины
Даты (к концепциям, принципиально важные)

\section{Тема 1}

\subsection{Основы методологии(как проводить историческое исследование)
 исторической науки}

\subsubsection{Методика(этапы) исторического исследования}

\begin{enumerate}
\item Анализ исторический источников
\item Изучение научных парадигм, концепций, разных точек зрения на эту проблему
\item Анализ первого и второго, если надо ищите дополнительные исторические факты
\item Формулируете собственную аргументированную точку зрения
\end{enumerate}

Исторический источник(созданный в ту эпоху, которую вы изучаете) - это любой памятник прошлого, который содержит исторический факт или информацию об историческом событии.
Группы исторический источников

\begin{enumerate}
\item материальные - археологические раскопки
\item письменные - повесть временных лет
\item изобразительные
\item лингвистические
\item кино, фото, фоно документы (XIX-XX в.)
\end{enumerate}

\section{Научные парадигмы}
Парадигма - концептуальная схема, модель, изучение исторических источников и трактовки исторических событий.
Выделяют пять базовых научных парадигм

\subsubsection{План}
\begin{enumerate}
\item Авторы
\item базовые факторы исторического развития
\item термины
\item модель исторического развития
\item историки исслеователи, представители этой парадигмы
\item +
\item - 

\end{enumerate}

\subsubsection{Религиозная(идеалистическая)}
\begin{enumerate}
\item Фома Аквинский V в.н.э.
\item провиденциализм - все события происходят по божеьей воле.
\item линейная и конечная
\item В.Н. Татищев - 18 в. (Основатель российской исторической науки),
Н.М. Карамзин
С.Н.Соловьев
В.О. Ключевский
\item Так как выразителем божественной воли является правитель, то историческое исследование будет выглядеть как история правления отдельных князей, царей и императоров.
\item - Недоказуема, так как основная на вере. Субъективна и однобока, так как изучает только один фактор - политику правителя
\item + Богатейший фактический материал по политической истории
    
\end{enumerate}

\subsubsection{Формационная}
Альт. названия:
    ОЭФ - общественно-экономическая формация
    Марксистская
    Материалистическая
    "Советская"
а) Фридрих Энгелься 40е годы 19 в.
К. Маркс 
б) Социально-экномические социальные факторы,
   Классы, классовая борьба(очень сильное сужение предмета исследования)
в) ОЭФ - ступень этап в развитии общества, который отличается от других этапов своим базисом(экономическим строем), который определяет особенности надстройки 
г) 5 базовых экономических формаций. Схема Ступеньки:
Первобытная
Рабовладельческая
Феодальная. Базовый фактор: земля. Ручной труд. Натуральное хозяйство. Базовые классы: феодалы, крепостные крестьяне.
Феодализм в россии 1649 г. Соборное уложение Алексея Михайловича.
Паровой двигатель => Капитализм. Преход от одной ОЭФ к другой осуществляется в результате социальной революции(изменение базовых классов общества). Изменение в средствах производства, новые технологии. Факторы производства: труд и капитал, в результате этого происходит переход к машинному индустриальному производству. Классы: буржуазия, пролетариата.

Два пути перехода: Кризис капитализма, или абсолютное большинство класса пролетариат.((Социализм)). Комунизм. Отсутсвие классов, всеобщее равенство. Формирование общественной собственности на средства производства. Нет небходимости в существовании государства. Базовый принцип: от каждого по возможности каждому по потребности.

Комментарии к схеме:
По Марксу преход к комунизму возможен при двух условиях. 
    Если капитализм себя изживет, прийдет к кризису, появятся новые технологии
    Если пролетариат составит 70 процентов населения страны, тогда он имеет право на революцию и установление диктатуры пролетариата.

    В. И. Ленин дополнил это учение учением о социализме, как осбой переходной стадии от капитализма к комунизму, цель которой увеличение численности пролетариата, завершение индустриализации. Т.к. в октябре 1917 года пролетариата в России 9\%, и следовательно объективных предпосылок для социалистической революции в россии не было.

Основатели советской исторической науки
    Б.А. Рыбаков

"Экономика определяет все" К. Маркс

В СССР формационная концепция стала доминирующей и единственно возможной и на этой модели основывалась вся советская историческая наука
+ 
    Идельно подходит для анализа стран Западной Европы
    Попыталась создать универсальную модель, которая анализировала бы закономерности исторического процесса
    Продуктивно изучать общество на основе экономики
-
    Не подходит для стран востока и России, следовательно не является универсальной. 
    Практически не изучали влияние других факторов на исторический процесс
    Т.к. эта модель стала доминирующей в СССР, произошли фальсификации, искажения исторического процесса, так как история России в эту модель не вписывается.

\subsubsection{Этногенетическая}
    Л. Н. Гумилев. 1930е годы. Существуюет с 90х годов
    Динамика развития этноса, этногенез
    Базовые термины: этнос - это большая группа людей, которая отличается от других групп,
    четко обозначеной и осознаваемой целью своего существования. Пассионарность - это особый вид интеллектуальной энергии. Способность отдельных личностей формулировать цели развития этноса и организовывать большие массы людей для достижения этих целей.
    
    Появление идеи
    Пассионарный толчок
    Формулирование идеи
    Появление этноса
    Развитие этноса
    Субпассионарии

    Дома к семинару сформулировать плюсы и минусы данной концепции.
\subsubsection{Цивилизационная}
О. Шпенглер
А. Тойнби
Н. Я. Данилевский
б) социо-культурные ценности идеи, идеологии культуры
в) Нет общего единого универсального определения понятия цивилизации. Есть только 3 критреия, которые всем понятны 
    Высокий уровень развития общества
    Каждая имеет свой уникальный путь исторического развития
    Одна цивилизация принципиально отличается от другой базовой системой идей ценностей.
г) Кружки с цивиоизациями греч вост европейская
- Вообще не изучаеют общество находящее на ранних стадиях развития
  Так как нету единого определения цивилизации, следовательно нет общих критериев для их сравнения. Следовательно модель вообще не изучает закономерности исторического процесса, следовательно не может являться универсальной.
  Изучать отдельно историю какой-то страны невозможно потому что н а нее постоянно влияют какие-то бесконечные внешние факторы.
+ ? Сформулировать свои размышления
\subsubsection{Новая социальная история. История повседневности. Французская школа анналов}
а) М. Блок и Л. Февр
б) быт, нравы, духовная атмосфера эпохи
в) - Историю нельзя загонять в заданные схемы развития.
г) 
+ Изучая изменение в повседневной жизни. Анализируют влияние на нее различных факторов социально-экономических и духовных. Следовательно это попытка провести комплексное исследование исторического процесса.  
%комплекный подход к изучению истории. Влияние человека на исторический процесс. Совокупность факторов.
- Очень большой объем материала, информации. Нету четкой методики, часто не доводятся до конца.

\section{Третий шаг исторического исследования}
Сравнение источников и выводов научных школ и парадигм на основе принципа историзма. Метод анализа причинно следственных связей, который анализирует количественные и качественные изменения в каждом этапе и дает оценку причин и последствий этих изменений.
\subsection{План подготовки }
Политическая раздробленость на Руси.
Причины раздробленности (Логика от концепций)
    соц экономические Причины
    духовные Причины
    политические причины внутренние/внешние
Сущность раздробленности(Определение, хранические рамки, сравнение с аналогичным процессом в других странах)
Последствия раздробленности
Точки зрения на раздробленность. +-. Собственное мнение.

\subsection{Проблема происхождения государства у восточных славян}
Это одна из самых горячо обсуждаемых проблем российской истории, так как отсутсвуют прямые исторические источники по этому периоду.
Единственный источник это повесть временных лет, это источник XII века. А повествует о событиях IX века. И в этом источнике приводится легенда о призвании варягов.
Славяне по причине междоусобиц в 862 году пригласили варягов - "Русь" 
Рюрик начал княжить в Новгороде. Трувор в Изборске. Синеус на Белоозере.
А Оскольд и Дир пошли на греки и начали княжить в Киеве

Все ученые которые обсуждают эту проблему обсуждают следующие вопросы
1) Было ли призвание варягов и кто они такие
2) Как трактовать термин "Русь"
3) Кто, когда (почему) образовал государство у Восточных славян. 

\subsubsection{Норманская концепция}
Авторы этой концепции 
Байер
Миллер 
Шлецер
Немецкие ученые 18 века
a) + Естественно было и ПВЛ об этом однозначно свидетельствует
Варяги это нормано-скандинавы, которые проживали по северному побережью Балтийского моря
б) Русь это самоназвание того племени варягов, которое пришло. 
в) Государство образовали 862 г. приглашенные норманы во главе с Рюриком, т.к. у них уже на тот момент уже существовало государство и они принели этот опыт с собой чтобы прекратить междоусобицы.
Выводы этой концепции сразу приобрели политический характер и стали использоваться врагами россии для обоснования неполноценности славянской нации(сами даже не смогли организовать государсво).

\subsubsection{Антинорманская концепция}
1   Дореволюционная Норманская концепция. М. В. Ломоносов.
    Реакция на норманскую концепцию. 
    + Призвание варягов было. Варяги происходит от славянского Вар. что означает море. Этим термином обозначали поморских славян. Которые проживали по южному побережью Балтийского моря. И которых поморские славяне, которых восточные славяне пригласили для оказания военной помощи. \\
    Термин Русь - видимо название племени поморских славян. Нет объяснения. \\
    Государство образовал в 882 году князь Олег. Так как во время похода из новгорода в Киев, он завоевал, объединил все славянские племена проживавщие по днепру и сформировал общую территорию и перенес столицу в Киев.
    Ломоносов упустил в своей концепции очень много проблем и в 30х годах XX в. немецкие историки используют эти пробелы и будут доказывать несостоятельность антинорманской концепции.
    Проблемы: 
        1 Против кого пригласили оказывать военную помощь в Новгород, если главный враг (Хазарский каганат) находится на юге?
        2 Если варяги это поморские славяне, то почему данные археологических раскопок курганов первых киевских князей показывают доминирование норманской, скандинавской культуры?
        3 А кто такие Оскольд и Дир. Какова их роль в образовании государства?

2   Советская Антинорманская концепция. Б.А. Рыбаков.
    Призвание варягов было. Варяги это нормано-скандинавы. О чем свидетельствуют ПВЛ и археологические источники. Норманов пригласило племя полян, которые жили в районе Киева, для оказания военной помощи против Хазарского каганата. Но увидев междуусобицы варяги этим воспользовались и завоевали незащищенные города по побережью Балтийского моря и начали в них княжить. А часть дружины отправили в Киев против Хазар.
    Термин Русь происходит от названия притока днепра реки Рось.
    По которой проживали поляне и варяги примут на себя это название и под ним потом войдут во все русские источники. Легенда о призвании Варягов вообще не имеет никакого отношения к процессу образования государства. Так как ни одна личность не может влиять на исторический процесс, если для этого не сложились необходимые исторические предпосылки. К моменту призвания варягов у славян уже были все необходимые предпосылки для образования государства.
\subsubsection{Комплексная концепция}
    Исаев. Юрганов. Кацва. 
    + Однозначно нормано-скандинавы. И ПВЛ и арх. источники и лингвистический анализ. Они проводят лингвистический анализ имен первых киевских князей и приходя к следующим выводам что все имена имеют нормано скандинавское происхождение. Они делают следующие выводы:
        Они признают историчность Рюрика, что он существовал, и считают что главный его вклад в основании первой княжеской династии на Руси и все последующие князья будут Рюриковичи. Они считают что Синеус и Трувор - вымышленные персонажи. неправильный перевод ``меч и дружина''.
    Термин Русь происходит от скандинавского ``ротс''(Гребцы) и этим термином славяне всегда называли представителей норманских племен.
    Вывод, к 862 году у славян уже существовало большинство признаков государства. А завершат процесс образования государства первые киевские князья. И окончательно оно сформируются только к концу правления святослава.
\subsection{Признаки государсва}
+ 1 Общность территории
+ 2 Публичная власть: Аппарат управления, Аппарат принуждения.
+ 3 Налоги
+-? 4 Формально определенная система права.
-+ 5 Суверенитет. Внутренний/Внешний.

К началу XIII века на территории по Днепру уже существовало три крупных конфедеративных союза восточных славян. 
    Славия со столицей в Новгороде. 
    Иордания со столицей в Смоленске. 
    Куявия со столицей в Киеве

Восточные славяне объединялись так как через их территорию проходил торговый путь из варяг в греки. Единственный в то время путь, который связывал Западную Европу и Азию. И Славяне объединялись для защиты этого торгового пути от нападения чтобы получать таможенные пошлины.

Аппарат управления
На рубеже VII-VIII веков у Славян происходит распад родовой общины. И формируется общественный строй который именуется ``Военная демократия'', это система управления когда верховная власть принадлежит народному собранию вече. В работе которого могут участвовать только мужчины, которые защищают племя. В промежутках между созывами речи власть переходит к совету старейшин.

Аппарат принуждения
Князь, НО Он не управляет, это только военный руководитель и плюс по делегированию вече он начинает выполнять судебные функции.

Налоги собирались с середины VIII века в форме полюдья. Его главная особенность в том, что размер не был четко фиксирован, четко определен и не князь решал сколько взять, а община сколько дать.

Первым из дошедших до нас русским законом является русская правда принятая при Ярославе Мудром. (только в 30х годах XI в.) Но в первых международных договорах Руси и Византии 911 и 941 гг. есть ссылки и приводятся статьи из закона Росикова. Видимо в XI в. существовал свод законов, который у нас не сохранился.

Внутренний суверенитет - это способность государственной власти принимать решения и проводить самостоятельную политику. Не существовал.
Внещний суверенитет - признание другими государствами в качестве самостоятельного независимого партнера.
Существовал так как племена объединялись для защиты своей территории и торгового пути. Существовала какая-то ситсема торговых отношений с другими странами.

Дома сформулировать собственную позицую с 2 аргументамя по этой проблеме.
Древнерусское государство. Проблема социально экономического развития киевской руси.
Кивеская Русь это древнее государство, которе существовало с XI в. до 30х годов XII в. И которое стремилось к проведению единой, централизованной политики. Самая Дискусионная проблема - сложился ли в Киевской Руси феодализм. А если нет, то какой тогда существовал социально-экономический строй.
Феодализм - это экономический строй(ОЭФ), при которой владелец земли феодал эксплуатирует экономически зависимых крепостных крестьян.
Признаки 
1   Наличие класса владельцев земли феодалов
2   Наличие класса крестьян, которые арендуют часть земли у феодала и за пользование этой землей платят феодальную ренту(барщина и оброк).
3   Наличие системы крепостного права. Это законодательное прикрепление крестьян к земле. По которому им запрещается переход от одного феодала к другому.

\section{Тема 2. Сущности раздробленности}

Дискуссии о сущности раздробленности:
Политическая или феодальная.

 
\subsection{Тема номер 2. Вопрос 1. Сущность раздробленности}
Ищи три модели политического развития русских княжеств.
В этот период система управления во всех княжествах представляла собой взаимодействие трех элементов: князь, вече и боярская дума. Которые боролись за власть друг с другом и в зависимости от доминирования того или иного элемента сформировались три модели политического развития.

\subsubsection{Владимиро-Суздальское княжество}
Формируются предпосылки для перехода к самодержавной монархии. Главной в этой системе становится князь, а два других ему подчинялись.

Найти как формировалась сильная княжеская власть. В рамках трех правлений.
\begin{enumerate}
\item Ю. Долгоруков. 
\item А. Боголюбского.
\item В. Большое гнездо
\end{enumerate}

\subsubsection{Новгородское княжество}
 
\subsubsection{Галицко-Волынское княжество.}
Олигархическая ограниченая монархия. 

Вопрос на ПК1: Причины сущности последствия раздробленности на Руси.

\subsection{Тема 2. Вопрос номер 2}
Русь между Ордой и католической Европой.

Главным негативным последствием раздробленности стало ослабление обороноспособности русских земель.
В 30-40х годах 12в. русь одновременно подвергнется нападению с двух сторон.

C запада шла волна католической экспансия. Крестовый поход объявленный римским папой. С целью обращения русских земель в католическую веру. Две волны нашествия были остановлены Александром Невским. Который в 1240 одержал победу в Невской битве. в 1242 году в ледовом побоище. НО!!! Экспансия на этом не остановилась и в 1246 году Папа Римский объявил о начале нового крестового похода на Русские земли. Параллельно с этим русские земли подверглись нападению монголо-татар во главе с Батыем.

Монголо-татары разгромили сожгли ограбили русские земли, но первоначально не обложили их данью а пошли дальше в Западную Европу. Русские княжества после нашествия оказались перед проблемой выбора. Было понятно что одновременно с двумя грозами русские князья не справятся и они должны выбрать кого-то в качестве союзника.

\subsubsection{Две позицие русских князей}
\textbf{Первая позиция}

\begin{enumerate}
    \item Даниил Галицкий(Галицко-Волынский князь, Киевский регент)
    \item Андрей Юрьевич.       
\end{enumerate}

Они выступили за союз с католической европой и следовательно за принятие католического христианства, чтобы образовать обще-европейский союз против монголо-татар. В 1253 году Даниил Галицкий отказался принять ханского наместника и подписывать договор. По этой причине состоялась еще одна волна нашествия. Котораятполучила название Невьюрова рать.
В результате второго нашесвия в Киевском и Галицко волынском княжестве было установлено прямое правление монголо-татар.


\textbf{Вторая позиция}
Александр Невский. Великий Владимирский князь. Он считал что необходимо заключать союз с монголо татарами, так как еще одной волны нашествия русские земли не переживут. И считал монголо татар меньшим злом по сравнению с западной европой. Так как они веротерпимые. И союз позволит сохранить основы русской православной культуры и потом на этой основе начать восстановление государсва. 

Он сам совершает две поездки 1253, 1262 к монголо-татар. В знак добрых намерений отдал в протекторат Новгород.
В 1253 году подписал с ними договор, по которому согласился платить ордынский выход - дань 10\% со всех княжеств. Рекрутов 1 от каждого из 10 дворов. Плюсом к этому он согласился принимать ханский ярлык и тем самым по документам он признал что русские земли попадают в состояние вассальной зависимости от золотой орды. Следствием этого выбора Становится РАСКОЛ русских 
земель.

\textbf{Южные княжества}
\begin{enumerate}
    
\end{enumerate}
Киевское. 
Галицко-Волынское.
Турово-Пинское(Украина и Белорусия)
Отказались подписывать договор с монголо татарами и чтобы им противостоять вошли добровльно в состав великого княжества Литовского.
В 14 в. это княжество подпишет договор с польшей и объедениться в единое государство Речь Посполитая.

Северо-Восточные земли примут вассальную зависимость от Золотой Орды сформируется так называемое иго.

\subsection{Дискуссия о монголо татарском иге}
Все ученые обсуждают 2 вопроса
а) Было ли иго
б) Какое влияние оно оказало на русские земли
\end{document}

Есть три точки зрения

19 В. Н.М. Карамзин.
а) +
Он доказывал что иго было, и предложил классическое определение этого понятия.
Иго, с его точки зрения это система экономической(платили дань) и политической(получали ярлык на княжение) и принимали ханских наместников(баскаков).
б) Последствия ига оценивал положительно так как считал что необходимость его свержения способствовала объединению князей. И более быстрому раннему началу процесса централизации, для которой никаких других предпосылок свержения ига не было.

Вторая точка зрения
Н. Гумилев.
а) Ига, в том значении, которое дает Карамзин на Руси не было. Т.к. термин иго в значении подчинение и зависимость стал использоваться в русском языке только с Петра I. До этого на старославянском. Термин иго переводился как союз или взаимодействие. Следовательно был союз добровльно заключенный Александром Невским для противодействия католической европе.
С этой точки зрения дань это плата за союзную помощь. Рекруты это армия которая охраняла российские границы. Вторая цель этого союза сохранение православия и пока монголо татары были веротерпимыми союз сохранялся. Однако в середине 14 в. государственной религией становится ислам, который должны принять все улусы. Попыткой противостоять принятию ислама стала Куликовская битва 1380 года. И так как русские в ней победили. Им разрешили сохранить православие. Союз сохранился еще на 100 лет.
б) Оценивал положительно, так как православие сохранили и на этой основе создали централизованное российское государство. А те земли, что вошли в состав княжества Литовского, в итоге подверглись катализации.

Третья точка зрения. 
Ключевский, Рыбаков.
а) Монголо татарское иго было. Та же позиция, что выдвинул Карамзин.
б) Влияние отрицательное.
У монголо татар русские князья унаследовали Восточно-деспотичный характер самодержавной власти. и т.д. Смотри учебник.

\section{Тема 3}
\subsection{Этапы централизации русских земель}
1301 - 1389 Борьба между Москвой и Тверью за лидерство в объелинении русских земель.
Базовые даты.
1327 год. Иван Калита, подавив востание в Твери, получил ярлык на великое княжение для московского княжества и право на пожизненый сбор дани со всех русских земель. Часть денег он оставлял себе, на эти деньги он покупал соседние земли.
Территория москвы увеличилась в 27 раз.
1380 год - победа в Куликовской битвы. Главный результат - 1389 год завещание Дмитрия Донского, по которому он оставляет московскую княжество как отчину, не спрашивая ярлыка в орде, и орда этот факт признает.
1389-1462 Период феодальных войн между Московскими князьями. 

Одна позиция - сторонники централизации
Василий 2 темный - сын Д. Донского.

Сторонники раздробленности.
Юрий Звенигородский брат Д.Донского.

Этот период завершился приходом к власти Василия 2 темного.

1462-1533 период завершения объединения русских земель вокруг москвы и начала формирования централизованного аппарата управления. 1584 завершение формирования централизованного аппарата управления, завершение процесса централизации при Иване Грозном.