\documentclass[a4paper]{article}
% Русский язык
\usepackage[utf8]{inputenc}
\usepackage[russian]{babel}
% Математические символы и формулы
\usepackage{mathtools}
% Разбиение на разные файлы
\usepackage{subfiles}
% Отступы с краев(margin)
\usepackage[margin=1in]{geometry}
% Форматирование заголовков
\usepackage{titlesec}
\usepackage{indentfirst} % Красная строка
    
\begin{document}

\section{Введение}

1 семинар - 1 тк (Лекции)
Д/з 1!, 3 вопросы (Проанализировать
политику первых князей по признакам гоударства)
\subsection{Литература}

\begin{enumerate}
\item Методичка 2208
"Отечественная история", Иваново 2014
\item Бумажные из библиотеки
    \begin{enumerate}
    \item Орлов и Георгиев "История России для тех. вузов"
    (Плюсы: вся фактура в одном месте, Минусы: Марксистский формационный подход)
    \item Личман "История России для тех. вузов"
    (Плюсы: есть концепции, Минусы: слабая фактура)
    \item "История России с др. времен до 19 в.", Кафедральный учебник
    (Говно, но нужно брать)
    \item "История России 1917 - 1945", Кафедральный учебник
    (Более приличная часть)
    \end{enumerate}
\item Найти(совет) "История России в схемах и таблцах"
\item Электронные учебники
    \begin{enumerate}
    \item "История России с др. времен до 19 в.", эл. ресурс
    (Посмотреть структуру)
    \item "История России 1917 - 1945", эл. ресурс
    \end{enumerate}
\end{enumerate}

\subsection{1 ТК}

Автор(Персона)
Названиние Концепии
Суть Концепции(Основной вывод по проблеме)
Базовые термины
Даты (к концепциям, принципиально важные)

\section{Тема 1}

\subsection{Основы методологии(как проводить историческое исследование)
 исторической науки}

\subsubsection{Методика(этапы) исторического исследования}

\begin{enumerate}
\item Анализ исторический источников
\item Изучение научных парадигм, концепций, разных точек зрения на эту проблему
\item Анализ первого и второго, если надо ищите дополнительные исторические факты
\item Формулируете собственную аргументированную точку зрения
\end{enumerate}

Исторический источник(созданный в ту эпоху, которую вы изучаете) - это любой памятник прошлого, который содержит исторический факт или информацию об историческом событии.
Группы исторический источников

\begin{enumerate}
\item материальные - археологические раскопки
\item письменные - повесть временных лет
\item изобразительные
\item лингвистические
\item кино, фото, фоно документы (XIX-XX в.)
\end{enumerate}

Научные парадигмы
Парадигма - концептуальная схема, модель, изучение исторических источников и трактовки исторических событий.
Выделяют пять базовых научных парадигм
\begin{enumerate}
\item религиозная(или идеалистическая)
\end{enumerate}
* - базовый фактор исторического развития
\# - модель исторического развития
\^ - историки исслеователи, представители этой парадигмы 
\subsubsection{Религиозная}
Фома Аквинский V в.н.э.
* провиденциализм - все события происходят по божеьей воле.
\# линейная и конечная
\^ В.Н. Татищев - 18 в. (Основатель российской исторической науки),
  Н.М. Карамзин
  С.Н.Соловьев
  В.О. Ключевский
Так как выразителем божественной воли является правитель, то историческое исследование
будет выглядеть как история правления отдельных князей, царей и императоров.
- Недоказуема, так как основная на вере. Субъективна и однобока, так как изучает только один фактор - политику правителя
+ Богатейший фактический материал по политической истории
\end{document}
