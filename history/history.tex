\documentclass[a4paper]{article}
% Русский язык
\usepackage[utf8]{inputenc}
\usepackage[russian]{babel}
% Математические символы и формулы
\usepackage{mathtools}
% Разбиение на разные файлы
\usepackage{subfiles}
% Отступы с краев(margin)
\usepackage[margin=1in]{geometry}
% Форматирование заголовков
\usepackage{titlesec}
\usepackage{ucs}
\usepackage{indentfirst} % Красная строка
\usepackage{tabulary}
\usepackage{float}

\begin{document}

\section{Введение}
1 семинар - 1 тк (Лекции)

Д/з 1!, 3 вопросы (Проанализировать политику первых князей по признакам гоcударства)

\subsection{Литература}
\begin{enumerate}
\item Методичка 2208
"Отечественная история", Иваново 2014
\item Бумажные из библиотеки
    \begin{enumerate}
    \item Орлов и Георгиев "История России для тех. вузов"
    (Плюсы: вся фактура в одном месте, Минусы: Марксистский формационный подход)
    \item Личман "История России для тех. вузов"
    (Плюсы: есть концепции, Минусы: слабая фактура)
    \item "История России с др. времен до 19 в.", Кафедральный учебник
    (Говно, но нужно брать)
    \item "История России 1917 - 1945", Кафедральный учебник
    (Более приличная часть)
    \end{enumerate}
\item Найти(совет) "История России в схемах и таблицах"
\item Электронные учебники
    \begin{enumerate}
    \item "История России с др. времен до 19 в.", эл. ресурс
    (Посмотреть структуру)
    \item "История России 1917 - 1945", эл. ресурс
    \end{enumerate}
\end{enumerate}

\subsection{1 ТК}

Автор(Персона)
Название Концепции
Суть Концепции(Основной вывод по проблеме)
Базовые термины
Даты (к концепциям, принципиально важные)

\section{Основы методологии (как проводить историческое исследование) исторической науки}

\subsection{Методика(этапы) исторического исследования}

\begin{enumerate}
    \item Анализ исторический источников
    \item Изучение научных парадигм, концепций, разных точек зрения на эту проблему
    \item Анализ первого и второго, если надо ищите дополнительные исторические факты
    \item Формулируете собственную аргументированную точку зрения
\end{enumerate}

\subsection{Исторические источники}
Исторический источник(созданный в ту эпоху, которую вы изучаете) - это любой памятник прошлого,
который содержит исторический факт или информацию об историческом событии.
Группы исторический источников

\begin{enumerate}
    \item материальные - археологические раскопки
    \item письменные - повесть временных лет
    \item изобразительные
    \item лингвистические
    \item кино, фото, фонодокументы (XIX-XX в.)
\end{enumerate}

\subsection{Научные парадигмы}
Парадигма - концептуальная схема, модель, изучение исторических источников и трактовки исторических событий.
Выделяют пять базовых научных парадигм

\subsubsection{План}
\begin{enumerate}
    \item Авторы
    \item базовые факторы исторического развития
    \item термины
    \item модель исторического развития
    \item историки исследователи, представители этой парадигмы
    \item +
    \item -
\end{enumerate}

\subsubsection{Религиозная(идеалистическая)}
\begin{enumerate}
    \item Фома Аквинский V в.н.э.
    провиденциализм - все события происходят по божьей воле.
    \item линейная и конечная
    \item
    \begin{enumerate}
        \item В.Н. Татищев - 18 в. (Основатель российской исторической науки),
        \item Н.М. Карамзин
        \item С.Н.Соловьев
        \item В.О. Ключевский
    \end{enumerate}
    \item Так как выразителем божественной воли является правитель, то историческое исследование будет выглядеть как история правления отдельных князей, царей и императоров.
    \item - Недоказуема, так как основная на вере. Субъективна и однобока, так как изучает только один фактор - политику правителя
    \item + Богатейший фактический материал по политической истории
\end{enumerate}

\subsubsection{Формационная}
\begin{enumerate}
    \item Альт. названия:
    ОЭФ - общественно-экономическая формация, \\
    Марксистская, \\
    Материалистическая, \\
    "Советская", \\
    \item
        \begin{enumerate}
        \item Фридрих Энгельс 40е годы 19 в.
        \item К. Маркс
        \end{enumerate}
    \item Социально-экономические социальные факторы,
    \item Классы, классовая борьба(очень сильное сужение предмета исследования) \\
        ОЭФ - ступень этап в развитии общества, который отличается от других этапов своим базисом(экономическим строем), который определяет особенности надстройки
    \item 5 базовых экономических формаций. Схема Ступеньки:
    \begin{enumerate}
        \item Первобытная
        \item Рабовладельческая
        \item Феодальная. Базовый фактор: земля. Ручной труд. Натуральное хозяйство. Базовые классы: феодалы, крепостные крестьяне. \\
        Феодализм в России 1649 г. Начинается с соборного уложения Алексея Михайловича.
        \item Паровой двигатель => Капитализм. Переход от одной ОЭФ к другой осуществляется в результате социальной революции(изменение базовых классов общества). \\ Изменение в средствах производства, новые технологии. Факторы производства: труд и капитал, в результате этого происходит переход к машинному индустриальному производству. Классы: буржуазия, пролетариата.
    \end{enumerate}


    Комментарии к схеме:
    По Марксу переход к коммунизму возможен при двух условиях. \\
    Два пути перехода: Кризис капитализма, или абсолютное большинство класса пролетариат.((Социализм)). Коммунизм. Отсутствие классов, всеобщее равенство. Формирование общественной собственности на средства производства. Нет необходимости в существовании государства. Базовый принцип: от каждого по возможности каждому по потребности.
    \begin{enumerate}
    \item Если капитализм себя изживет, придет к кризису, появятся новые технологии
    \item Если пролетариат составит 70 процентов населения страны, тогда он имеет право на революцию и установление диктатуры пролетариата.
    \end{enumerate}

    В. И. Ленин дополнил это учение учением о социализме, как особой переходной стадии от капитализма к коммунизму, цель которой увеличение численности пролетариата, завершение индустриализации. Т.к. в октябре 1917 года пролетариата в России 9\%, и следовательно объективных предпосылок для социалистической революции в россии не было.

    Основатели советской исторической науки
    Б.А. Рыбаков \\
    "Экономика определяет все" К. Маркс \\
    В СССР формационная концепция стала доминирующей и единственно возможной и на этой модели основывалась вся советская историческая наука
    \item
    Идеально подходит для анализа стран Западной Европы
    Попыталась создать универсальную модель, которая анализировала бы закономерности исторического процесса
    Продуктивно изучать общество на основе экономики
    \item
    Не подходит для стран востока и России, следовательно не является универсальной.
    Практически не изучали влияние других факторов на исторический процесс
    Т.к. эта модель стала доминирующей в СССР, произошли фальсификации, искажения исторического процесса, так как история России в эту модель не вписывается.
\end{enumerate}
\subsubsection{Этногенетическая}
\begin{enumerate}
\item Л. Н. Гумилев. 1930е годы. Существуюет с 90х годов
\item Динамика развития этноса, этногенез. Базовые термины:
    Этнос - это большая группа людей, которая отличается от других групп, четко обозначенной и осознаваемой целью своего существования. \\
    Пассионарность - это особый вид интеллектуальной энергии. Способность отдельных личностей формулировать цели развития этноса и организовывать большие массы людей для достижения этих целей.
\item Схема:
    \begin{enumerate}
        \item Появление идеи
        \item Пассионарный толчок
        \item Формулирование идеи
        \item Появление этноса
        \item Развитие этноса
        \item !TODO FIX THIS
    \end{enumerate}
    Субпассионарии
\end{enumerate}

Дома к семинару сформулировать плюсы и минусы данной концепции.

\subsubsection{Цивилизационная}
\begin{enumerate}
    \item О. Шпенглер, А. Тойнби, Н. Я. Данилевский
    \item социо-культурные ценности идеи, идеологии культуры
    \item Нет общего единого универсального определения понятия цивилизации. Есть только 3 критерия, которые всем понятны
    Высокий уровень развития общества
    Каждая имеет свой уникальный путь исторического развития
    Одна цивилизация принципиально отличается от другой базовой системой идей ценностей.
    \item Кружки с цивиллизациями греч восточно-европейская
    \item Вообще не изучают общество находящее на ранних стадиях развития
    Так как нету единого определения цивилизации, следовательно нет общих критериев для их сравнения. Следовательно модель вообще не изучает закономерности исторического процесса, следовательно не может являться универсальной.
    Изучать отдельно историю какой-то страны невозможно потому что н а нее постоянно влияют какие-то бесконечные внешние факторы.
    \item Сформулировать свои размышления
\end{enumerate}

\subsubsection{Новая социальная история. История повседневности. Французская школа анналов}
\begin{enumerate}
    \item М. Блок и Л. Февр
    \item быт, нравы, духовная атмосфера эпохи
    \item Историю нельзя загонять в заданные схемы развития. -
    \item Изучая изменение в повседневной жизни. Анализируют влияние на нее различных факторов социально-экономических и духовных. Следовательно это попытка провести комплексное исследование исторического процесса.
%комплексный подход к изучению истории. Влияние человека на исторический процесс. Совокупность факторов.
    \item Очень большой объем материала, информации. Нету четкой методики, часто не доводятся до конца.
\end{enumerate}

\subsection{Третий шаг исторического исследования}
Сравнение источников и выводов научных школ и парадигм на основе принципа историзма. Метод анализа причинно следственных связей, который анализирует количественные и качественные изменения в каждом этапе и дает оценку причин и последствий этих изменений.

\section{План подготовки к семинару}
\begin{enumerate}
\item Политическая раздробленость на Руси.
\item Причины раздробленности (Логика от концепций)
    Социально-экономические Причины.
    Духовные Причины.
    Политические причины внутренние/внешние.
\item Сущность раздробленности(Определение, хранические рамки, сравнение с аналогичным процессом в других странах)
\item Последствия раздробленности
\item Точки зрения на раздробленность. +-. Собственное мнение.
\end{enumerate}

\section{Тема 2. Проблема происхождения государства у восточных славян}
Это одна из самых горячо обсуждаемых проблем российской истории, так как отсутсвуют прямые исторические источники по этому периоду.
Единственный источник это повесть временных лет, это источник XII века. А повествует о событиях IX века. И в этом источнике приводится легенда о призвании варягов.
Славяне по причине междоусобиц в 862 году пригласили варягов - "Русь"
Рюрик начал княжить в Новгороде. Трувор в Изборске. Синеус на Белоозере.
А Оскольд и Дир пошли на греки и начали княжить в Киеве

Все ученые которые обсуждают эту проблему обсуждают следующие вопросы:
\begin{enumerate}
    \item Было ли призвание варягов и кто они такие?
    \item Как трактовать термин "Русь"?
    \item Кто, когда и почему образовал государство у Восточных славян?
\end{enumerate}

\subsection{Норманская концепция}
Авторы этой концепции: Байер, Миллер, Шлецер. Немецкие ученые 18 века
\begin{enumerate}
    \item Да, естественно было и ПВЛ об этом однозначно свидетельствует
    Варяги это нормано-скандинавы, которые проживали по северному побережью Балтийского моря
    \item Русь это самоназвание того племени варягов, которое пришло.
    \item Государство образовали 862 г. приглашенные норманы во главе с Рюриком, т.к. у них уже на тот момент уже существовало государство и они принели этот опыт с собой чтобы прекратить междоусобицы.
\end{enumerate}
Выводы этой концепции сразу приобрели политический характер и стали использоваться врагами россии для обоснования неполноценности славянской нации(сами даже не смогли организовать государсво).

\subsection{Антинорманская концепция}
Дореволюционная Норманская концепция. М. В. Ломоносов.
Реакция на норманскую концепцию.
\begin{enumerate}
    \item + Призвание варягов было. Варяги происходит от славянского Вар. что означает море. Этим термином обозначали поморских славян. Которые проживали по южному побережью Балтийского моря. И которых поморские славяне, которых восточные славяне пригласили для оказания военной помощи.
    \item Термин Русь - видимо название племени поморских славян. Нет объяснения.
    \item Государство образовал в 882 году князь Олег. Так как во время похода из новгорода в Киев, он завоевал, объединил все славянские племена проживавщие по днепру и сформировал общую территорию и перенес столицу в Киев.
\end{enumerate}
Ломоносов упустил в своей концепции очень много проблем и в 30х годах XX в. немецкие историки используют эти пробелы и будут доказывать несостоятельность антинорманской концепции.
Проблемы этой концепции:
\begin{enumerate}
    \item Против кого пригласили оказывать военную помощь в Новгород, если главный враг (Хазарский каганат) находится на юге?
    \item Если варяги это поморские славяне, то почему данные археологических раскопок курганов первых киевских князей показывают доминирование норманской, скандинавской культуры?
    \item А кто такие Оскольд и Дир. Какова их роль в образовании государства?
\end{enumerate}

\subsection{Советская антинорманская концепция. Б.А. Рыбаков.}
\begin{enumerate}
    \item Призвание варягов было. Варяги это нормано-скандинавы. О чем свидетельствуют ПВЛ и археологические источники. Норманов пригласило племя полян, которые жили в районе Киева, для оказания военной помощи против Хазарского каганата. Но увидев междуусобицы варяги этим воспользовались и завоевали незащищенные города по побережью Балтийского моря и начали в них княжить. А часть дружины отправили в Киев против Хазар.
    \item Термин Русь происходит от названия притока днепра реки Рось.
    По которой проживали поляне и варяги примут на себя это название и под ним потом войдут во все русские источники.
    \item Легенда о призвании Варягов вообще не имеет никакого отношения к процессу образования государства. Так как ни одна личность не может влиять на исторический процесс, если для этого не сложились необходимые исторические предпосылки. К моменту призвания варягов у славян уже были все необходимые предпосылки для образования государства.
\end{enumerate}

\subsubsection{Комплексная концепция}
Исаев. Юрганов. Кацва.
\begin{enumerate}
    \item однозначно нормано-скандинавы. И ПВЛ и арх. источники и лингвистический анализ. Они проводят лингвистический анализ имен первых киевских князей и приходя к следующим выводам что все имена имеют нормано скандинавское происхождение. Они делают следующие выводы:
    Они признают историчность Рюрика, что он существовал, и считают что главный его вклад в основании первой княжеской династии на Руси и все последующие князья будут Рюриковичи. Они считают что Синеус и Трувор - вымышленные персонажи. неправильный перевод ``меч и дружина''.
    \item
    Термин Русь происходит от скандинавского ``ротс''(Гребцы) и этим термином славяне всегда называли представителей норманских племен.
    \item Вывод, к 862 году у славян уже существовало большинство признаков государства. А завершат процесс образования государства первые киевские князья. И окончательно оно сформируются только к концу правления святослава.
\end{enumerate}

\subsection{Признаки государсва}
\begin{enumerate}
    \item + Общность территории
    \item + Публичная власть: Аппарат управления, Аппарат принуждения.
    \item + Налоги
    \item +-? Формально определенная система права.
    \item -+ Суверенитет. Внутренний/Внешний.
\end{enumerate}

К началу XIII века на территории по Днепру уже существовало три крупных конфедеративных союза восточных славян.
\begin{enumerate}
    \item Славия со столицей в Новгороде.
    \item Иордания со столицей в Смоленске.
    \item Куявия со столицей в Киеве
\end{enumerate}

\subsubsection{Территория}
Восточные славяне объединялись так как через их территорию проходил торговый путь из варяг в греки. Единственный в то время путь, который связывал Западную Европу и Азию. И Славяне объединялись для защиты этого торгового пути от нападения чтобы получать таможенные пошлины.

\subsubsection{Аппарат управления}
На рубеже VII-VIII веков у Славян происходит распад родовой общины. И формируется общественный строй который именуется ``Военная демократия'', это система управления когда верховная власть принадлежит народному собранию вече. В работе которого могут участвовать только мужчины, которые защищают племя. В промежутках между созывами речи власть переходит к совету старейшин.

\subsubsection{Аппарат принуждения}
Князь, НО Он не управляет, это только военный руководитель и плюс по делегированию вече он начинает выполнять судебные функции.

Налоги собирались с середины VIII века в форме полюдья. Его главная особенность в том, что размер не был четко фиксирован, четко определен и не князь решал сколько взять, а община сколько дать.

Первым из дошедших до нас русским законом является русская правда принятая при Ярославе Мудром. (только в 30х годах XI в.) Но в первых международных договорах Руси и Византии 911 и 941 гг. есть ссылки и приводятся статьи из закона Росикова. Видимо в XI в. существовал свод законов, который у нас не сохранился.

\subsubsection{Суверинитет}
Внутренний суверенитет - это способность государственной власти принимать решения и проводить самостоятельную политику. Не существовал.
Внещний суверенитет - признание другими государствами в качестве самостоятельного независимого партнера.
Существовал так как племена объединялись для защиты своей территории и торгового пути. Существовала какая-то ситсема торговых отношений с другими странами.

Дома сформулировать собственную позицую с 2 аргументамя по этой проблеме.
Древнерусское государство. Проблема социально экономического развития киевской руси.
Кивеская Русь это древнее государство, которе существовало с XI в. до 30х годов XII в. И которое стремилось к проведению единой, централизованной политики. Самая Дискусионная проблема - сложился ли в Киевской Руси феодализм. А если нет, то какой тогда существовал социально-экономический строй.
Феодализм - это экономический строй(ОЭФ), при которой владелец земли феодал эксплуатирует экономически зависимых крепостных крестьян.
Признаки:
\begin{enumerate}
    \item Наличие класса владельцев земли феодалов
    \item Наличие класса крестьян, которые арендуют часть земли у феодала и за пользование этой землей платят феодальную ренту(барщина и оброк).
    \item Наличие системы крепостного права. Это законодательное прикрепление крестьян к земле. По которому им запрещается переход от одного феодала к другому.
\end{enumerate}

\section{Тема 2. Сущности раздробленности}
Дискуссии о сущности раздробленности:
Политическая или феодальная.


\subsection{Тема номер 2. Вопрос 1. Сущность раздробленности}
Ищи три модели политического развития русских княжеств.
В этот период система управления во всех княжествах представляла собой взаимодействие трех элементов: князь, вече и боярская дума. Которые боролись за власть друг с другом и в зависимости от доминирования того или иного элемента сформировались три модели политического развития.

\subsubsection{Владимиро-Суздальское княжество}
Формируются предпосылки для перехода к самодержавной монархии. Главной в этой системе становится князь, а два других ему подчинялись.

Найти как формировалась сильная княжеская власть. В рамках трех правлений.
\begin{enumerate}
\item Ю. Долгоруков.
\item А. Боголюбского.
\item В. Большое гнездо
\end{enumerate}

\subsubsection{Новгородское княжество}

\subsubsection{Галицко-Волынское княжество.}
Олигархическая ограниченая монархия.

Вопрос на ПК1: Причины сущности последствия раздробленности на Руси.

\subsection{Тема 2. Вопрос номер 2}
Русь между Ордой и католической Европой.

Главным негативным последствием раздробленности стало ослабление обороноспособности русских земель.
В 30-40х годах 12в. русь одновременно подвергнется нападению с двух сторон.

C запада шла волна католической экспансия. Крестовый поход объявленный римским папой. С целью обращения русских земель в католическую веру. Две волны нашествия были остановлены Александром Невским. Который в 1240 одержал победу в Невской битве. в 1242 году в ледовом побоище. НО!!! Экспансия на этом не остановилась и в 1246 году Папа Римский объявил о начале нового крестового похода на Русские земли. Параллельно с этим русские земли подверглись нападению монголо-татар во главе с Батыем.

Монголо-татары разгромили сожгли ограбили русские земли, но первоначально не обложили их данью а пошли дальше в Западную Европу. Русские княжества после нашествия оказались перед проблемой выбора. Было понятно что одновременно с двумя грозами русские князья не справятся и они должны выбрать кого-то в качестве союзника.

\subsubsection{Две позицие русских князей}


\textbf{Первая позиция}
\begin{enumerate}
    \item Даниил Галицкий(Галицко-Волынский князь, Киевский регент)
    \item Андрей Юрьевич.
\end{enumerate}
Они выступили за союз с католической европой и следовательно за принятие католического христианства, чтобы образовать обще-европейский союз против монголо-татар. В 1253 году Даниил Галицкий отказался принять ханского наместника и подписывать договор. По этой причине состоялась еще одна волна нашествия. Котораятполучила название Невьюрова рать.
В результате второго нашесвия в Киевском и Галицко волынском княжестве было установлено прямое правление монголо-татар.


\textbf{Вторая позиция}
Александр Невский. Великий Владимирский князь. Он считал что необходимо заключать союз с монголо татарами, так как еще одной волны нашествия русские земли не переживут. И считал монголо татар меньшим злом по сравнению с западной европой. Так как они веротерпимые. И союз позволит сохранить основы русской православной культуры и потом на этой основе начать восстановление государсва.

Он сам совершает две поездки 1253, 1262 к монголо-татар. В знак добрых намерений отдал в протекторат Новгород.
В 1253 году подписал с ними договор, по которому согласился платить ордынский выход - дань 10\% со всех княжеств. Рекрутов 1 от каждого из 10 дворов. Плюсом к этому он согласился принимать ханский ярлык и тем самым по документам он признал что русские земли попадают в состояние вассальной зависимости от золотой орды. Следствием этого выбора Становится РАСКОЛ русских земель.

\textbf{Южные княжества}
\begin{enumerate}
    \item Киевское
    \item Галицко-Волынское
    \item Турово-Пинское(Украина и Белорусия)
\end{enumerate}

Отказались подписывать договор с монголо татарами и чтобы им противостоять вошли добровльно в состав великого княжества Литовского.
В 14 в. это княжество подпишет договор с польшей и объедениться в единое государство Речь Посполитая.

Северо-Восточные земли примут вассальную зависимость от Золотой Орды сформируется так называемое иго.

\subsection{Дискуссия о монголо татарском иге}
Все ученые обсуждают два вопроса:
\begin{enumerate}
    \item Было ли иго
    \item Какое влияние оно оказало на русские земли
\end{enumerate}

Есть три точки зрения

\subsubsection{XIX В. Н.М. Карамзин}
\begin{enumerate}
    \item +
    Он доказывал что иго было, и предложил классическое определение этого понятия.
    Иго, с его точки зрения это система экономической(платили дань) и политической(получали ярлык на княжение) и принимали ханских наместников(баскаков).
    \item Последствия ига оценивал положительно так как считал что необходимость его свержения способствовала объединению князей. И более быстрому раннему началу процесса централизации, для которой никаких других предпосылок свержения ига не было.
\end{enumerate}

\subsubsection{Н. Гумилев}
\begin{enumerate}
    \item Ига, в том значении, которое дает Карамзин на Руси не было. Т.к. термин иго в значении подчинение и зависимость стал использоваться в русском языке только с Петра I. До этого на старославянском. Термин иго переводился как союз или взаимодействие. Следовательно был союз добровльно заключенный Александром Невским для противодействия католической европе.
    С этой точки зрения дань это плата за союзную помощь. Рекруты это армия которая охраняла российские границы. Вторая цель этого союза сохранение православия и пока монголо татары были веротерпимыми союз сохранялся. Однако в середине 14 в. государственной религией становится ислам, который должны принять все улусы. Попыткой противостоять принятию ислама стала Куликовская битва 1380 года. И так как русские в ней победили. Им разрешили сохранить православие. Союз сохранился еще на 100 лет.
    \item Оценивал положительно, так как православие сохранили и на этой основе создали централизованное российское государство. А те земли, что вошли в состав княжества Литовского, в итоге подверглись катализации.
\end{enumerate}

\subsubsection{Ключевский, Рыбаков}
\begin{enumerate}
    \item Монголо татарское иго было. Та же позиция, что выдвинул Карамзин.
    \item Отрицательное. У монголо татар русские князья унаследовали восточно-деспотичный характер самодержавной власти и т.д. Смотри учебник.
\end{enumerate}

\section{Тема 3}
\subsection{Этапы централизации русских земель}
\subsubsection{1301 - 1389 Борьба между Москвой и Тверью за лидерство в объелинении русских земель.}
\textbf{Базовые даты.}
1327 год - Иван Калита, подавив востание в Твери, получил ярлык на великое княжение для московского княжества и право на пожизненый сбор дани со всех русских земель. Часть денег он оставлял себе, на эти деньги он покупал соседние земли.
Территория москвы увеличилась в 27 раз.

1380 год - победа в Куликовской битвы. Главный результат - 1389 год завещание Дмитрия Донского, по которому он оставляет московскую княжество как отчину, не спрашивая ярлыка в орде, и орда этот факт признает.

\subsubsection{1389-1462 Период феодальных войн между Московскими князьями. }
Среди князей было две позиции.
Одна позиция - сторонники централизации, Василий II Темный - сын Д. Донского.
Вторая позиция - сторонники раздробленности,
Юрий Звенигородский, брат Д.Донского.
Этот период завершился приходом к власти Василия II Темного.

\subsubsection{ 1462-1533 период завершения объединения русских земель вокруг москвы и начала формирования централизованного аппарата управления.}
1584 год - завершение формирования централизованного аппарата управления, завершение процесса централизации при Иване Грозном.


\section{Россия в XVII веке}

\subsection{Смутное время в России. Дискуссии историков}
Главная дискуссия по поводу причин, сущности. Последствия, этапы

Две точки зрения

\subsubsection{Советская: Смирнов, Федоров и Буганов}
Смута - это высшая форма борьбы угнетаемых крепостных крестьян против эксплуататорских классов.
Причины:
Экономические: усиление крепостного права, как следствие обострение социальных противоречий и как следствие начало народных бунтов и крестьянских восстаний.
Высшая точка или пик смутного времени: "крестьянская война" под руководством Болотникова. 1606-1607 годов. Максимальное вовлечение крестьян.
\subsubsection{Современная: Кобрин и Скрынников}
Смута - первая гражданская война так как в событиях этого периода участвовали все социальные слои выступая друг против друга. 1598 - 1613
Причины:
    \textbf{Системный кризис}. Кризис который одновременно охватил все сферы общественной жизни. Он стал следствием поражения России в Ливонской войне(1558-1582) и разорением южных регионов+Москвы. 1571-1572 Девлет-Герей. Следствием всех этих событий стало сокращение на 70\% посевных площадей в северо-западных и южных регионах страны. Следствием этого становится массовое бегство крестьян на восток, голод 1601-1603.

    \textbf{Социальный кризис}. Все слои общества были недовольны своим положением.
            \begin{enumerate}
                \item Следствием массового бегства крестьян на восток стало принятие двух указов 1581г указ ``о заповедных летах'', по которому временно запрещали крестьянские переходы в Юрьев день. Фактически этот закон означал введение крепостного права. Указ 1597 года ``об урочных летах'', срок от 5 до 10 лет поимки беглых крестьян. Следствием принятия этих двух указов становится массовое недовольствие всех крестьян.

                \item Бояре в период опричнины потеряли значительную часть своих привилегий и по военной реформе Иван Грозный обязательно заставил их нести военную службу за землю. Следовательно они хотят вернуть свой привиллегированный статус в обществе.

                \item Дворяне, с одной стороны выдвинулись на высшие государственные посты, но с другой стороны их статус никак не был закреплен. Они требуют уравнивание поместья с вотчиной.

                \item Городское население. В ходе реформы 1549 года Иван Грозный ввел в России земские соборы - выборный законо совещательный орган центральной власти, в работе которого участвовали представители всех сословий, кроме крепостных крестьян. Городское население впревые получило возможность влиять на политику государства. Однако в период опричнины созыв земских соборов и выборный принцип формирования органов местного самоуправления был отменен. Следовательно, городское население выступает за возращение себе все этих прав и полномочий

                \item Церковь. В рамках стоглавого собора 1551 г Иван Грозный запретил рост церковного землевладения.
            \end{enumerate}

        3) Политический кризис.
            \begin{enumerate}
                \item Внутри политический был связан с династическим кризисом власти. Федор Иванович (1584-1598) - умолишенный, регентом был Борис Годунов. Умер не оставив наследников. Земский собор 1598 года выбрал царем Бориса Годунова. Однако население его в качестве законного правителя не признало.
                \item Внешнеполитический кризис следствием неудачной политики Ивана Грозного стало то что три страны Польша, Швеция и Крымское ханство имели территориальные претензии к России и они попробуют решить эти претензии в период смуты.
            \end{enumerate}

\subsection{Основные этапы смуты}
\subsubsection{Династический кризис. 1598-1606 гг.}
Население в этот период было готово признать на российском престоле любых самозванцев, но не представителей законной власти.
Закончился тем, что в 1605 году Москву захватил Лжедмитрий I. (1605-1606 гг.)
Попытался провести реформы направленные на модернизацию России по европейскому образцу, настроил против себя все общество.
\subsubsection{Социальный кризис. 1606-1610 гг.}
Так как в этот период в события смуты оказались одновременно вовлечены все социальные слои и все регионы страны. События разворачиваются по 4 параллельным процессом.
1606-1610. Новым русским царем бы избран Василий Шуйский. Во время выборов он подписал закон с боярской думой. С подписанием этого соглашения в России сложилась ограниченная олигархическая монархия.
Второй параллельный процесс - "Крестьянская война" - Крестьянские выступления по руководством Болотникова и предпринял поход на Москву и был разгромлен под Коломной.
Третий процесс - появление Лжедмитрия II. 1610 г. дошел до г. Тушино но напасть на Москву так и не решился.
Четвертый процесс - начало открытой польско-шведской интервенции. 1609-1610 гг.
Свержение Василия Шуйского
\subsubsection{Национальный кризис. 1610-1613 гг.}
В этот период была реальная угроза потери суверенитета и прекращение существования России.

Власть переходит к боярской думе. Семибоярщина. Они подписывают соглашение с Польшой, приглашают на престол польского царевича Владислава.
Параллельно с этим идет шведская интервениция. Они захватили Новгород, все северо-западные земли и к 1612 г. дошла до смоленска.
Попытка создания первого народного ополчения под руководством Прокопий Липунов потерпело поражение.
1612г было создано второе народное ополчение под руководством Минина и Пожарского, которые четвертого ноября 1612 года выйграли битву под москвой и остановили польскую интервенцию.

\subsection{Последствия смуты}

\subsubsection{Внутриполитические}

В 1613 году земский собор выбирает на царствование нового царя - Михаила Романова. Начинается сращивание государственной и церковной власти. Усиливаются самодержавно-деспотические тенденции. При Алексее Михайловиче(1645-1676) в России начинат формироваться предпосылки абсолютной монархии.

\subsubsection{Внешнеполитические}
 Россия потеряла огромные территории все Северо-Западные земли до Смоленска отошли к польше и все северные земли и выход к балтийскому морю отошли к швеции.

 \subsubsection{Экономические}
 Окончательное закрепощение крестьян и в 1649 г. принимается соборное уложение, которое окончательно отменяет Юрьев день и вводит бессрочный срок поимки беглых крестьян.

 \subsubsection{Социальные последствия}
 Возвышение позиций дворянства. Будут приниматься указы которые будут уравнивать статус поместья и вотчины. К концу XVII поместье=вотчина.
 Закрепощение крестьян.

 \subsubsection{Духовные}
 Церковь в ходе всех этих процессов окончательно подчиняется государству. Фактически происходит сращивание государственных аппаратов.
 Происходит падение авторитета царской и любой власти вообще. Остаток XVII в называют "Бунташный век".

... Проебал начало надо спиздить у кого нибудь

\section{Тема ....}
Главное следствие - проблема выбора каким путем дальнейшего развития идти, так как в России только окончательно оформляется феодализм и крепостное право, а в западной европе параллельно с этим идут процессы развития капитализма и начало первых буржуазных революций.
Следовательно в россии оформляется система крепостного права

С одной стороны оформился феодализм Под влиянием новых заподно европейских технических открытий и западно европейских буржуазных революций.

Капитализм - это экономический строй при котором владельцы средств производства(фабрики заводы и оборудование) буржуазия эксплуатирует лично свободных наемных рабочих, которые живут за счет продажи своего труда

\textbf{Признаки капитализма}
\begin{enumerate}
\item Товарное хозяйство
\item Формирование рыночной экономики
\item Наличие капитала, который позволяет развивать переходить к фабрично машинному производству.
\item Свободные наемные руки
\item Конкуренция товаропроизводителей, которая способствует развитию предпринимательства а сдругой стороны стимулирует развитие рыночной экономики.
\end{enumerate}
\textbf{TODO починить таблица не влезает на лист}
\subsection{Особенности социально экономического строя в XVII веке}
\begin{table}
    \centering
    \begin{tabular}{|c|c|c|}
    \hline
    Новые капиталистические явления & Феодальные черты \\ \hline
    Сельское хозяйство & Хозяйственная специализация районов (Например поволжье специалищируется на хлебе, а центральные регионы на животноводстве) & Оформляется крепостное право, в следствие чего осуществляется переход к экстенсивному способу ведения хозяйства. Феодалы начинают получать основной доход только за счет феодальной ренты. Путь усиления эксплуатации.
    Самая главная проблема - в соборном уложении не был ограничен размер феодальной ренты. Следствием роста барщины становится череда экономических кризисов. \\
    Промышленность & Начинается промышленная специализация районов(Тула оружие). Появляются первые мануфактуры - первый тип промышленного предприятия с разделением труда, с примитивными первыми механизмами и которая производит продукцию для продажи на рынке. Считается \textbf{первым типом} экономического предприятия. & Доминирует ремесленное производство - это тип предприятия без разделения труда, без механизмов, которое производит продукцию на заказ а не для продажи на рынке. Не развивается товарное хозяйство. Но в России мануфактуры не стали капиталистическими предприятиями так как: Все мануфактуры были государственными или работали на госзаказ. Не способствовали развитию конкуренции и торговли. Базировались на феодальном способе производства. Нет свободных наемных рук. Бесплатный труд крепостных крестьян \\ \hline
    Торговля  & Начинает формироваться всероссийский рынок, уникальная экономическая система, в рамках которой два раза в год, через систему всероссийских ярмарок осуществляется обмен товарами произведенной в ходе специализации районов & Доминирует натуральное хозяйство, товарное хозяйство и торговля не развивается \\\hline
    Внешняя торговля & Россия перешла к политике протекционизма & В ходе смуты россия потеряла выходы ко всем морям, следовательно всю внешнюю торговлю, контролировали иностранные купцы 1653. Россия доходов не получала.\\
    \hline
    \end{tabular}
\end{table}

Таким образом проявления капитализма были крайне незначительными, доминировал феодализм. Противоречия между феодальным и капитализмом станет базовым противоречием Российского общества вплоть до октября 1917 года.

\begin{table}[H]
    \centering
    \begin{tabular}{|c|l|}
    \hline
        Либеральные &
        \begin{minipage}{4in}
            Те правители которые способствуют развитию капитализма
        \end{minipage}\\
    \hline
        Консервативные &
        \begin{minipage}{4in}
            Те кто закреплять или усиливать феодализм и ликвидировать элементы капиталистического строя.
        \end{minipage}\\
    \hline
    \end{tabular}
\end{table}
Проанализировать политики Екатерины II, Павла I, Александра I, Николай I по плану.

\begin{enumerate}
    \item Политика в крестьянском вопросе. Общее направление экономической политики
    \item Политика в дворянском вопросе. Дворянские права и привилегии.
    \item Глобальные изменения. Проекты глобальных изменений в системе гос управления
    \item Гиперуспехи или гипернеудачи во внешней политике.
    \item Диаметрально противоположные точки зрения по оценке этих правления с аргументами
    \item Сформулировать собственное мнение с четырмя аргументами.
\end{enumerate}

\section{Проблемы социально экономического и политического развития россии в XVIII-XIX веках}

\subsection{Оценка правления Петра I(1682/1696-1725)}

Сложилась в XIX в. в рамках полемики между западниками и славянофилами. Соотвественно получается что западнки оценивали правление Перта I исключительно положительно. Их главный вывод - реформы Петра I способствовали модернизации России по западному образцу.

Славянофилы оценвиали правление исключительно отрицательно и приводили три основных аргумента:
\begin{enumerate}
    \item Внедрил западно-европейскую культуру. Что стало началом кризиса традиционной русской православной культуры.
    \item Ликвидировав земские соборы разорвал традиционную связь царя и народа.
    \item Создав бюрократический аппарат управления, разрушил традиционно обищинную систему управления.
\end{enumerate}

\subsubsection{Советская точка зрения}
В. Сталин.
Краткий курс истории 1936 года.
\begin{enumerate}
    \item ``Петр первый великий реформатор''
    \item ``Прорубил окно в европу''
\end{enumerate}
Минус - насильственные методы проведения реформ

\subsubsection{Современная точка зрения}
Нет однозначной оценки петра первого. Принято выделять плюсы и минусы в его правлении.

\subsection{Провел или нет модернизацию россии по западному образцу???}
\subsubsection{План}
\begin{enumerate}
    \item Западная Европа
    \item Плюсы
    \item Минусы
\end{enumerate}

\subsection{Экономика}
\subsubsection{Западная Европа}
Начало промышленного переворота, переход от мануфактурного к фабрично-заводскому производству, следовательно капитализм
\subsubsection{Плюсы}
Для обеспечения армии и флота Петр I стимулировал развитие российских мануфактур и тем самым он заложил основу для развития легкой текстильной и тяжелой металлургичесокй промышленности
\subsubsection{Минусы}
Большинство мануфактур государственные и работают на гос заказ, развиваются на гос. средства и следовательно частный капитал не развивается. Для работы на мануфактурах Петр I вводит две категории крестьян. 1703 г. Приписные крестьяне - три дня в неделю бесплатной работы на мануфактуре. Низкая производительность труда, низкое качество работы. Рабочий класс не формируется. в 1721 году - Посессионные крестьяне - собственность мануфактур - три дня работил в счет уплаты гос. налога. три дня в счет мануфактры и только один день на себя. Не отменил крепостное право, а наоборот усилил процесс закрепощения крестьян.

\subsection{Социальная}
\subsubsection{Западная Европа}
Идет политика ликвидации сословий и принятия законов которые способствовали широкой социальной мобильности населения
\subsubsection{Плюсы}
+сословного строя, сословной политики. Петр I мечтал создать государство всеобщего блага, в котором все сословия служат государству и все платят какой-то вид налога. 1714 год. Указ о единонаследии, по которому поместье уравнивается с вотчиной. Может передаваться по наследств только старшему сыну, а все остальные дворянские дети должны нести пожизненную службу государству. 1703 г. Ликвидировался институт холопства. Все крестьяне были либо государственными - Лично свободные крестьяне, которые платят только гос налог, либо крепостными
\subsubsection{Минусы}
Только только окончательно оформляется сословный строй и четко определяются права и обязанности каждого сословия. Оформление сословного строя.

\subsection{Политическая}
\subsubsection{Западная Европа}
\begin{enumerate}
    \item Период начала буржуазных революций => ограничение самодержавия(Парламентская монархия в Англии) или переход к буржуазным демократическим республикам(Нидерланды).
    \item Уравнивание всех перед законом
\end{enumerate}
\subsubsection{Плюсы}
\begin{enumerate}
    \item Создал эффективную систему центрального и местного самоуправления, которая с незначительными изменениями до 1905 г.
    \item Ввел в стране систему контрольно-надзорных и репрессивных органов. Фискальная служба и генеральная прокуратура.
\end{enumerate}
\subsubsection{Минусы}
\begin{enumerate}
    \item В россии сложился военно-полицейский тип абсолютной монархии, которая характеризовалась установить абсолютный контроль над всеми сферами жизни общества и государства. Многочисленные законы детально регламентировали жизнь граждан
\end{enumerate}

\subsection{Духовная}
\subsubsection{Западная Европа}
Отделение государсва от церкви, свобода совести, свобода вероисповедания. Как следствие переход к светской культуре. Формирование научно рационалистической картины мира, развитие образования,
науки и техники. Доминирующими становятся идеи антропоцентризма
\subsubsection{Плюсы}
Способствуют проникновению в России научно технических открытий, вводит европейскую систему образования для дворянства, способствует распространению Европейской бытовой культуры среди дворян, в результате всего это начинается преодоление религиозно-патриархальных устоев общества, закладываются основы светской культуры
\subsubsection{Минусы}
1) ----, 2) 3) Только для дворянтсва. В результате этих реформ произошел разрыв между дворянской культурой и культурой всего остального общества, что заложило основу для культурного кризиса. 4) Все культурные преобразования носили внешний характер и не затронули базовых ценностей российской культуры. Идея антропоцентризма вообще не получила никакого распространения.

Реформы Петра I. не способствовали модернизации России по западному образцу, так как меняла только отдельные элементы системы не меняя феодальную структуру общества в целом, наоборот усиливали феодализм.

\section{Эпоха дворцовых переворотов}
Период в истории России 1725-1762 г. Который характеризовался борьбой дворянских группировок за власть.

\begin{enumerate}
    \item "Новые дворяне" - дворяне которые выделились при Петре I, благодаря табелю о рангах. Не знатные, не потомственные хотели сохранить свое положение и выступали за продолжение Петровских реформ.
    Группы: Меньшиковы, Бестужие, Лопухины, Орловы, Толстые
    \item "Старые дворяне" - потомственное родовитое дворянство, имеющие земли. Которое в период правления Петра I утратило свои позиции. Было отстранено от управления государством. Выступало за отмену табеля о рангах. Ликвидацию нового дворянства, свертование петровских реформ, возврат собственных привиллегированный. Голицины, Долгорукие, Минихи, Остерманы
\end{enumerate}

\subsection{Дворцовыe переворотов}
Формальным поводом к череде дворцовых переворотов был указ о престолонаследии 1724 года, что царь сам может выбирать любого наследника.

\subsubsection{Первый дворцовый переворот}
Совершили новые дворяне во главе с Меньшиковым, посадив на престол его жену Екатерину I 1725-1727. При ней был Верховный тайный совет, во главе с меньшиковым, который стал органом реальной власти.
Екатерина I продолжала политику Петра I. Умирая она оставила престол внуку Петра I - Петру II 1727-1730.

\subsubsection{Второй дворцовый переворот}
Второй дворцовый переворот произойдет через полгода вошествия на престол.
Совершат его князья Голицины и Долгорукие, они отправят меньшикова в сибирь, сами возглавят Верховный тайный совет и начнут политику по свертыванию Петровских реформ. Столица перенесена обратно в Москву. Прекратится строительство и финансирование флота. Будет ликвидирован сенат.
После внезапной смерти Петра II Князья Долгорукие приглашают племянницу Петра I - Анну Иоановну, они предложили ей подписание кондиций, это соглашение по которому вся реальная власть передавалась Верховному тайному совету, а она оставалась номинальной правительницей.

\subsubsection{Третий дворцовый переворот}
Произошел в день коронации. Совершили его новые дворяне.
Татищевы и Черкаские, Прокопович, Толстые. Анна Ионовна, несмотря на фаворита Бирона, тем не менее проводила политику на продолжение петровских преобразований. Вернула столицу обратно в петербург. Восстановила сенат. Ликвидировала верховный тайный совет.
В два раза увеличила расходы на содержание флота и армии. Передала власть сыну своей племянницы - Ивану VI(2 месяца, 1740-1741 гг.). Формально регентом при нем была Анна Леопольдовна. Реально провителем стал Бирон.

Елизавета. Новые дворяне (Братья Орловы). Продолжает политику Петра I по всем направлениям. При ней новые дворяне получают самое большое число привиллегий. Оставила престол племяннику Петру III Федоровичу. (1761-1762). Настроил против себя все российское общество тем что отдал Пруссии все территории, которые были получены в ходе 7-летней войны.

\subsubsection{Вывод}
Независимо от того какая дворянская группировка приходила к власти, каждая из них получала какие-либо особые права, привилегии. В результате к концу эпохи дворцовых переворотов дворянство трансформируется из служилого в привиллегированное сословие и окончательно их привиллигерованный статус закрепит жалованная грамота дворянству, которая будет дарована им Екатериной II в 1785 году.

\section{Реформы XIX в.}
\subsection{Период правления Павла I(1796-1801),
         Александра I(1801-1825), Николая I(1825-1855)}

Базовое противоречия между капитализмом и феодализмом решено не было. И к 50-ым годам эта проблема усилена следующими факторами

1) Экономический кризис в 30-х годах XIX в. в России начался переход от ремесленных мануфактур к крупному фабрично машинному производству, но поссесионных и крепостных(приписных) крестьян не хватало чтобы работать на фабриках, свободных рук не было.

2) Социальный кризис - I половина XIX в. постоянный рост эксплуатации крестьян. Попытки Павла I ограничения баршины 3 днями в неделю провалились. В основном по стране норма - это 6 д. в нелелю, в черноземных районах крестьяне были переведены месячину => рост социального недовольства крестьянства.
К середине XIX в. в 50\% - 60\% губерний происходят крестьянские бунты или волнения.
Рост общественной активности и появление массовых общественных движений, вплоть до радикальных, которые начинают призывать не только к отмене крепостного права,
но и к свержению самодержавия.

Внешнеполитический кризис.
Россия проиграла крымскую войну(1853-1856) => внешнеполитический долг, потеря статуса международной арене.

\subsection{Великие буржуазные реформы второй половины XIX в.}
Александр II (1855-1881)

\subsubsection{Аграрная реформа}
19 февраля 1861 - манифест о крестьянах, вышедших из крепостной зависимости:
\begin{enumerate}
    \item Крестьяне сразу получали личную свободу и признавались субъектами права.
    \item Вся земля признавалась собственностью помещика, и предполагалось что крестьянин захотят выкупить землю и заключат выкупную сделку.
    \item Пока они не выплатили выкуп, они считались временнообязанными, которые по прежнему за землю должны были платить несть барщину и оброк.
\end{enumerate}

Выкупная сделка считалась заключенной, если крестьянином единовременно выплачено помещику 20\% от стоимости земли, оставшиеся 80\% помещику выплачивало государство а крестьяне должны были вернуть это государству в течении 49 лет - 6\% годовых.

\section{Выкупные платежи}
Стоимость земли для крестьян была увеличена примерно в 10-12 раз, так как в нее была включена компенсация за барщину и оброк.

Опасались бегства крестьян, государство сохранило крестьянскую общину и значительно усложнила выход из нее. Таким образом отмена крепостного права не означала отмену феодализма, т.к. феодализм частновладельческий сменился государственным феодализмом => реформа не решила базовых противоречий общества и не способствовала развитию капитализма.

Все реформы носили противоречивый характер, вызвали рост социального недовольства и появление радикальных народническиз террористических движений.

На Александра II совершили 6 покушений. Убит в 1881 году представителем организации "народная воля" накануне подписания Конституции по проекту Лорис-Меликова

\subsection{Александр III(1881-1894)}
Перешел к политике контрреформ, усиление и укрепление абсолютизма и ликвидации капиталистических явлений в обществе.

Россия в конце XIX-начале XX вв. от реформ к революции почему провалились реформаторские альтернативы?

Экономические реформы конца XIX в.

Большую роль сыграл министр финансов Сергей Юльевич Витте. Он понимал что необходимо развивать промыщленность и начал \emph{форсированную индустриализацию}.

Цели: развитие капитализма
\begin{enumerate}
    \item Налоговая реформа 1892-1897
    \item Ввел очень много государственных монополий
    \item Ввел огромное количество косвенных налогов
    \item Ввел золотое обеспечение рубля
    \item Ужесточение контроля за сбором налогов
\end{enumerate}
Активное привлечние иностранных инввестиций(поощряет смешение российских-иностранных предприятий и все отечественные предприятия трансформируются в акционерные общества)

В итоге к началу XX в. 68\% иностранного капитала в России.
Государство вкладывает деньги в развитие крупной промышленности
Российский рубль вошел в пятерку наиболее твердых валют мира
Благодаря этой реформе в 1890е годы в России завершился \emph{промышленный переворот}(начался в 1830е).

Развитие промышленности в следствие этих реформ приобретет следующие особенности:

\begin{enumerate}
    \item Высокая концентрация производства. Преобладание крупных предприятий с числом рабочих более 1000 человек)
    \item Высокая роль государства в экономике
    \item Очень быстрое создание монополий, но в отличие от других стран, где государство ограничивает развитие монополий. В России они поддерживаются государством.
    \item Низкая покупателбная способность населения, постоянные экономические кризисы перепроизводства
\end{enumerate}

С.Ю. Витте предлагал проект капитализации сельского хозяйства, который предлагал решение осовных проблем российской деревни.
\begin{enumerate}
    \item Малоземелье
    \item Высокие выкупные платежи
    \item Сохранение крестьянской общины
\end{enumerate}

С.Ю. Витте предлагает отменить общину, простить недостатки по выкупным платежам и продать пустующие помещичьи земли крестьянам. В начале XX в. сильное влияние на Николая II прибобрели консерваторы.

И под влиянием Победоносцева Царь отказывался проводить аграрную реформу Витте и принял манифест ``О незыблемости общинного земледелия''.

Таким образом, Витте смог провести реформу промышленности, но не смог провести модернизацию деревни.

Следовательно, его реформы носили половинчатый характер, до конца экономические проблемы не решили - \textbf{главная причина революции 1905-1907 года}

\begin{enumerate}
    \item Провал реформаторской альтернативы Витте. (проблемы в с/х)
    \item Национальный вопрос. В конце XIX в. Александр III перешел к политике насильственной русификации. Политика насильственного распространения среди нерусских народов русского языка, культуры, православия и запрет местных религий и культур => рост национальных движений. Станет одной из движущих сил революции 1905 г.

    \item В 1903 Россию догнал мировой кризис, рост безработицы, ухудшение положения рабочего класса, радикализация пролетариата, что выразилось в массовом росте численности радикальных партий.

    \item Внешнеполитическая 1904-1905 проиграна русско-японская война => революция 1905-1907 года. В результате которой у России появилось еще 2 альтернативных развития:
        \begin{enumerate}
            \item Парламентская альтернатива
            \item Социально-экономические реформы столыпина
        \end{enumerate}
\end{enumerate}

17 октября 1905 года - был принят ``Манифест о совершенствовани государственного порядка'', который вводил в России свободу слова, печати, собраний, свободу союзов, объединений и политических партий и объявлял о начале выборов в Российский парламент, в Государственную Думу.

В апреле 1906 г. были приняты ``Основне законы российской империи'', которые являются первой российской конституцией, которая формально закрепляла ограниченную парламентскую монархию, но реально в россии сложилась дуалистическая монархия. Но реально в россии сложилась дуалистическая монархия (это стиль монархии, при которой монарх контролировал 2 из 3 ветвей власти). Независимой стала судебная власть. Создавался высший орган исполнительной власти - Совет министров, но император получил право назначать, распускать и формировать состав правительства по собственному усмотрению. Законодательный орган двухпалатный парламент. Верхняя палата - Государственный Совет - полностью формировался и утверждался императором.

Нижняя Государственная Дума формировалась и утверждалась выборным путем из депутатов политических партий. Она формировалась на 5 лет, но император получал правао распускать думу и назначать новую.

В России начался процесс создания политических партий программы которых обсуждали следующие 4 вопроса:

\begin{enumerate}
    \item Форма правления
    \item Форма государственного территориального устройства
    \item Аграрный вопрос
    \item Рабочий вопрос
\end{enumerate}

\section{Политические партии}

\subsection{Консервативные партии}
``Союз русского народа''(1905 г.) Лидер Николай Дубровин в 1907 г. партия раскололась на 2 части
\begin{enumerate}
    \item  ``Союз русского народа'' Н. Дубровин
    \item ``Союз Михаила Архангела'' Пуришкевич.
\end{enumerate}

Имели одинаковую программу. Их называли черносотенцы(произошло от боевых отрядов ``Черная сотня'', которые в период подавления революции в 1907 г. разыскивали революционеров, прославились еврейскими погромами)

\subsubsection{Форма правления}
Соединили теорию официальной народностии идеи славянофилов,
Государственная дума - не нужна.
Идеал - ограниченная земская сословно представительная монархия.
\subsubsection{Форма территориального устройства}
Сохранение унитарной империи
Продолжение политики русификации
Черта оседлости для евреев
\subsubsection{Аграрный вопрос}
Cохранение общины, отмена выкупных платежей и недоимок по ним.
\subsubsection{Рабочий вопрос}
Рабочий вопрос не решали.


\subsection{Либералы}
Умеренное
``Союз 19 октября''(октябристы) Лидер Гучков. Крупные буржуазные чиновники

\subsubsection{Форма территориального устройства}
Унитарное государство, но прекратить политику русификации
\subsubsection{Аграрный вопрос}
Отмена общины, переселенческая политика и продажа крустьянам пустующих государственных земель. Поддерживали реформу столыпина.
\subsubsection{Рабочий вопрос}
Разрешить забастовки и стачки, но не на государственных предприятиях, разрешить рабочим профсоюзы.


\subsection{Либералы(кадеты)}
Мелкая буржуазия.
Расслоились на правых и левых кадетов

\subsubsection{Форма правления}

Правые(Милюков) идеал - парламентская республика, но переход мирным путем. А пока ее не ограниченная конституционная монархия.

Левые(Струве) - парламентская республика, допускали возможность буржуазно-демократических реформ.

\subsubsection{Форма территориального устройства}

Унитарное государство, но автономия Польши и Финляндии. Развитие и сохраниние национальных культур.

\subsubsection{Аграрный вопрос}
Провести конфискации помещичьих земель и продажа ее крестьянам по низким ценам

\subsubsection{Рабочий вопрос}
Принять трудовое законодательство, сократить рабочий день, обязательные выходные и страхование.

\subsection{Радикалы}
РСДРП - Формально 1897 год - I съезд, реально 1903 г. - II съезд

Ближайшая цель буржуазно-демократическая революция, свержение самодержавия

Раскололись из-за программы ``минимум''
Большевики считали что буржуазно демократическую революцию нужно переводить в социалистическую, т.к. союзником пролетариата будет трудовое крестьянство.
Меньшевики считали что революция будет через 70 лет.

\subsubsection{Меньшевики}
Плеханов, Марков

\subsubsection{Большевики}
Ульянов

\subsubsection{Форма территориального устройства}
Создание федетивного государства, право наций на смоопределение.

\subsubsection{Аграрный вопрос}
Аграрный вопрос: национализация земли и раздача крестьянам.

\subsubsection{Рабочий вопрос}
Принятие трудового кодекса, введение 8-ми часового рабочего дня.


\subsection{Партия социалистов-революционеров(эссеры)}
Образовались в 1902 г. в Женеве на основе народников.
В Чернов, партия крестьян.

Главный вопрос аграрный.
\subsubsection{Аграрный вопрос}
Они предложили программу социализации земли: конфискация земли у помещиков, передача ее в общину, распределение по потребностям, сохранение коллективного принципа ведения хозяйства.

\subsubsection{Рабочий вопрос}

Делились на правых(Чернов) - Парламентские методы борьбы
Левые Парламент - нет, только террор.

\section{Революции 1917}
27 февраля 1917 года произошла Буржуазно-Демократическая революция.
Неожиданная революция, и поэтому ни одна политическая партия оказалась не готова взять власть в свои руки. Формируется двоевластие

\subsection{Петросовет} 27 февраля 1917 г. Петроградский совет рабочих, крестьянских и солдатских депутатов (Петросовет). В состав вошли умеренные социалистические партии меньшевики(Чхеидзе) и эссеры(Чернов и Керенский).
Большивики не вошли, так как на момент совершения революции большая часть партии за границей.

\subsection{Провал умеренно социалистической альтернативы?}
Следуя своим идеологическим и партийным установкам они считали что россия не готова к социализму, так как с начала должен сложится капитализм и прийти к кризису. Следовательно власть надо отдать буржуазному временному правительству. Которое будет строить капитализм под контролем социалистических партий.
Могли бы сохранить власть, так как сразу же опубликовали приказ 27 февраля №1, который отменил дисциплинарную власть офицеров.

\subsection{Временное правительство} 28 февраля - 2 марта. после отречения царя от престола образовалось временное правительство. В его состав изначально вошли буржуазные партии кадеты и октябристы. Первый состав временного правительства возглавил князь Львов, либеральная альтернатива
Главные вопросы революции:
\begin{enumerate}
    \item Вопрос о власти
    \item Вопрос о завершенности революции
    \item Вопрос о мире
    \item Вопрос о земле
    \item Вопрос о собственности
\end{enumerate}

Временное правительство считали что революция в России завершена, но само правительство не в праве решать основные вопросы революции.
И должно отложить решении всех вопросов до созыва учередительного собрания.

18 апреля Милюков публикует воззвание "нота Милюкова" - Россия будет продолжать войну до победного конца.

\subsection{Апрельские тезисы}
Следствием публикации этой ноты - Первый кризис временного правительства.
Первое коалиционное временное правительство, в состав этого правительства войдут из Петросовета меньшевики и эссеры.
В результате этого партии меньшевиков и эссеров дискредитируют себя и теряют поддержку населения
и вследствие этого появятся третья радикально социалистическая альтернатива, В. И. Ленин "Апрельские тезисы", которые станут программой партии большевиков.
\begin{enumerate}
    \item Большевизация советов
    \item Вся власть советам
    \item Революцию буржуазно-демократическую переводить в социалистическую.
\end{enumerate}

Так как союзником пролетариата в этой революции является беднейшее крестьянство
Так как 9\% рабочих и 60\% крестьян - это уже большинство населения страны, следовательно предпосылки для социалистической революции сформировались

2 июля большевики первый раз попытались реализовать свою альтернативу. Началось немецкое наступление на фронте и солдаты петроградского гарнизона поддержали большевиков и начали наступление.

\subsection{Третий кризис временного правительства. Генерал Корнилов}
Вооруженные выступления были подавлены и было сформировано второе коалиционное временное правительство в его состав вошли умеренные социалистические партии. Его возглавил Керенский.
\textbf{По прежнему отказываются решать основные вопросы революции и откладывают их до созыва учередительного собрания.}
И в результате этого в Августе 1917 появляется четвертая "Правая" альтернатива, которую предлагает генерал Корнилов "военная диктатура". Он предлагает навести в стране порядок, запретить все радикальные партии и выслать из Петрограда Большевиков. Провести новую мобилизацию для быстрого окончания войны. И потом передать власть учередительному собранию. Его поддерживает либеральная партия кадетов, его поддерживает Керенский.
27 августа Керенский обвиняет Корнилова в попытке государственного переворота и военного мятежа. Керенский чтобы остановить корнилова вооружает отряды красной гвардии - петроградских рабочих, которые после подавления Корниловского мятежа становятся на сторону большевиков.

Таким образом к сентябрю месяцу все остальные альтернативы проваливаются себя дискредитируют и большевики становятся реальной политической силой
берут курс на вооруженное восстание.
25 октября 1917 года великая октябрьская социалистическая революция в результате которой к власти приходит партия большевиков.

\subsection{Россия в первые годы советсвкой власти}
В ночь когда большевики захватили власть
25 октября - 27 октября работал II всероссийский съезд советов. Первоначально на него собрались представители всех социалистических партий, кроме левых эссеров. Они покинут съезд и тем самым полностью отдали реальную власть большевикам.
Принимаются документы о легитимном статусе власти.
Предлагают решения основных проблем
\begin{enumerate}
    \item Декрет о мире без аннексии и контрибуций
    \item Декрет о земле. Конфискация и национализация всей помещичьей земли и бесплатную раздачу земли крестьянам по потребности
    \item Декрет о праве нации на самоопределение, по которому Россия провозглашается федерацией, с правом свободного входа и выхода их состава федерации
    \item Об установлении фабрично заводского контроля.
    Получат впоследстиве общее название ``декларация прав трудящегося и эксплуатируемого народа''
    \item ``Об органах государственного управления''
    По которому провозглашается совесткая федеративная республика и советская власть, как власть центральных и местных рабочих, солдатских и крестьянских депутатов. Формируется система управления Совесткой республики.
\end{enumerate}

В состав первого советского правительтсва кроме большевиков войдут левые эссеры.

\subsection{Центральная власть}
Сроки? Всероссийский съезд советов. (законодательная + исполнительная + судебная)

В промежутках между созывами съезда формируется реальный орган власти - ВЦИК.
Первым председателем становится Л. Каменев. Потом Свердлов. М. Калинин.

В. И. Ленин - глава совесткого правительства, после него им станет Н. Рыков. Формируется Совет народных коммисаров (СНК) - советское правительство(исполнительная+законодательная)

\subsection{Местная власть}
\begin{enumerate}
    \item Губернские советы
    \item Городские советы
    \item Сельские советы
\end{enumerate}

\subsection{Этапы укрепления большевиков у власти}
\begin{enumerate}
    \item 27 октября 1917 года выходит указ запрещающий действие всех буржуазных партий
    \item 7 декабря создается ВЧК - всероссийская черезвычайная комиссия по борьбе с саботажем и контреволюцией. Она получает права черезвычайной или внесудебной юстиции.
    \item 5 января открывают и закрывают учередительное собрания.
    \item Март 18 года подписывается брест-литовский мирный договор с германией. Позорный мир. Россия теряет Польшу, Финляндию и Укранину. Большевики сохраняют свою власть.
    \item В Июле 18 года принимается первая совесткая конституция 1918 года, которая получила название конституции диктатуры пролетариата. Так как главной задачей было объявлено подавление и уничтожение эксплуататорских классов, через диктатуру пролетариата.
    \item Июнь - июль 1918 года - Левые эссеры выступают против Брест-Литовского мирного договора, их обвиняют в попытке лево-эссеровского мятежа и в июле 1918 года выходит закон о запрете всех оппозиционных в том числе социалистических партий, в следствие чего устанавливается однопартийная система.
    \item Большевики официально объявляют о начале красного террора. Таким образом, к июлю 1918 года в советской России формируется однопартийная система, которая действует на основе террора и черезвычайной юстиции.
\end{enumerate}

\section{Гражданская война в России}
\subsection{Причины}
\begin{enumerate}
    \item Разгон учередительного собрания и подписание Брест-Литовского мирного договора, привели к появлению белого движения.
    \item В мае-июне 1918 г. была введена продовольственная диктатура, согласно которой все излишки сельскохозяйственной продукции подлежали изъятию, за отказ от уклонения был введен 15ти летний уголовный срок. Для реализации этой политики из большевиков были собраны продовольственные отряды, которые были направлены в деревни и в деревнях были организованы комитеты деревенской бедноты, которые осуществляли эту политику на местах. Следствием этой политики становятся массовые крестьянские выступления и появление неорганизованного движения зеленых.
    \item Большевики отказались платить по царским долгам и кредитам и национализировали все иностранные предприятия. Следствием этого решения становится иностранная интервенция с целью вернуть потерянные капиталы.
\end{enumerate}
\subsection{Этапы гражданской войны}
\begin{enumerate}
    \item март-ноябрь 1918 года - оборонительный этап "оборонительный до начала мировой революции в европе"
    \item март-июнь 1918 года - начало иностранной интервенции, объединенные войска трех стран Англии Франции и США одновременно высадились в Мурманске и Владивостоке
    май 1918 года - начинается восстание чехославацкого корпуса
    \item июнь-июль 1918 года - начинаются лево-эссеровские мятежы
    \item \textbf{Мероприятия большевиков по этой ситуации:}
    \item 23 февраля 1918 года - создана рабоче-крестьянская красная армия
    приглашают бывших ``военспецов'' из царской армии
    \item в июне 1918 года создается рев воен совет республики, во главе с Троцким. И он же становится главнокомандующим красной армии. Страна объявляется ``военным лагерем''
    \item июль 1918 года - создается совет рабоче-солдатской обороны. На время войны становится высшим государственным органом. Главой этого органа становится Ленин. первое решение = переход к политике военного коммунизма. Главный результат этого этапа - смогли остановить иностранную интервенцию и подавить лево-эссеровские мятежи.
    \item  Ноябрь 1918 года, июль 1919 года - наступательный этап. В ноябре 1918 года начинается революция в германии. Большевики пытаются перенаправить войска с востока в европу, чтобы поддержать европейские революции. Провал, большевики не успевают вмешаться.
    \item июль 1919 - февраль 1920 года. Оборонительный и максимальное участие белых, красных и зеленых
    \begin{enumerate}
        \item Генерал Юденич начинает наступление на Петроград, большевики останавливают это наступление в сентябре 1919 года
        \item Генерал Колчак начинает активное наступательное движение в Сибири и на Урале. Провозглашает себя верховным клавнокомандующим в России. И Большевики останавливают его в октябре 1919 года.
        \item Деникин начинает наступление на Дону и Кубани. Большевики останавливают его в ноябре 1919 года.
        \item Врангель наступает в Крыму. в декабре 1919, январе 1920 года большевики его останавливают.
    \end{enumerate}
    \item февраль 1920-декабрь 1921 - наступательный, цель этого этапа принести революцию на штыках в европу и бывшие части Российской Империи. Для реализации этой цели Тухачевский предприимает поход на Польшу, который заканчивается поражением большевиков. После этого они расстаются с идеей мировой революции. Устанавливают советскую власть на Кавказе и в Средней Азии.
\end{enumerate}

    \subsection{Причины победы большевиков в гражданской войне}
    \begin{enumerate}
        \item главная причина - политика максимальной мобилизации всех ресурсов страны (политика военного коммунизма). Суть политики: национализация всех промышленных торговых и финансовых предприятий, запрет торговли и денежного обращения введение жесткой системы распределения через карточки, введение продразверстки в сельском хозяйстве. Брали не только излишки а все, что могли забрать. Результат этой политики + одержали победу в гражданской войне, - в 1921 страшный голод
    \end{enumerate}

\section{Новая экономическая политика}
\subsection{Причины}

\begin{enumerate}
    \item Голод 1921 года
    \item Экономический кризис и страшная разруха производства после первой мировой войны
    \item Потеря социальной опоры
\end{enumerate}

Советская власть объялялась как союз рабочих, крестьян и солдат. В 1921 году все эти группы выступили против большевиков

\begin{enumerate}
\item Рабочие - выступление в Петрограде, митинги, забастовки с требованием отменить политику военного коммунизма

\item В Тамбовской губернии возникается движение Антоновщина, которое требует отмены продразверстки

\item Солдаты и матросы начинают кронштадский мятеж и выдвигают мятеж - советы без большевиков
\end{enumerate}

10 съезд ВКП(б)
Переход к новой экономической политики 1921-1928 год.
\subsection{Мероприятия НЭПа}
\subsubsection{Торговля}
Восстанавливается свобода торговли и денежное обращение
Вновь начинают платить зарплаты
1924 год. - Соколов проводит денежную реформу, по которому вводится золотое обеспечение советского рубля.
\subsubsection{Сельское хозяйство}
в Сельском хозяйстве продразверстка заменяется продналогом, его размер четко определен. Все излишки крестьяне могут оставлять себе и пускать в свободную торговлю.

В 1922 году - принимается земельный кодекс. По которому разрешается аренда земли и использование наемного труда в сельском хозяйстве. Вводятся налоговые льготы для крупных крестьянских хозяйств.

\subsubsection{Промышленность}
Создавалось три типа промышленных предприятий
\begin{enumerate}
    \item крупные - оставались государственными, работали на гос. заказ - в основном тяжелая промышленность
    \item средние - и крупные легкой промышленность, передавались на самоокупаемость и саморазвитие, разрешалась частная промышленность
    \item смешанные российско-иностранные предприятий, иностранные концессии. Получали освобождение от налогов или налоговые льготы
\end{enumerate}

\subsubsection{Результат всех этих мероприятий:}
\begin{enumerate}
    \item Остановили голод
    \item Восстановили разрушенную в годы войны промышленность
    \item с 1925 года начали первый этап индустриализации
\end{enumerate}

\subsection{Кризисы и противореция НЭПа}
Либерализация экономической жизни сопровождалась усилением контроля над духовной сферой. Так как большевики боялись потерять власть

Сразу же в 1921 году начинаются первые открытые политические процессы. Партия ``эссеров''. Идеологическое обоснование однопартийности

1922 год начинается политика репрессий и запретов в отношении православной церкви. Цель этой политики - одна идеология

1922 год философский пароход, на котором из страны высылается вся бывшая интеллигенция

Главлит - главное управление по делам литературы и искусства

Приоритет отдавался развитию промышленности над сельским хозяйством. И в результате этого начался кризис связанный с нехваткой оборудования и товаров для сельского хозяйства. Слествием становится 1925 г. кризис цен. 1927-1928 года - кризис сбыта

Следствием двух кризисов становится свертывание НЭПа и перехож к форсированными индустриализацией и коллективизацией

\section{Образование СССР}
\subsection{Причины образования}
\begin{enumerate}
    \begin Победа большевиков в гражданской войне
    \begin Приход к власти в национальных регионах социал демократических партий
    \begin Тесные хозяйственные экономические связи и невозможность национальных регионов существовать самостоятельного
\end{enumerate}

В 1922 предложено два проекта

\subsubsection{План автономизации Сталин}
Предложил план создания унитарное государство, где все национальные регионы жестко подчиняются центру.

\subsubsection{План федерализации предложил Ленин}
Он предложил идею создания федеративного государства с широкими правами субъекта.

Победила точка зрения Ленина. Формально начали план создания федеративного союзного государства.

Договор об образовании СССР. Декларация об образовании СССР 1922 год.
Этот договор подписали четыре республики

\begin{enumerate}
    \item РСФСР(Российская советская федеративная союзная республик)
    \item УСФСР(Украина),
    \item БСФСР,
    \item ЗСФСР(Закавказская федеративная республика, Армения, Азербайджан и Грузия)
\end{enumerate}

\subsection{Конституция}
Была принята конституция СССР 1924 года.
Высший органом власти объявлялся (Все)союзный съезд советов. Он созывался 1 раз в три года.
В промежутках созывался ЦИК. Двухпалатный включал совет союза и совет национальностей.
Президиум ЦИК формальным постоянно действующим органом определялся Президиум цик.

Реальная власть перешла к политбюро партии большевиков.
По конституции
\begin{enumerate}
    \item Право свободного входа и выхода всех республик
    \item Четкое разграничение полномочий между центральной и республиканской властью
    \item Республикам гарантировалось сохранение национального языка и культуры
    \item провозглашался принцип равноправия всех субъектов в составе СССР
\end{enumerate}

Но первый и второй принципы никогда не соблюдались и реально СССР стал унитарным государством, где все субъекты жестко подчиняются центральной власти

\section{Период форсированного строительства СССР в конец 20ых-начало 40х годов}
Индустриализация, коллективизация, культурная революция.
\subsection{Индустриализация в СССР}
Индустриализация это процесс создания \textbf{крупного} промышленного производства и преодоление технико-экономической отсталости от стран западной европы.

\subsection{Цели индустриализации}
\begin{enumerate}
    \item Завершение процесса модернизации
    \item Создание тяжелой военной оборонной промышленности
    \item Создание основы для механизации сельского хозяйства
    \item Расширение социальной опоры совестской власти, увеличение численности пролетариата
\end{enumerate}

\subsection{Этапы}
\begin{enumerate}
    \item 1925-1928 В НЭП. Умеренные темпы
    \item 1928-1941 Форсирование, максимальное ускорение темпов индустриализации. Причина ожидание войны.
\end{enumerate}

\subsection{Особенности процесса индустриализации}
\begin{enumerate}
    \item Очень быстрые темпы
    \item Приоритет развитию тяжелой промышленности в ущерб легкой и товаров народного потребления
    \item Переход к плановому принципу ведения хозяйства
    \item Использование исключительно внутренних источников для проведения индустриализации(коллективизация сельского хозяйства, то есть максимального перекачивания денег из деревни. Займы у населения. Стимулирование трудового энтузиазма масс(стахановское движение, социалистическое соревнование, движение ударников соц. труда), бесплатный труд заключенных гулага)
\end{enumerate}

\subsection{первые пятилетки}
Создание Госплана в 1922 г.
\begin{enumerate}
    \item 1928/29 - 33 г.(Заявили о выполнение первого пятилетнего плана уже в 32 году)(пятилетка в четыре года, увеличение плановых заданий)
    \item 1933 - 37 г. Пятилетка в четыре года
    \item 1937 - 1941 г. Завершить не успели
\end{enumerate}

Поскольку плаовые задания все время завышались и были нереальными для выполнения, то получили распространие приписки и дезинформация реальной экономической ситуации в стране

\subsection{Итоги и результаты индустриализации}
\subsubsection{Плюсы}
к 1937 преодолели технико-экономическую отсталость от стран европы.
укрепили обороноспособность страны
ликвидировали безработицу
произошел значительный рост численности пролетариата
\subsubsection{Минусы}
В стране сложилась административно-командная директивная система регулирования экономики
Вообще не затронула сельское хозяйство и следовательно модернизация завершена не была
Резкое снижение уровня жизни населения
Внеэкономические методы стимулирования труда

\subsection{Коллективизация сельского хозяйства в СССР}
Коллективизация это процесс создания крупных массовых коллективных хозяйств(колхозов)
1929-1930 гг.

\textbf{Главная цель коллективизации} - изъять из деревни максимальные средства для проведения индустриализации
Увеличение численности пролетариата за счет организации массовой миграции сельского населения в города

\subsubsection{Первый этап}
Процесс коллективизации начался в середине 1920 годов и с 1925 по 1928 год носил агитационно-показательный характер
Создавались образцово показательные совхозы и колхозы и к 1929 году вошло около 1\% всего населения страны

\subsubsection{Второй этап}
Начался с 1929 года - максимальное форсирование темпов коллективизации
в 1929 году в газете "правда" - центральный орган большевиков. Статья сталина которая называлась "Великий перелом"
И в которой было обозначено что 80\% должны вступить в колхозы
Ликвидация кулачества как класса - цель кулаки самые зажиточные крестьяне и за счет экспроприации их хозяйств удастся создать без вложений со стороны государства экономическую базу для коллективных хозяйств в деревне
Этот процесс спровоцировал массовые крестьянские волнения
в 1930 году Сталин публикует еще одну статью - Головокружение от успехов и в которой он снимает ответственность с высшей партий.
Местная власть еще больше усилила форсирование темпов коллективизации
Провалы посевных компаний 1932-1933 следствием стал голод, следствием стал массовый побег в города.

Чтобы остановить массовый поток миграции в 1934 году в СССР вводятся паспорта с обязательной пропиской. Крестьянам паспортов не выдают, а без паспорта устроиться на работу не возможно. и таким образом их насильно прикрепляют к колхозам

В 1935 году принимается "колхозный устав"
Вообще полностью ликвидируется частная собственность на землю
Колхозы должны арендовать эту землю у государства. И за аренду поставлять большую часть произведенной продукции.
Устанавливалась норма выработки для колхозников - трудодень, за то что государство предоставило ему приусадебный участок.

в 1937 году торжественно объявили что процесс коллектвизации завершен и в колхозы вступило 93\% процента крестьян.

\subsubsection{Плюсы}
Были получены деньги на проведения индустриализации
\subsubsection{Минусы}
\begin{enumerate}
    \item Уничтожен частный сектор в сельском хозяйстве и потеряны экономические стимулы к труду
    \item Cократилось производство зерна
    \item Произошло резкое замедление темпов развития сельского хозяйства и продовльственная проблема станет одной из главных проблем ссср вплотьдо 1991 года.
    \item Принятие колхозного устава фактически означало реставрацию государственного феодально обшинного строя
\end{enumerate}



\section{Культурная революция}
Политика советсвкого государства с 1919 до 1941 года направленная на распространение новой пролетарской культуры и формирования нового типа советского человека

\begin{enumerate}
    \item 1919 год - компания по ликвидации безграмотности в рамках которой были созданы на всех местах региональные комитеты ликбезы. Образование населения любыми методами.
    \item 1919/20 - формирование пролетарской интеллигенции. Все бывшие дворяне, священнослужители..... лишались права получать высшее образование
    Была организована по созданию рабфаков - рабочих факультетов
    \item 1932 году были запрещены все независимы творческие объединения и союзы
    вместо них были созданы союзы писателей. Все члены этих союза должны работать по методам соц. реализма. То есть реализм не значит правда, изображать советскую действиетльность не такой какая она есть на самом деле. А такой какой должна быть согласно идельным представления о социализме и комунизме. И следствием этого постановления становится глобальнейшая мифологизация общественного сознания в СССР.
    \item 1938 - компания перестройки сознания, компания по созданию нового советского человека. Компания по массовому распространению коммунистической идеологии. Выход краткого курса истории, который редактирует лично Сталин и в котором происходит полная фальсификация всего исторического процесса на основе формационного подхода и на основе неизбежности победы большевиков в 1917 году.
\end{enumerate}

В результате всех трех процессов государство установило жесткий контроль над экономической духовной и политической сферами жизни человека. В следствии чего в СССР сформировался тоталитарный политический режим.4

\section{Внешняя политика СССР в 20х - 90х годах}

\subsection{Внешняя политика в 20х-40х годах}
Главная задача внешней политики - вывод страны из международной изоляции после гражданской войны и интервенции.

Генуэская конференция 1922 года, которая была посвящена проблемам царских долгов россии, и возмещение убытков россии в результате конференции(провалилась)

Следствием этой конференции стало подписание Рапальского мирного договора между СССР и Германией.

Германия первая признала СССР.
1924-1933 года - череда дипломатических признаний
1933 год последний двухсторонний мирный договор c СССР подписывает с США.

\textbf{Вся остальная политика носила противоречивый, двойственный характер}
С одной стороны:
\begin{enumerate}
    \item Создание системы коллективной безопасности в Европе
    \item 1934 год СССР вступает в лигу наций
    \item 1936 создается ось Берлин-Рим
    \item 1937 создается ось Берлин-Рим-Токио
    \item в 1937 году СССР предложила Франции и Англии программу ограничения агрессоров
\end{enumerate}

С другой стороны
Вторым направлением политики была поддержка коммунистических и социалистических партий для подготвки мировой революции:

\begin{enumerate}
    \item 1922 был создан коминтерн.
    Франция и англия отказываются подписывать договор с СССР
    \item 1937 год провал политики коллективной безопасности
    \item в 1939 году Англия и Франция подписывают с Германией соглашение "Мюнхенский сговор"
    \item Захват части чехославакии
\end{enumerate}

Следствием этого становится поворот внешней политики, целью этого направления становится отсрочка войны, для этих целей 23 августа 1939 года СССР подписывает соглашение с Германией, пакт Молотова-Рибентропа.

Это соглашение содержит секретные протоколы о разграничении сфер влияния СССР и Германии в начале войны.

Зоной интересов СССР признается восточная Польша, Бессарабия, Прибалтика и Финляндия.
1 сентября 1939 Германия нападает на Польшу и начинается вторая мировая война

В сентябре 1939 года СССР вводит войска в восточную польшу и возвращает территории западной Украины и Белоруссии, которые были утеряны по мирному договору с Польшей в 1921.

В сентябре-октябре 1939 году СССР включает в свой состав Латвию, Литву и Эстонию.

В июле 1940 года - бессарабию, которая становится Молдавской ССР

В ноябре 1939 года Россия начинает совестко-финскую войну, которая заканчивается в марте 1942 года с присоединением к России Выборга и Карелии

\subsubsection{Общий итог этого периода}
СССР не могла создать антигитлеровскую коалицию и осталась один без союзников.
Расширил территории и значительно отодвинул границы от центральных регионов СССР.

\subsection{Внешняя политика СССР в 1940-1960 годы холодная война}
Основы новой системы международных отношений были заложены на Ялтинской 1943 г. и Постдамской конференции 1945 г.
Раздел сфер влияния между СССР и США и Юридическая система для перехода к биполярной системе международных отношений
В зону влияния СССР переходила вся восточная Европа Польша, Венгрия, Чехословакия, Румыния, Болгария, Югославия, Албания, Восточная Германия(ГДР) + Китай, Южная Корея и Куба)

\subsection{В США система капитализма}
Вся западная Европа и западная германия(ФРГ)
1945 США проводят ядерную бомбардировку Хиросимы и Нагасаки.
1946 Речь Черчилля в Фуллтоне, который предложил доктрину сдерживания социализма с помощью ядерного оружия - идеологическая основа начала обоснования холодной войны.

Параллельно с этим 1948-51 план Маршалла - план экономического восстановления западной Европы. В результате доллар становится ведущей международной валютой.

В ответ на это в 1949 СССР создают СЭВ(Совет экономической взаимопомощи) и в рамках него СССР выделяет деньги на восстановление Восточной Европы.

1949 год СССР испытывает Ядерную бомбу, следовательно устанавливается паритет в системах вооружения и в ответ на это соединенные штаты Америки создают НАТО - североатлантический союз или североатлантический альянс, в этот блок входят 11 государств США, западные европы, ФРГ и Канада.

В ответ на это в 1955 году. Система социализма создает ОВД(организацию варшавского договора), который является военным блоком социалистических государств.

1949 год - начинается холодная война. Гонка вооружений - бесконечное наращивание ядерных оружий. Противостояние блоков через череду локальных конфликтов

1950 год самые известные конфликты холодной войны
1950-1953 год - Корейская война которая закрепит ``феномен разорванных государств''
1958-1961 год - западно-берлинский конфликт
1961 - сооружение берлинской стены между ФРГ и ГДР
1962 - Карибский кризис - Кеннеди Ядерные военные базы в Турции
                        СССР - куба
Следствием карибского кризиса становится соглашение между СССР и США всеми силами избегать начала ядерной войны, потому что она уничтожит весь мир, но сдерживать друг друга при помощи бесконечной гонки вооружений.

\subsection{1960-1970 год СССР}

\subsubsection{Капитализм}

1960е - гонка вооружений
1970е - Политика разрядки или ограничения ядерного вооружения
1972 год - договор ПРО
1973 - договор ОСВ1 - договор об ограничении и сокращении ядерных вооружений
1974 - договор ОСВ2 - договор о сокращении вооружений средней дальности
1974-1976 - ограничние испытания ядерного вооружения

Политика разооружения закончилась в 1979 году, после ввода Советских войск в Афганистан

1979-1985 - новая волна гонки вооружений
1985-1990 годы - Горбачев предложил доктрину ``Нового политического мышления'', согласно которой мир не может быть биполярным и должна проводится политика на основе единой системы ценностей
Вследствие этого он начинает разрушать биполярную систему ценностей
1987 году подписывается Берлинское соглашения, которое предусматривает ликвидация ОВД и НАТО. Горбачев распутил а НАТО существует в неизменном виде
1987 о РСМ - сокращение ядерного вооружения средней и малой дальности
1987 Горбачев подписывает соглашения, по которым он обязуется с 1989 начать советских войск из Афганистана и Восточной Европы начиная с 1989.
Следствием этого становится развал системы социализма переход этих стран в зону влияния США и фактически это означает что СССР в 1989 году проигрывает холодную войну.

\subsubsection{Социализм}
Доктрина Брежнева - очень жесткая политика, вмешательства во внутренние дела государств
1968 год - Пражская весна, в расках которой Чехославакие выступила за демократизацию и обновления социализма
В ответ были введены ОВД войска, восстание было подавлено.

1979-1983 год - события в Польше. В польше вводится черезвычайная ситуация. В польше вводится постоянный контингент советских войск.

\end{document}
