\documentclass[a4paper]{article}
% Русский язык
\usepackage[utf8]{inputenc}
\usepackage[russian]{babel}
% Математические символы и формулы
\usepackage{mathtools}
% Разбиение на разные файлы
\usepackage{subfiles}
% Отступы с краев(margin)
\usepackage[margin=1in]{geometry}
% Форматирование заголовков
\usepackage{titlesec}
\usepackage{indentfirst} % Красная строка

\begin{document}

\section{Введение}

1 семинар - 1 тк (Лекции)
Д/з 1!, 3 вопросы (Проанализировать
политику первых князей по признакам гоударства)
\subsection{Литература}

\begin{enumerate}
\item Методичка 2208
"Отечественная история", Иваново 2014
\item Бумажные из библиотеки
    \begin{enumerate}
    \item Орлов и Георгиев "История России для тех. вузов"
    (Плюсы: вся фактура в одном месте, Минусы: Марксистский формационный подход)
    \item Личман "История России для тех. вузов"
    (Плюсы: есть концепции, Минусы: слабая фактура)
    \item "История России с др. времен до 19 в.", Кафедральный учебник
    (Говно, но нужно брать)
    \item "История России 1917 - 1945", Кафедральный учебник
    (Более приличная часть)
    \end{enumerate}
\item Найти(совет) "История России в схемах и таблцах"
\item Электронные учебники
    \begin{enumerate}
    \item "История России с др. времен до 19 в.", эл. ресурс
    (Посмотреть структуру)
    \item "История России 1917 - 1945", эл. ресурс
    \end{enumerate}
\end{enumerate}

\subsection{1 ТК}

Автор(Персона)
Названиние Концепии
Суть Концепции(Основной вывод по проблеме)
Базовые термины
Даты (к концепциям, принципиально важные)

\section{Основы методологии(как проводить историческое исследование)
исторической науки}

\subsection{Методика(этапы) исторического исследования}

\begin{enumerate}
\item Анализ исторический источников
\item Изучение научных парадигм, концепций, разных точек зрения на эту проблему
\item Анализ первого и второго, если надо ищите дополнительные исторические факты
\item Формулируете собственную аргументированную точку зрения
\end{enumerate}

\subsection{Исторические источники}
Исторический источник(созданный в ту эпоху, которую вы изучаете) - это любой памятник прошлого, который содержит исторический факт или информацию об историческом событии.
Группы исторический источников

\begin{enumerate}
\item материальные - археологические раскопки
\item письменные - повесть временных лет
\item изобразительные
\item лингвистические
\item кино, фото, фоно документы (XIX-XX в.)
\end{enumerate}

\subsection{Научные парадигмы}
Парадигма - концептуальная схема, модель, изучение исторических источников и трактовки исторических событий.
Выделяют пять базовых научных парадигм

\subsubsection{План}
\begin{enumerate}
\item Авторы
\item базовые факторы исторического развития
\item термины
\item модель исторического развития
\item историки исслеователи, представители этой парадигмы
\item +
\item -

\end{enumerate}

\subsubsection{Религиозная(идеалистическая)}
\begin{enumerate}
\item Фома Аквинский V в.н.э.
 провиденциализм - все события происходят по божеьей воле.
\item линейная и конечная
\item
\begin{enumerate}
    \item В.Н. Татищев - 18 в. (Основатель российской исторической науки),
    \item Н.М. Карамзин
    \item С.Н.Соловьев
    \item В.О. Ключевский
\end{enumerate}
\item Так как выразителем божественной воли является правитель, то историческое исследование будет выглядеть как история правления отдельных князей, царей и императоров.
\item - Недоказуема, так как основная на вере. Субъективна и однобока, так как изучает только один фактор - политику правителя
\item + Богатейший фактический материал по политической истории

\end{enumerate}

\subsubsection{Формационная}
\begin{enumerate}
\item Альт. названия:
ОЭФ - общественно-экономическая формация, \\
Марксистская, \\
Материалистическая, \\
"Советская", \\
\item
    \begin{enumerate}
    \item Фридрих Энгельс 40е годы 19 в.
    \item К. Маркс
    \end{enumerate}
\item Социально-экномические социальные факторы,
\item Классы, классовая борьба(очень сильное сужение предмета исследования) \\
    ОЭФ - ступень этап в развитии общества, который отличается от других этапов своим базисом(экономическим строем), который определяет особенности надстройки
\item 5 базовых экономических формаций. Схема Ступеньки:
    \begin{enumerate}
        \item Первобытная
        \item Рабовладельческая
        \item Феодальная. Базовый фактор: земля. Ручной труд. Натуральное хозяйство. Базовые классы: феодалы, крепостные крестьяне. \\
        Феодализм в России 1649 г. Начинается с соборного уложения Алексея Михайловича.
        \item Паровой двигатель => Капитализм. Преход от одной ОЭФ к другой осуществляется в результате социальной революции(изменение базовых классов общества). \\ Изменение в средствах производства, новые технологии. Факторы производства: труд и капитал, в результате этого происходит переход к машинному индустриальному производству. Классы: буржуазия, пролетариата.
    \end{enumerate}


    Комментарии к схеме:
    По Марксу преход к комунизму возможен при двух условиях. \\
    Два пути перехода: Кризис капитализма, или абсолютное большинство класса пролетариат.((Социализм)). Комунизм. Отсутсвие классов, всеобщее равенство. Формирование общественной собственности на средства производства. Нет небходимости в существовании государства. Базовый принцип: от каждого по возможности каждому по потребности.
    \begin{enumerate}
    \item Если капитализм себя изживет, прийдет к кризису, появятся новые технологии
    \item Если пролетариат составит 70 процентов населения страны, тогда он имеет право на революцию и установление диктатуры пролетариата.
    \end{enumerate}

    В. И. Ленин дополнил это учение учением о социализме, как осбой переходной стадии от капитализма к комунизму, цель которой увеличение численности пролетариата, завершение индустриализации. Т.к. в октябре 1917 года пролетариата в России 9\%, и следовательно объективных предпосылок для социалистической революции в россии не было.

    Основатели советской исторической науки
    Б.А. Рыбаков \\
    "Экономика определяет все" К. Маркс \\
    В СССР формационная концепция стала доминирующей и единственно возможной и на этой модели основывалась вся советская историческая наука
    \item
    Идельно подходит для анализа стран Западной Европы
    Попыталась создать универсальную модель, которая анализировала бы закономерности исторического процесса
    Продуктивно изучать общество на основе экономики
    \item
    Не подходит для стран востока и России, следовательно не является универсальной.
    Практически не изучали влияние других факторов на исторический процесс
    Т.к. эта модель стала доминирующей в СССР, произошли фальсификации, искажения исторического процесса, так как история России в эту модель не вписывается.
\end{enumerate}
    \subsubsection{Этногенетическая}
    \begin{enumerate}
    \item Л. Н. Гумилев. 1930е годы. Существуюет с 90х годов
    \item Динамика развития этноса, этногенез. Базовые термины:
        Этнос - это большая группа людей, которая отличается от других групп, четко обозначеной и осознаваемой целью своего существования. \\
        Пассионарность - это особый вид интеллектуальной энергии. Способность отдельных личностей формулировать цели развития этноса и организовывать большие массы людей для достижения этих целей.
    \item Схема:
        Появление идеи \\
        Пассионарный толчок \\
        Формулирование идеи \\
        Появление этноса \\
        Развитие этноса \\
        Субпассионарии \\
    \end{enumerate}

    Дома к семинару сформулировать плюсы и минусы данной концепции.

\subsubsection{Цивилизационная}
\begin{enumerate}
    \item О. Шпенглер, А. Тойнби, Н. Я. Данилевский
    \item социо-культурные ценности идеи, идеологии культуры
    \item Нет общего единого универсального определения понятия цивилизации. Есть только 3 критреия, которые всем понятны
    Высокий уровень развития общества
    Каждая имеет свой уникальный путь исторического развития
    Одна цивилизация принципиально отличается от другой базовой системой идей ценностей.
    \item Кружки с цивиоизациями греч вост европейская
    \item Вообще не изучаеют общество находящее на ранних стадиях развития
    Так как нету единого определения цивилизации, следовательно нет общих критериев для их сравнения. Следовательно модель вообще не изучает закономерности исторического процесса, следовательно не может являться универсальной.
    Изучать отдельно историю какой-то страны невозможно потому что н а нее постоянно влияют какие-то бесконечные внешние факторы.
    \item Сформулировать свои размышления
\end{enumerate}
\subsubsection{Новая социальная история. История повседневности. Французская школа анналов}
\begin{enumerate}
\item М. Блок и Л. Февр
\item быт, нравы, духовная атмосфера эпохи
\item Историю нельзя загонять в заданные схемы развития. -
\item Изучая изменение в повседневной жизни. Анализируют влияние на нее различных факторов социально-экономических и духовных. Следовательно это попытка провести комплексное исследование исторического процесса.
%комплекный подход к изучению истории. Влияние человека на исторический процесс. Совокупность факторов.
\item Очень большой объем материала, информации. Нету четкой методики, часто не доводятся до конца.
\end{enumerate}
\subsection{Третий шаг исторического исследования}
Сравнение источников и выводов научных школ и парадигм на основе принципа историзма. Метод анализа причинно следственных связей, который анализирует количественные и качественные изменения в каждом этапе и дает оценку причин и последствий этих изменений.
\section{План подготовки }
\begin{enumerate}
\item Политическая раздробленость на Руси.
\item Причины раздробленности (Логика от концепций)
    Социально-экономические Причины.
    Духовные Причины.
    Политические причины внутренние/внешние.
\item Сущность раздробленности(Определение, хранические рамки, сравнение с аналогичным процессом в других странах)
\item Последствия раздробленности
\item Точки зрения на раздробленность. +-. Собственное мнение.
\end{enumerate}
\subsection{Проблема происхождения государства у восточных славян}
Это одна из самых горячо обсуждаемых проблем российской истории, так как отсутсвуют прямые исторические источники по этому периоду.
Единственный источник это повесть временных лет, это источник XII века. А повествует о событиях IX века. И в этом источнике приводится легенда о призвании варягов.
Славяне по причине междоусобиц в 862 году пригласили варягов - "Русь"
Рюрик начал княжить в Новгороде. Трувор в Изборске. Синеус на Белоозере.
А Оскольд и Дир пошли на греки и начали княжить в Киеве

Все ученые которые обсуждают эту проблему обсуждают следующие вопросы
\begin{enumerate}
\item Было ли призвание варягов и кто они такие
\item Как трактовать термин "Русь"
\item Кто, когда (почему) образовал государство у Восточных славян.
\end{enumerate}
\subsubsection{Норманская концепция}
Авторы этой концепции: Байер, Миллер, Шлецер. Немецкие ученые 18 века
\begin{enumerate}
\item Да, естественно было и ПВЛ об этом однозначно свидетельствует
Варяги это нормано-скандинавы, которые проживали по северному побережью Балтийского моря
\item Русь это самоназвание того племени варягов, которое пришло.
\item Государство образовали 862 г. приглашенные норманы во главе с Рюриком, т.к. у них уже на тот момент уже существовало государство и они принели этот опыт с собой чтобы прекратить междоусобицы.
\end{enumerate}
Выводы этой концепции сразу приобрели политический характер и стали использоваться врагами россии для обоснования неполноценности славянской нации(сами даже не смогли организовать государсво).
\end{document}
